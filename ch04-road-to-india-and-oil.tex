
UNDER FATH ALI'S SUCCESSORS (Mohammad, 1834-1848; Nasiri ed-Din Shah, 1848-1896; and Mozaffar ed-Din, 1896-1907); Persian weakness turned to apathy. Although at the outset of Mohammad Shah's reign we were still fighting for the province of Herat which belongs to us, we were destined to abandon it finally and to recognize the kingdom of Afghanistan merely as the result of a British threat. And why did “Her Majesty's Government consider the occupation [sic] of the province of Herat to be an act of hostility?” Not because we, the Iranians, represented a threat to India, but certainly because Russian consuls would have opened offices in a reconquered Herat. 

From the Treaty of Paris in 1857 until 1921, our unfortunate country had no government which dared to move one soldier, grant one concession, or pass one law concerning Iranians without the agreement, tacit or otherwise, of either the British ambassador or the Russian ambassador, or of both. Our policies--if such they can be called--were developed in the two embassies, and two governments barely disguised the fact that they considered Persia to be a sort of “untouchable” servant. Their diplomatic communications were orders, which we carried out, and in the event of our showing any sign of recalcitrance, they became threats. As a stopgap, Great Britain would “invade” us: An “expeditionary force” (several hundred men) would land in the Gulf. Everything would be returned to order except when tribes like the Tchahontahi or the Tangestani of their own accord exterminated the expeditionary force. At all costs the English had to keep the road to India open. 


It was along this route that two French scientists, the geologist Cotte and the archeologist Jacques de Morgan, found evidence of deposits of petroleum. In Persia since antiquity it had been known as “Napht” and used by Zoroastrians for their sacred fires. It had long been known that the Persian subsoil, especially in the southeastern part of the country, was rich in oil. In 1872 an English baron, Julius von Reuter, obtained wide-ranging commercial concessions. His efforts to profit from them were in vain and he gave them up before losing his entire fortune. Happily, he took consolation in the creation of a press agency that is still well-known today. 

Success came more readily a few years later to the two Frenchmen. It was easier for them to find a financier who was interested in their discovery. The British minister in Persia, Sir Henry Drummond, put them in touch with the Australian banker, William Knox d'Arcy, who lived in London. 

Knox d'Arcy had a massive fortune which came from a gold mine in Queensland, Australia. A taste for adventure combined with a sound business sense persuaded him that he could double his money with black gold. On May 28, 1901, after lengthy negotiations which had been complicated by the somewhat menacing demands of the Russians, the Shah finally granted Knox d'Arcy an exclusive sixty-year concession “to find, extract, transport and commercialize natural gas, petroleum, asphalt and other derivatives of petroleum throughout the land "--with the exception of the provinces bordering on Tsarist Russia. In fact, on this occasion, Knox d'Arcy had been excessively optimistic. The sums which had to be invested were soon beyond his purse and he had to hand over the concession to the Anglo-Persian Oil company. 

It was not until May 26, 1908, that oil finally flowed at Masjid-i-Suleiman (or Suleiman’s Mosque from its proximity to the ruins of the temple). Knox d'Arcy’s name was destined from that time to take its place in the history of petroleum, although he never set foot in Persia and probably never saw a barrel of oil in his life. 

Meanwhile, on August 31, 1907, Great Britain and Russia had signed a decree which divided Persia between them. 


From 1905, the Russian revolutionary agitation, with its uprisings at Tiflis and Baku, had provoked a politico-religious movement in Teheran. The movement, which was favored by Great Britain, obligated the weak Mozzaffar ed-Din Shah to grant a paper constitution in 1906, only a few days before his death. But, apart from the election of an assembly--the election was completely dominated by feudal landowners--there were few real reforms. 

The social and political climate in Persia was a nightmare. The central government was so weak that its authority did not even extend to the whole of the capital, and the army and the police force were almost nonexistent. The few soldiers were badly paid or not paid at all and thus were obligated to take on small jobs in order to survive. In addition, they took orders from Russian officers in the North and English officers in the South. 

Strength alone ruled. This was in the hands of large landowners and the heads of local tribes who, under British control, were entrusted with the security of oil-producing operations. In the countryside it was in the hands of brigands, and thieves in the towns. 

Persia was one of the poorest countries in the world. The government had to borrow from merchants in order to entertain a foreign guest. The only individuals to prosper were engaged in trading abroad. Foreigners had been granted the right to exploit the principal national resources and services: oil, fishing, telegraphy, customs, etc. Agriculture, craft, and commerce remained in the Middle Ages. Serfdom still existed. 

Health conditions were appalling. Not only was the average life expectancy a mere thirty years, but the infant mortality rate was one of the highest in the world. Malnutrition and unhygienic conditions afflicted a people who had been a model of health and vitality. The total absence of hygiene caused typhoid, malaria, and trachoma to become endemic in some regions, while epidemics of the plague and cholera were not unusual, nor were famines caused by drought. Sadly, thanks to today's so-called government, cholera again plagues my country. 

Ignorance walked hand in hand with poverty and sickness. Fewer than one percent of the population were literate. There was only one lycée in Teheran. Women did not go to school and were deprived of all rights. All the privileges of Western material civilization which, to a certain extent, reached the Ottoman Empire, India, and our other neighbors, were practically unknown in Iran, There were no railways, no proper roads, no cars, no electricity, no telephones. Where any of these were to be found, they were considered a real luxury. Material and spiritual poverty marched side by side with the spread of corruption, deceit, hypocrisy, opium addiction, and superstition. 

This decadence may have resulted partly from the weakness and ignorance of the Iranian people, but it sprang, above all, from the impotence of the Persian authorities, from the egoism of the feudal aristocracy and from the deliberate will of foreign colonialists. Many of the British remembered Nadir and dreaded the Iranians. Also, they were putting into practice a policy of a no-man's-land between Russia and India. Like a condemned man who has lost all hope, the dying country awaited a coup de grace; and it little mattered whether it came from the north or the south. It was at this moment that a man of strength came forward: My father. 
