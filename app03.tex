By 1982 our armies would have increased from 540,000 men to 760,000. The following summary will give an idea of how our strength was distributed. 
\section*{TANKS}
1,500 “Lions of Iran.” Specially designed for us by British engineers, this tank is equal, if not superior to all others because of its new engine, its 120mm gun, its laser-beam range-finder and new armor-plating. 

800 “Chieftain” tanks with an improved engine, fitted with the same cannon and the same firing quipment. 

406 M-60 modernized tanks (U.S.) with a 105 mm gyroscopic gun. 

400 modernized M-47s, with a90mm gun which could be replaced by a 105mm. 

250 “Scorpion” reconnaissance tanks. Other tanks of this variety were due to be ordered in the meantime. 

Besides a thousand armored troop-transporters and armored mobile-control posts mounted on American M113 chassis, we had 2,000 Russian-made armored troop-transporters on tracks and on wheels. They are the best of their kind. Many of them would have had, in addition, a 73mm gun and anti-tank missiles. 
\section*{ARTILLERY}
In principle our artillery batallions were to have the same firing strength as NATO artillery, By 1982, our armament factories would have produced 105mm, 120mm, and 150mm cannons and more. 
\section*{AIR FORCE}
78 Fl4s with Phoenix missiles having a 90-mile range, and 150-mile range radar equipment, and able to fire six missiles at different targets at the same time. 

250 regularly modernized “Phantoms”; the oldest of these having laser bombs; and the most modern with a “black box” capable of warding off enemy missiles. 

Over 100 F5Es 

About 100 Fl4s (or Fl5s, depending on which the Americans decided to build). 

106 Fl6s already ordered. We were negotiating to order an additional 140. 

7 airborne radar systems, reaching to at least 35,000 feet, which would have meant that we could economize on 30 ground systems; and other airborne electronic look-out devices. 

24 modernized 747 and 707 air-tankers capable of refueling each other. We had had this air-tanker modified according to my own plans. This fleet would have allowed us to keep the maximum amount of airplanes airborne and would have saved time. We were going to order at least another twelve. 

57 military transporters of the C130 “Hercules” variety, with propellors. 

Besides these, some of our factories which were working on behalf of national defense were ready to manufacture: 

Anti-aircraft missiles of the SAM 7 type (U.S.S.R.). 

Air-to-ground “Maverick” missiles (U.S.) with extremely precise homing devices reaching as far as twelve miles. I am told that the factory at Shiraz where the Mavericks” and TOW s would have been built has been destroyed. 

Anti-tank TOW missiles. We thought that we would surpass the Americans and change from the sub-sonic radio-controlled TOW to supersonic laser team TOW. 

Anti-tank rockets (U.S.S.R.). 

“Dragon” radio-controlled rocket launchers (U.S.) for the infantry. They would have twice the usual range--i.e., 1,000 instead of 500 meters. 

Several hundred “Oerlimpon” anti-aircraft guns. 

We were also developing a new twin barrel 35mm anti-aircraft gun which was better and simpler. 

In addition to this 35mm gun, we were also developing a 20mm anti-aircraft gun. 

100 “Rapier” anti-aircraft missiles, both towed and selfpropelled. We had the capability to manufacture these in our own country. 

We already had three airborne brigades and, by 1982, we would have had five, which meant more than a division. 

\section*{THE NAVY}
Our expectations were as follows: 

Four 8,000-ton cruisers, standard sea-to-air missile launchers witha speed of Mach 3, and sea-to-sea “Harpoons’ with a subsonic speed whose normal range would have been increased from 90 to 150 kilometres thanks to a relay helicopter. We were also planning to make the “Harpoons” supersonic, and a submarine launch was under study. 

Twelve 3,000-ton destroyers armed with standard “Harpoons” ordered from Holland and Germany. 

12 “Combattant IIs” (France). 

3 U.S. submarines, already ordered. 

9 submarines to be ordered from Europe, probably from West Germany. 

50 naval helicopters. 

A fleet of troop ships, tankers, supply ships. This naval force could have not only cruised in the Gulf, but could have reached the farthest shores of the Indian Ocean. 

Some "Orion” (Lockheed) long-range reconnaissance planes which would have been used by the navy while remaining answerable to the Air Force. 