MY ACTS HAVE OFTEN been criticized, sometimes justly. However, few of my critics have stopped to objectively evaluate the difficulties which we had to overcome. Fewer still have considered whether Iran would exist at all without the efforts which we made for her survival. 

We strengthened Iran's independence and unity in 1945-46; we pulled the country out of chaos in 1953. We next put our economy and finances in order; we wrested our oil resources from foreign ownership; and from 1963 we set our people, with their overwhelming approval, upon the road of common sense and progress, toward the Great Civilization. 

For 37 years all my political activities were carried out with the aim of placing my people upon the path leading to this Great Civilization. When I began my White Revolution, a shock program which would allow Iran to overcome in 25 years its centuries of suppression, I understood that its realization would not be possible except through a mobilization of all forces within the country. A permanent state of urgency was necessary if we were to prevent hostile elements from becoming stumbling blocks--elements such as reactionaries, large landowners, communists, conservatives, and international agitators. To mobilize a country, one must win it over, push it, pull it and while it is engaged in work, defend it against those who want to prevent it from working. 



To let saboteurs act freely would certainly not have permitted this program's realization. Without the White Revolution, democracy in Iran would be merely a mirage, for a democracy based on hunger, ignorance, and physical and moral degeneration is but a caricature, and ultimately democracy's worst enemy. 

Indeed, the road to this Great Civilization was not an easy one. But it led toward a higher standard of living. What then is this Great Civilization that I wanted for Iran? To me, it is an effort toward understanding and peace which creates the perfect environment in which everyone can work. I believe each nation has the right, the duty to reach or to return to a Great Civilization. That is why Iran cannot but be faithful to its ancestral, universalist tradition. This tradition in fact always combined certain values and a certain purely national Iranian spirit with the best available in other civilizations. 

In our march toward this Great Civilization, Iran was one vast workshop in which all the elements indispensable to modernization sprang up: universities, school groups, professional institutes, hospitals, roads, railroads, dams, electric plants, pipelines for gas and oil, factories, industrial, cultural, artistic and sports complexes, cooperatives, metropolitan areas, and new villages. 

When Mossadegh was in power, Iran’s budget was around \$400 million. Our last budget was \$57 billion, of which approximately \$20 billion came from oil revenues and the rest from taxes that the people could now afford; in 1963 our per capita income was \$174, in 1978, the last year of my reign, it was \$2,540. And all this was accomplished at a time of great population growth, from 27 million in 1968 to 36 million in 1978. And our social programs were developing under the White Revolution at a remarkable rate. 

In pursuing these economic and social programs we still depended heavily on our oil revenues, which were around \$22 billion in 1977. Despite our care to conserve our oil, we continued to extract five to six million barrels a day from our wells, in order to create the infrastructure which was indispensable to our development in the Twenty-first Century. 

According to all forecasts, three or four years would be sufficient to fill our most important gaps. Between 1978 and 1982, our institutions of higher learning would have graduated four classes of various technicians. Steel production would have reached 10 million tons per year. I 


was hoping that we could later increase our level of production to 25 million tons, which is near that of France. Sixty kilometers from the Pakistani border, the vast port complex of Chahar-bahar on the Sea of Oman was nearing completion, as was the one in Bandar-Abbas on the Persian Gulf. Here ships weighing 500,000 tons could be placed on dry docks. Other large ports under construction were to have opened in 1982. 

Beyond this, we planned the construction of four electro-nuclear plants, Iran I and Iran II to be built by the Germans near Bushehr and Iran IIT and Iran IV by the French on the Karoun River near Ahvaz. The first two were to begin to operate in 1980 and 1981 and the second two were scheduled to open at the ends of 1983 and 1984. At present, these projects have been abandoned. The sums invested have without doubt been lost without any return. Fourteen other nuclear stations were planned which would ultimately supply Iran with 25,000 megawatts of nuclear energy. A nuclear research center was planned for Isfahan. 

We had also scheduled the construction of the Teheran subway, the electrification and doubling of the Teheran-Bandar Shapour railroad, and the construction of a six-lane highway along the same route. 

The benefits of so many years of effort are now reduced to nothing. Our oil production is far from having regained its former momentum and--it has been declared in Teheran--it never will. The wells are managed by incompetent people, perhaps untrained workers, and oil exploitation is carried out so poorly that some wells have already been ruined. 

The Communists and their allies have gradually seized the workers’ and peasants organizations. It is, moreover, the rank and file who give orders, dismiss or appoint directors, engineers, and supervisors. They make decisions about production and vote themselves enormous salary increases. 

Since it has been forbidden to fire any employee of a business, sabotage can only be prevented by closing down plants or by “nationalization, which means allowing the nation to pay for a deficit made greater by a myriad of new bureaucracies, incompetence, and corruption. In certain cases the owners themselves have asked for “nationalization.” 

During the summer of 1979 the industrial infrastructure of the country practically collapsed as a result of all this--notably the large steel, copper, and aluminum industries, mines, docks, and car and tractor manufacturers. Before they disappear forever, these industries will cost the taxpayers a great deal. 

Most factories lack both orders and primary materials, and can operate at only 25-30 percent capacity while they await the closing of their doors. Needless to say, the workers have lost all the advantages which the White Revolution bestowed upon them. Besides, there is nothing left to share. The working classes have indeed been profoundly hurt by the “Islamic Revolution.” It was made against their interests. I had given the greatest possible number of citizens the power to rise and free themselves from proletarian conditions. It is with despair that I see workers, peasants, and employees without work, falling again into financial difficulty and sometimes into misery. 

The rapid depreciation of money and galloping inflation make it practically impossible to establish reasonable programs of production even on a short-term basis. The net cost of a manufactured object is impossible to ascertain. 

In public works and construction, stagnation has been the rule since the winter of 1979. The State has renounced almost all the major projects, and the multinational enterprises do not want to carry out any activity under a regime which has cancelled past orders without paying the contractual penalties. 

Of the 180 French firms which were working in Iran in 1978, I understand that half had already withdrawn by March 1979. All the foreign firms that remained have had to work under very difficult conditions, the enterprises having to receive “revolutionary committees” with claims that are impossible to satisfy. 

The new Iranian authorities let it be known that the indemnities for broken contracts would not be paid to foreigners. Revolution and change of regime now fall under the category of “unforeseen circumstances which are an absolute necessity.” 

The present political and religious chaos and xenophobia are no less costly to American, German, Italian, Japanese, and other firms which have lost important orders for materials or work. Dupont de Nemours (U.S.A.) and Mitsubishi (Japan) invested and lost considerable capital in our country. The greatest loser is the United States. We signed a commercial treaty with the United States which foresaw their selling us ten billion dollars of materials and equipment during a five year period. It is now defunct. 


The moral aspect of this catastrophe is no less grave. The consequences of the bloody fiasco of Khomeini may prove to be disastrous for Islam in general and Shi'ism in particular. The megalomania of the troubled mind of Qom and the dictatorship of a band of mullahs are in formal contradiction to the essential principles of Islam. History will reveal how a head of state anxious about the future of his people, a man calling for solidarity among other peoples, was to be ostracized. My exile has permitted, or will permit, certain people to take up the so-called policy of expensive oil for their own benefit and to their own profit. 

My disappearance from the world political and economic scene has coincided with a general offensive against the economic and even political stability of Western powers. This crisis, moreover, is only one of several means employed to destabilize not only the Middle East but the world economy. 

It is said that today the events in Iran have altered the rapport of world forces. It is said that “times have changed” and that the common man or citizen has to adapt himself to the new state of the world.” I am afraid that signifies he must adapt himself to chaos. 
