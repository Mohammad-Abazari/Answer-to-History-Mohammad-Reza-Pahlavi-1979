THE BLACK AND RED alliance that would prove so destructive goes back far in time and is deeply rooted in Iran's consciousness. The black-- the clergy--had opposed my father, and supported my rule only sporadically. Religious fanatics who did not understand the true nature of Islam had allied with the Tudeh back in the 1940s. Our investigation of my would-be assassin, Fakhr Arai in 1949, attested to this confederation. Mossadegh’s government demonstrated how such a red-black alliance could thrive without a clear understanding by the noncommunist members of its consequences. In the early fifties, the Tudeh Party was merely biding its time, waiting for Mossadegh to oust me so that it in turn could safely eliminate him. We know this beyond question from documents found in Tudeh’s offices after Mossadegh’s sudden fall. The situation is no different today. The communists are waiting for Khomeini to lead my country into chaos, poverty, and despair before taking over. 

The media, the major oil companies, and the British and American governments were the other ingredients in this strange amalgam. My success in negotiating a 75/25 oil royalty agreement with Enricco Mattei of E.N.I. in 1957 enraged the international oil cartel. Politically, my action showed a greater degree of independence than the Americans and British were willing to tolerate. The British, of course, had always considered Iran as a fiefdom and tried to dominate our national life. The Mossadegh misadventure had brought the Americans onto the scene and given Washington an incentive to intervene. 

The results of their combined displeasure became obvious in the late fifties. The first student demonstrations against my regime broke out then, interestingly enough, in the United States. In 1962, while at a ceremony in San Francisco, I saw a plane overhead with a streamer proclaiming “need a fix, see the Shah.” Clearly, this was part of an organized effort to discredit me and my government. I cannot help but believe that the oil companies and an organization like the CIA were somehow involved in fomenting and financing this campaign against me. This effort acquired a professional polish over the years that students could not have achieved on their own. It is hard to believe that the KGB was quite that effective in the United States in those days. 

For the next twenty years students and media echoed the same anti-Iran themes intermittently, whenever the West felt my wings needed clipping. The late fifties and early sixties were one such period. The years after the oil embargo and the hike in world oil prices were another. 

The early sixties were a particularly turbulent time for us. They coincided with the advent of the Kennedy administration and increased U.S. intrigue against our country. Repeatedly, I tried to warn Kennedy of the oil crunch I saw ahead. But the President scoffed at my concerns. At one of our meetings he expressed confidence that even if all Mideast oil were shut off, the U.S. had large enough reserves to meet world needs for two years. It was difficult to argue against such deluded self-confidence. 

In Iran this period was marked by student and teacher demonstrations, by the temporary closing of Teheran University, and by renewed National Front agitation. The U.S. pressed me to name Ali Amini, Mossadegh's Minister of Economics and by then an opposition leader, as prime minister. Sharif Emami, my Prime Minister at the time, warned me against U.S. intrigues but I did not believe him. In May 1961 I gave in and appointed Amini to head the government. Fifteen months later--a time marked by economic and political mismanagement--Amini resigned. Even the Americans had lost faith in his abilities. In better control of the situation, I was finally able to act and to inaugurate the White Revolution, thus winning a decade's respite from my foreign “friends” and domestic enemies. 

Well, let us say I won a partial respite. Terrorism has been a feature of Iranian political life for a long time. No matter how hard I tried, and God knows I tried, my country was never quite rid of it. I have already discussed the 1949 attempt on my life and the 1951 murder of my Prime Minister, General Razmara. In late 1955 assassins shot and wounded my then Prime Minister, Hussein Ala. As I previously discussed, I was the victim of a second assassination attempt on April 10, 1965. A young soldier shot his way into the palace. A subsequent investigation showed that he had been part of an extreme left-wing plot masterminded by university people. 

On January 21, 1965, Moslem fanatics shot and killed another of my Prime Ministers, Hasan Ali Mansur. Later, several of my generals were killed, and in the early seventies terrorists murdered three American colonels in the streets in Teheran. This mayhem was not only directed at the prominent or public figures. I remember other victims, a taxi driver and a car washer, for example, shot and killed trying to disarm terrorists. 

The media, of course, never let an opportunity go to play up acts of violence and to make them reflect badly on my rule. Sometimes they would go to ludicrous lengths. In late 1959 President Eisenhower visited me in Teheran. As is the custom for visiting heads of state both in monarchies and democracies, I had stationed an honor guard along the route from the airport to the palace. The newspapers reported that the Shah feared so much for his life that he put al! his soldiers into the streets to protect him. On another occasion in the sixties when we made our annual move to our summer palace on the Caspian Sea, the Western press reported that the Shah has fled Teheran and taken refuge in the North. 

Nothing I did could stop these preposterous stories, not the interviews I gave to every foreign journalist who spent any time in my country, not the frequent press conferences open to all media, not my regular appearances on American television, both on trips to the U.S. and from Teheran. Perhaps I was too confident in what I said and how I said it, so at times I may have appeared arrogant. But all I ever wanted was that foreign journalists understand Iran--its history, its problems, its future. Unfortunately, most media people came with preconceived ideas about what Iran was and should be. Too often they based their thinking on Western values. Nevertheless, I always encouraged debates, which permit you to express yourself, and always enjoyed provocative questioning. Unfortunately, the media continually viewed my country through Western eyes. They never looked at Iran as a Middle Eastern nation bordering on the Soviet Union and with a vastly different cultural and religious value system. Thus, they unfairly categorized my regime as repressive. Small wonder that Iranian students abroad, exposed to a steady diet of vilification of their homeland, would also demand that Iran become what it was not, and could not become in so short a time. 

The media campaign grew more virulent after the oil embargo and the price hike. The statements of certain American government leaders did not help. Thus, Secretary of the Treasury William Simon called me a “nut” and the media picked up this controversy with glee. Simon's insults led the media to label me as the man responsible for expensive oil. But nobody bothered to explain my reasons for raising oil prices as I detailed earlier. 

As a result of this calumny I can well understand the anger of Western drivers as the cost of gasoline rose and shortages appeared. They were simply told it was all the Shah's fault, while few bothered to tell them how much their government's tax revenue rose as a result. 

Media attacks, terrorism, student agitation, the beginnings of Western pressure to liberalize my regime, all began to converge on my government in the mid-seventies. Curiously, the clergy were largely silent in those years. Some Moslem extremists dabbled in the left-wing ideologies responsible for part of the terrorist violence, but they were not in the clerical mainstream. When I became Shah, I swore an oath to uphold the constitution and to defend the faith, the Shiite religion of the Twelve Imams. In turn, the clergy recognized me as sovereign and as Defender of the Faith. I am a religious man, a believer; I follow the precepts of the sacred Book of Islam, precepts of balance, justice, and moderation. Nevertheless, there were elements of the clergy during my reign who opposed my reforms and my efforts to modernize Iranian society. We have already seen the violent form that opposition took on occasion, for example during the agrarian reforms of the early sixties. But as late as spring 1978, during my annual pilgrimage to Meshed, I was not only given a warm welcome from the local population, but hundreds of clerics there showed me respect, loyalty, and affection. And this was at a time when some of the mullahs had already joined the subversive moment. 

The first signs of organized opposition to my rule came toward the end of 1976 from liberals, left-wingers, and people of wealth and power inside my country. Meanwhile, I had already allowed the International Red Cross, the International Association of Jurists, and Amnesty International to review our criminal justice system. I readily asked for and accepted their comments, criticism, and suggestions. We paid a good deal of attention to some of their recommendations. Needless to say the media reported the alleged abuses in great detail but paid little heed to the changes we made as a result of these missions. 

Early in 1977 the terrorism that had shaken our society for so long abruptly stopped. I knew this was no fortuitous accident. Such actions are calculated and carefully thought out and I was convinced that this ominous silence was no exception. If these hardened terrorists who routinely had sprayed the streets with machine-gun fire and thus randomly killed innocent men, women, and children had ceased these activities, something else must have started. True, I had “eyes and ears” among the population, but they brought me little in the way of conclusive information so that I was unable to discern the deeper meaning of this cessation of terrorism. From this developed a need for more expeditious changes. Thus, over the next two years the urgency for completion of my reasonable and viable reforms was vastly increased. In this way, I hoped to diffuse my opponents’ unreasonable and unrealistic demands for radical changes that my government's infrastructure would not be capable of withstanding. 

Their demands for “authentic parliamentary democracy” were in reality nothing more than demagoguery that would result in a caricature of democracy such as has been seen so often in discredited multi-party systems. I wanted a true democracy designed to foster my country’s real interests. But my opponents were not interested in that approach. As a result, the more I liberalized, the worse the situation inside Iran became. Every initiative I took was seen as proof of my own weakness and that of my government. 

By the summer of 1977 I had a sense of the new troubles ahead. Perhaps my own instincts were more accurate than the information my people gathered in the towns and the countryside. I felt that some political change was needed. Amir Abbas Hoveyda had been Prime Minister since 1965, the longest tenure of any of my heads of government. Hoveyda was a talented and dedicated man. But he too felt that he had been in office too long. In fact, he told me with some relief when I asked him for his resignation that no Prime Minister should serve longer than five years. Since I valued his counsel, I named Hoveyda my Minister of Court and thus was assured of his continued advice on a close, personal basis. 

In August 1977, lappointed Namshid Amouzegar Prime Minister. At the time foreign policy issues occupied most of my attention and I thought Amouzegar well suited to handle them. He had represented Iran at OPEC for a number of years and acquitted himself with distinction during the tough, arduous sessions of that body. Moreover, he had completed his engineering studies in the United States and still had good friends in that country. In Iran he enjoyed a reputation for integrity. As the Resurgence Party's general secretary, he had a political base of his own. Finally, his appointment could serve as a spring-board for a new liberalization drive that would promote democracy within Iran. I wanted to avoid a general collapse which I knew would ensue if liberal and leftist demands were realized. 

That autumn in a Newsweek interview I had the opportunity to reiterate for the public’s attention an apparently forgotten agreement between the U.S. and Iran, as well as to raise in the West's consciousness of the Soviet Union's dogged determination to dominate the world. In that interview with Arnaud de Borchgrave I discussed the bilateral agreement signed on March 5, 1959, by the U.S. and Iran. I noted that by this agreement the U.S. considered Iran's independence vital to its own national interest and therefore would furnish military assistance to Iran. The U.S. also committed itself to come to Iran's assistance if we were attacked, although this latter commitment was couched in vague diplomatic terms--that is, it merely stated appropriate action would be taken when mutually agreed upon. I also expressed my concerns about the West's determination to fight Soviet aggression. I noted that it is becoming increasingly difficult to discern just at what point the U.S. would be willing to stand up and fight. This interview also afforded me the opportunity to discuss the North-South problem and the growing debt of underdeveloped nations. I said then and still strongly believe that though the West may not be in a position to wipe out these debts, a new deal must be struck with the underdeveloped countries. Ten percent of the world’s population cannot for long control 90 percent of its riches without world upheaval. Redistribution is not the answer; creation of ‘new’ sources of wealth is the basic need of the developing nations today. Cooperation by all developed nations--East and West alike--is vital to this effort, 

Our foreign difficulties were highlighted during our visit to the U.S. later that fall. As usual on these official visits, the Empress and I spent a night in Williamsburg, Virginia, to tour the historic restorations. Several hundred Iranian students had gathered near my hotel to express their loyalty. I stopped to chat briefly with the group when I noted a smaller knot of people, most of them masked, standing around a red flag complete with hammer and sickle. They hurled insults at us. Why the masks? I would learn in the next day's newspapers that these demonstrators feared the SAVAK. A more likely explanation than that ridiculous charge, it seemed to me, was that the masks hid non-Iranian demonstrators--professional troublemakers hired on the spot. In any event, the anti-Shah protesters could not have numbered more than fifty while my supporters were close to five hundred. 

Imagine my amazement the next day when I saw the press had reversed the numbers and wrote that the fifty Shah supporters were lost in a hostile crowd. A similar reversal appeared in print during our visit to Washington. This time the demonstrations were larger and more violent. A crowd of several thousand Iranians living in the U.S. had gathered to greet us on the Ellipse south of the White House. They sat on hastily erected bleachers and had hoisted a huge white banner that said: "Welcome Shah.” A much smaller group of demonstrators, several hundred at the most, had gathered in Lafayette Park on the other side of the Executive Mansion. Again, they wore masks and carried placards explaining that "Masks Protect Us From SAVAK.” 

After we had arrived for the welcoming ceremonies on the south lawn of the White House, the “student” demonstrators charged the peaceful assembly of my supporters. They swung placard handles and planks armed with nails. The police were not prepared for the violence and gave way, using tear gas in an effort to restore order. More than 130 persons were injured before the riot was over. Again the media switched the figures. Time, for example, wrote that a thousand “opponents of the Shah” had gathered in Lafayette Park to face “several hundred of the Shah's supporters,” even though the opposite numbers were correct. Nor did the U.S. press chastise the aggressors. One newspaper wrote meaningfully, “Who, then, paid for the Shah's supporters to come to America?” Another charged that my government had paid their expenses and additionally gave each supporter a \$100 bill. There was, of course, not a word of truth to the allegations. On the other hand, no one bothered to investigate the troublemakers’ sources of support. 

I know that very few of them were Iranians. Most of them were young Americans--blonde, blacks, Puerto Ricans, together with some Arabs. Moreover, there was foreign money involved in paying their bills. I had reports at the time of a dummy foundation set up in New York with money sent from Europe and other places to pay for anti-Shah demonstrators to come to Washington. Buses had been organized to bring people from New York and other cities. And I know that many other demonstrators were flown in from as far away as Los Angeles. These were facts no one bothered to report. The press was too busy detailing my alleged human rights violations or reporting that ‘the Shah's political jails hold 25,000 to 100,000 political prisoners.” 

In all fairness, however, Time did report that I had embarked ona program of liberalization and that Prime Minister Amouzegar had moved to end press censorship, had loosened controls, and had freed several hundred political prisoners. “Ina few months,’ Time wrote in November 1977, “the police-state atmosphere has altered drastically to a mood of vastly greater individual freedom and relaxation.” This report came fifteen months before my ouster! 

My talks with President Carter had gone well. Iran's relationship with the U.S. had been so deep and so friendly during the last three Administrations--I had counted Lyndon Johnson, Richard Nixon and Gerald Ford among my friends--that it seemed only natural that our friendship would continue. After all, good relations were in the best interests of both nations. Carter appeared to be a smart man. 

My favorable impression of the new American President deepened when he visited Teheran to spend New Year's Eve with us at Niavaran Palace. I have never heard a foreign statesman speak of me in quite such flattering terms as he used that evening. “Iran, because of the great leadership of the Shah, is an island of stability in one of the more troubled areas of the world,” Carter said in his prepared remarks at dinner. He went on to say: 


“Our talks have been priceless, our friendship is irreplaceable, and my own gratitude is to the Shah, who in his wisdom and with his experience has been so helpful to me, a new leader.” 


A week later, on January 7, 1978, the first riots erupted in which the clergy were the major source of the agitation. Demonstrators surged through the streets of the holy city of Qom where thousands of pilgrims annually visit the tomb of Massoumeh, the sister of Imam Reza. There is little doubt in my mind that communist elements had infiltrated the 4,000 religious students and their supporters who took part in the protest. I am equally certain that rebellious and dissatisfied mullahs were at the center of the unrest. Six people were killed during the disturbance, a number duly exaggerated in media accounts of the incident to dozens of dead and hundreds of injured. The renewal of violence after a year’s calm troubled me. I realized that political agitation in Iran was entering a new phase, that the conspirators, whatever their origins, had changed their tactics and we were now faced with organized violence. 

As a counter to this demonstration, a crowd of 30,000 paraded through Qom some days later to express their support for my government and my rule, but it was already too late. That first demonstration in Qom gave the mullahs their martyrs and soon they were able to spread a refined and sophisticated "bereavement" technique to other cities. According to Moslem tradition, parents and friends of the deceased gather at the tomb forty days after the death. SAVAK “victims” were produced enmasse as a pretext for holding such graveside assemblies. My opponents seized corpses from grieving families and paraded them through city streets, shouting: “Here is another victim of the regime, another of SAVAK’s crimes.” Many of those carried on the shoulders of demonstrators had died from natural causes. Some were Jews whose religious laws require burial within 24 hours of death and whose beliefs were carelessly trampled by malcontents intent on fraud and violence. They had only one aim--to trigger daily demonstrations which were ever more frequent once the forty-day mourning cycle had been established. Of course, these protests were always accompanied by more violence and often by more deaths, creating a self-perpetuating cycle of destruction. 

It is hard to imagine any more vile behavior by supposed religious people than those upheavals of 1978. These insurgents often pretended to be wounded and had themselves daubed with mercurochrome to appear battered and bloody--especially when unscrupulous news photographers were on the scene. 

The next wave of violence hit the city of Tabriz in the riots of February 18, and soon reached the holy cities of Qom and Meshed. These were contained, but it took considerable force to do so, perhaps too much force. On March 5 my government disciplined some police andSAVAK officials for the way they had handled the riots. This shows, incidentally, that SAVAK was never allowed to operate unchecked in Iran. As the riots spread, underlying social issues began to emerge. The unholy alliance of red and black was beginning to solidify. Looking back, the uprisings in Tabriz marked the beginning of efforts to reduce my authority, to turn me into a weak and ineffectual “constitutional” monarch, and finally to oust me. 

True, under my rule Iran had progressed at a rapid pace. The growth of Teheran in a few years froma city of one million to a metropolis of 4.5 million could not be accomplished without some social and economic dislocation. There were plenty of jobs. When I left my country, a million foreign workers were employed at all levels of the economy. (Under Khomeini's government, all the foreign workers have left and more than four million Iranians are unemployed.) Housing construction could not keep up with demand. Thus, real estate prices rose to dizzying heights as they have in other major cities gripped by housing shortages, and many fortunes were made. Luxury apartments dotted the cityscape. My push for construction of low-income housing lagged because production of brick and cement could not keep pace with our needs. This problem was compounded by bottlenecks at our ports; imported cement spoiled at dockside. 

But the situation had not yet slipped out of my control. Popular support for the crown ran strong and deep. In spring of 1978 I traveled to the holy city of Meshed. The warmth of my welcome was overwhelming. I drove in an open car at not more than five miles an hour while crowds pressed close to me. My security was minimal]. There could have been a sniper behind any window. My reception in Meshed was not an isolated incident. When Prime Minister Amouzegar visited the provinces several weeks later, his reception was equally tumultuous. In Tabriz, so hard hit by the February riots, 300,000 turned out to cheer the head of the government. 

Nelson Rockefeller visited me that spring. Times were ominous, not only in Iran. Across the border in Afghanistan, the communists had quietly taken control of a neutralist government, with few in the West aware or concerned. For ten years I had been warning the West about Afghanistan, her strategic importance, and her unstable politics; but my warnings were ignored. It is worth noting here that had I remained in power, the Russians would not have dared invade Afghanistan. By January 1980, Iran would have had a standing army of 700,000 men equipped with the most modern weaponry available and supported by F-15 and F-16 fighter-bombers. It is very doubtful that the Red Army would have challenged this force. 

This was one of the issues Rockefeller and I discussed. We were old friends. He was no longer in office and could talk freely. “Is it conceivable,” I asked him pointblank, “that the Americans and the Russians have divided the world between them?” “Of course not,” he replied. And added, At least as far as I know...” 

In mid-March disturbances had swept through several cities including Teheran. However, the next two months proved quiet. I began to hope that the worst was over. In mid-May the deceptive calm shattered. Students went on strike. The bazaars closed. The links among students, merchants, and clergy tightened. Much has been made of bazaaris Opposition to my rule and their relationship with the mullahs. 

Some Western accounts have alleged that the CIA had been paying between \$400-450 million annually to Iran’s clergy and that in 1977 Carter ordered this assistance ceased. I have no knowledge of any CIA support of our mullahs. For some time my government had been providing our clergy with substantial support. In 1977 due to the exigencies of our economy, Prime Minister Amouzegar was forced to eliminate these payments. Western analysts have theorized that these actions precipitated an organized rebellion of the clergy. Other stories allege that the bazaaris revolted against my government because we had moved to impose price controls, a vital effort to curb inflation, and because we had pushed to develop Western-style supermarkets and shopping malls at the expense of the bazaars. 

In retrospect, I regret Amouzegar's actions in these areas. But I cannot subscribe to these Western postulates that they were prime factors in the overthrow of my government. Too many other factors were involved. 

The fight against inflation, for example, dated back to the mid-seventies. As we have seen, the crackdown although justified, entailed hardship. In August 1975, some 8,000 price gougers were tried in our courts. We had hired students to check on merchant compliance with price guidelines. We had counted on their incorruptibility, but many of them may have been too zealous. Others were probably saboteurs who saw in these jobs an opportunity to sow dissension and foment trouble. Our price controls were imposed across the economy; the bazaars were not singled out. 

Bazaars are a major social and commercial institution throughout the Mideast. But it remains my conviction that their time is past. The bazaar consists of a cluster of small shops. There is usually little sunshine or ventilation so that they are basically unhealthy environs. The bazaaris are a fanatic lot, highly resistant to change because their locations afford a lucrative monopoly. I could not stop building supermarkets. I wanted a modern country. Moving against the bazaars was typical of the political and social risks I had to take in my drive for modernization. 

As I continued to liberalize the opposition grew increasingly vocal and organized. Unfortunately, the press focused not on their actions but on SAVAK. 

The very worst crimes have been attributed to SAVAK. It has been alleged that millions of Iranians were employed by it. In fact, at the end of 1978, SAVAK employed less than 4,000 people. 

The name SAVAK comes from the initials in Persian of the Organization for State Security and Information. Similar organizations exist worldwide since every country is obliged to protect its populace from subversion. Accordingly they are called KGB, CIA or FBI, Intelligence Service, M15, or SDECE. 

Is it necessary to add that Iran has no more reason to tolerate terrorism than the Italians have to tolerate the activities of the Red Brigades or the Germans the demands of the Baader Meinhof gang? And when in Germany, six prisoners commit suicide all on the same day, by aiming bullets into the backs of their necks to be sure of success, barely any surprise is shown at their having firearms in their cells. This series of altogether peculiar coincidences is regarded as perfectly normal. World public opinion accepts the story without a frown. 

SAVAK was instituted in Iran to combat communist subversion after the disastrous Mossadegh episode. It is not for me to judge the attitude adopted by Western countries towards their communists. However, a common frontier with the Soviet Union tends to sharpen one’s perceptions of their activities. 

Although I managed to have neighborly relations with Russia, as well as advantageous economic cooperation, we did pass through difficult times immediately after the war. Remember that the Soviet troops who occupied our country in World War II only left it in April 1946 and that the Tudeh party really thought its hour had come during the last months of Mossadegh’s rule. Thus, we were obliged to outlaw this party which threatened not only the regime, but the country’s territorial integrity. 

SAVAK was created, then, to put an end to these subversive activities. The organization was first headed by General Bakhtiar in 1953, and he called in the CIA to advise him. Subsequently many SAVAK officials went to the U.S. for training by the CIA. They also went to Great Britain and other Western countries to observe their operations. 

General Bakhtiar kept this position until 1962, when his ambitions and inquisitorial methods interfered with his effectiveness in office. He was exiled and several years later he was assassinated in Iraq. 


There were in Iran, as elsewhere, traitors, spies, agitators, and professional saboteurs, about whom our government and military leaders required intelligence. This was SAVAK’s responsibility. As an information and counterespionage organization it originally served civil magistrates. But as a result of recommendations made by international human rights groups, this function was left to the ordinary police force. 

The officers of SAVAK were trusted soldiers who recruited agents from the army, the police, and university graduates. But the majority of its staff were civilians. Although this force at one time alone handled the questioning procedure in our judicial process, by late 1978 on the advice of international lawyers, this was modified so that defense lawyers participated in the proceedings. 

SAVAK'’s activities were exaggerated by insurrectionists to further incite the nation. According to “informants” the number of “political prisoners” and people being “tortured” in our prisons fluctuated between 25,000 and 100,000. Now in Chronicles of the Repression, a clandestine paper printed in Iran and used by the opposition against SAVAK, it was specified that between 1968 and 1977, that is over nine years, the number of people arrested for political reasons was exactly 3,164. 

Our Prime Minister was directly responsible for the day-to-day operations of SAVAK. As head of state, I could only intervene at the request of the Minister of Justice to exercise the Right of Pardon over condemned men. However, when I learned that torture and abuse existed, as a matter of policy I ordered it stopped. 

I was deeply moved when I heard that before being tortured and assassinated, Mr. Hoveyda, the former Prime Minister, and the former heads of SAVAK, Generals Pakravan, Nassiri, and Moghadam had insisted that they had never received any order whatsoever from me with regard to a suspect, an accused man, or a condemned one. 

I had the power to commute sentences, which I always used as widely as possible. I signed all applications for pardon or remission of sentences which the magistrates presented to me. As for those who made attempts on my life, I always pardoned them, even against the advice of the Public Prosecutor. 

I cannot defend SAVAK’s every action and will not attempt to do so here. There were people arrested and abused. Unfortunately, this is nota perfect world. Worldwide police brutality exists. Inherent in police work is the potential for abuse and cruelty. My country, too, fell victim to such excesses. However, whenever I learned of abuse, I put an end to it. When the International Red Cross wished to investigate, the prisons were opened to their representatives. Their recommendations were followed. 

I must draw a distinction here between terrorists and political prisoners. It was inevitable that some terrorists died in confrontation with SAVAK and local police. No one forced them to start fires, to pillage, and to murder. They were the victims of their own choice. 

With regard to those who were arrested for political reasons--I cannot include arsonists and saboteurs in this category--I affirm that they were properly treated and that they were never molested in any way. No one can tell me the name of a single politician who has been “liquidated” by SAVAK. 

In June 1978 I replaced the long-time head of SAVAK, General Nassiri, with General Moghadam, who was more a philosopher than a soldier and from whose benign attitude I expected beneficial results. Again, I would be sadly disappointed. Riots shook the holy city of Meshed on July 22. Clerical unrest was growing and from Iraq Khomeini increased his attacks. Cassettes of his speeches and harangues were smuggled into our country and used by his supporters to incite the masses. 

Hoping to stem the rising violence, I announced on August 5, 1978, our Constitution Day, that free elections would take place at the end of the current parliamentary session, in the spring of 1979. The elections would culminate my steady drive to create a true democracy in Iran. Thus, the polls would be open to anyone, including those opposed to the present government. My announcement of open elections was followed by riots in Isfahan, Shiraz, and Teheran, proof positive that the opposition was not interested in a more democratic government. Their sole aim and interest was the overthrow of my regime. I am convinced that if these free elections had taken place, the country would have clearly pronounced itself in favor of the democratic government for which we had laid the foundations. It is because we were reaching this goal that an alliance of destructive forces took place against us. It is thus that one sees the representatives of certain bazaarss and a pseudo theocratic feudalism joining hands with parties and sects of the extreme left. 

The situation in Isfahan grew so violent that the government had to impose martial law there. My opponents could not afford to relax their agitation lest common sense prevail and persuade their naïve followers of the folly of their actions. No matter how often I reaffirmed my commitment to liberalization, the answer was always the same: more riots. 

On August 19, fire swept a cinema in Abadan killing 477 people. Many were burned alive, others asphyxiated. The fires were set deliberately in an act of brutal mass murder. Immediately my government was blamed for this atrocious act. Supposedly, the police had locked the doors of the cinema so that no one could escape. We were also charged with having started the fire. Khomeini needed yet another provocation to whet the appetites of his fanatic following. 

The real culprit fled to Iraq, where he was arrested. He confessed, but frightened or pusillanimous magistrates covered up the affair. The arrested man alone could say on whose behalf he committed this outrage. 

Popular reaction, fulfilled the instigators’ every dream. Rioting swept the country and raged for several days. At the height of this upheaval, General Moghadam came to see me. He had just met with one of the major religious leaders. His message for me: Do something spectacular! He had repeated that word, “spectacular” several times. 

Under the circumstances, I wondered what dramatic action I could take to save the country from chaos and destruction. Perhaps a new government might provide the answer, a government to which I would relinquish my own powers. I discussed this possibility with Prime Minister Amouzegar and he promptly offered me his resignation. I accepted--a great mistake on my part. I never should have allowed this wise and unbiased counselor to withdraw. On August 27, I named Sharif Emami, who had been Prime Minister twenty years before, to head a new government. 

He tried to disassociate his government from the past in an effort to win a new start for himself and his cabinet. He denounced the Resurgence Party, a move which failed to appease the opposition, but succeeded in cutting off that party's support for his government. Emami also tried to curry favor with Shiite clergy, by reintroducing the old Hegira calendar and ordering all casinos and gambling clubs closed. This too proved ineffectual. 

Street rioting continued. By September 8, it had reached such proportions that Emami was forced to impose martial law in Teheran. Under our Constitution, such an action must be approved by the Majlis within a week of its imposition. Accordingly, on September 10, Emami requested and received the required approval--for eleven other cities as well as Teheran. That very day he and his cabinet also were given a vote of confidence by the Majlis. That September 8 would become known as “Black Friday.” The violence of that Friday’s demonstrations reached such a pitch of murder, pillage, and arson that the security forces had no choice but to fire. The death toll of 86 was reported by the Teheran Martial Law Office on September 10, the very day the Majlis confirmed the new government. Emami stated publicly that he took full responsibility for the consequences of martial law. 

For all the bloodshed and despite all the vilification heaped on our police and soldiers, I must pay tribute to the sang-froid they showed. Uncontrolled mobs who had savagely murdered their comrades in arms failed to provoke them into equally bloody revenge. Moreover, martial law was not strictly enforced. Its imposition was little more than a warning. Riots continued and insurgents paid little heed to the troops who in fact fired only on arsonists, pillagers, or armed saboteurs. 

The Camp David meetings on the Middle East were underway as unrest mounted in Iran. I have been told that some Americans, Israelis, and Egyptians taking part in those meetings expressed considerable concern about the events in my country. Some reports suggested that Israeli spokesmen told the Americans that Iran was more important than their own negotiations. If so, I had no knowledge of it, nor did these warnings have much effect on American action. President Sadat’s concern had equally little effect. Sadat called me late on the night of September 9 and we talked for a few minutes. As always, Sadat offered his encouragement and his help. I have no way of knowing what he said to President Carter later that night. But I do know that reports widely circulated in the West about a Carter telephone call to me later that night are false. President Carter has never called me--except once at Lackland Air Force Base in December 1979. 

In early September, the American ambassador, William Sullivan, returned from holiday to his post in Teheran. In any number of interviews Mr. Sullivan has given since my exile--and his own retirement from the U.S. Foreign Service--he has said that he knew in September 1978 that I could not survive. He has told this to the International Herald Tribune and anyone else who would listen. But he never told me. For the next four months the only word I ever received from Mr. Sullivan was reiteration of Washington's complete support for my rule. To be more specific, the U.S. backed me 100 percent and hoped I would establish law and order, as well as continue my program of political liberalization. 

For the balance of the year I received numerous messages from various people in and out of the Carter Administration pledging U.S. support. Whenever I met Sullivan and asked him to confirm these official statements, he promised he would. But a day or two later he would return, gravely shake his head and say that he had received “no instructions’ and therefore could not comment. Sullivan appeared to me always polite, always grave, always concerned. He came to see me several times a week. He seemed to take seriously everything I said to him. But his answer was always the same: I have received no instructions. Is it any wonder that I felt increasingly isolated and cut off from my Western friends? What were they really thinking, what did they want--for Iran and of me? I was never told. I never knew. Sullivan, and the British ambassador, Anthony Parsons, who so often accompanied him, met in a stiff diplomatic ballet that ended without resolution of any kind. Meanwhile, I was forced to deal with what is usually described as a prerevolutionary situation. In truth, it was the end of modern civilization in Iran. 

The mosques had become hotbeds of dissent. Street riots were orchestrated by the mullahs. For the first time the slogans of Islamic Marxism were trumpeted before large audiences. The Mujahidin-e Khalq (Fighters for the People)--saboteurs trained in Lebanon and Libya--took this surprising theory to our naïve masses who listened with interest for the first time; the agitators now had the backing of the mullahs and thus the blessing of organized religion for their preposterous statements. 

How can communists, plutocratic bazaaris, and Islamic clergy join hands in the same revolution? One cannot imagine such divergent philosophies coalescing. The responsibility for this union rests upon the shoulders of those prelates who naively put their hands in those of militant atheists. At present, they feel obliged to make a higher revolutionary bid, to show themselves as being more demagogic, more pitiless, and more destructive than their “fellow travelers” whose prisoners and hostages they will ultimately become. They can only denounce in vain this fatal alliance, for they are doomed to advance on its cursed path. 

These prelates have compromised themselves. Tomorrow they will find themselves isolated. Then as is the rule in the Marxist universe, they have to surrender and no one will defend them. They will leave their mark upon history only through the crimes which they were forced to commit and for which they took frightful responsibility “in the name of God.” The tragedy is that religion will be obliterated by militant atheism, in the name of the Sovereign People and the Communist Gospels. 

The Western press continues to play the communist game--they claim that bloodshed and death that marred our cities had nothing todo with terrorists but was the work of SAVAK agents and the police. If SAVAK had only been as effective as my enemies claimed, they would not have been out in the streets shouting vilifications. By November 1978 our total prison population was only 300--this in a nation of 35 million! But the Western press continued to accept blindly Khomeint's figures of 100,000. 


By October it became increasingly clear that careful plans had been made for the collapse of Iran. In the larger towns where martial law was still in force, terrorist groups, armed with automatic rifles and explosives, battled troops and police. They moved with the stealth of classic urban guerrillas. Soon they attacked government offices and foreign embassies. Confrontation grew more ominous and more dangerous. 

On October 5 Iraq expelled Khomeini and he flew to France where he took up refuge in the village of Neauphle-le-Chateau near Paris. Should I have stopped the move and persuaded the Iraqi government to keep him? The question is a difficult one to answer. It is true the French government asked me at the time whether I had any objections to Khomeini’s change of venue. I did not, believing that he could do as much damage from Hamburg or Zurich as he could from Paris and that I lacked the power to line up the world in a solid phalanx against a frail and crazy old man. I could hear the thunder in the Western media should I attempt any action so harsh and autocratic. 

Under these most difficult and trying conditions, I continued to push reforms and liberalization to prepare the country for free elections. I moved against corruption on a broad basis. Businessmen and officials who had enriched themselves illegally were arrested and I fully intended to bring them to trial. In a little-noticed action that summer I had issued a code of conduct for my own family as insurance against slander and calumny. Censorship was lifted: Newspapers were free once again to print what they liked; Iranian television broadcast parliamentary debates during which opposition deputies made fiery antigovernment speeches. Parliament, I thought, was a much more appropriate forum in which to voice dissent than the streets of our cities. 

But the war on the streets continued. After Khomeini left for Paris, so-called students again used their “bereavement” tactics to confront the authorities. Rioting broke out in many of the provinces; demonstrators went on a rampage. Banks, ministries, and businesses were set afire, windows of schools and offices were broken, banks, shops and restaurants were looted. Police and security guards were forced to fire on the crowds in order to end this senseless vandalism and rioting and to restore peace, At the end of the first week in October, strikers succeeded in shutting down government ministries and closing airports and schools. On the 16th a crowd of 100,000 marched through Teheran to 

protest the victims of "Black Friday,” again exaggerating the number of dead and wounded. For the first time I began to detect a defeatist tone in the conversations of the British and American ambassadors. I suggested holding a pro-Shah rally. My people were sure that such a rally would be a huge show of force and strength for the Crown. The envoys shrugged and said what is the point in that? The next day the opposition will have double that number opposing you. It is a race you cannot win. I believed them then, I now know I was wrong. I could have won such a contest. 

Sharif Emami's government began to weaken. The concessions it had offered had been too few and mostly ill-chosen. The Americans began to push the idea of a coalition government. Why not bring in members of the opposition, specifically the leaders of the old National Front of Mossadegh’s day? Reaching out across political barriers, they said, could re-establish badly frayed national unity. I tried. I contacted Karim Sanjabi and several other opposition leaders. But their demands were unacceptable. Shortly thereafter Sanjabi and Mehdi Bazargan, a rich lawyer and human rights “activist” who would head Khomeini’s first government, visited the aged mullah in his French retreat. Other efforts to broaden the government's political base failed, although on the day before my 59th birthday, October 26, I freed another 1,500 political prisoners and later eliminated those undesirables from SAVAK who had been brought to my attention. 

Strikes spread--at the end of October oil workers began to walk out and paralyze our \$20 billion oil industry. Oil production began a dizzying fall from more than 6 million barrels a day to barely 1.5 million. Air Iran went on strike again. More riots swept the capital on November 5. Loudspeakers were set up on university campuses and in secondary schools to broadcast messages of hate and rage. University and high school students responded by joining “Islamic Revolution” activists in the streets. Army and police were ordered to contain the demonstrations but to shoot only if absolutely necessary. Banks, cinemas, public buildings, and hotels in the capital's western and central districts were sacked and burned, Soldiers guarding the British embassy could not stop an attacking mob and had to watch the building burn, destroying a part of the complex. The Information Ministry was besieged and sacked. 

The messages I received from the United States while all this was going on continued to be confusing and contradictory. Secretary of State Vance issued a statement endorsing my efforts to restore calm and encouraging the liberalization program. Such Herculean fantasies left me stunned. President Carter's National Security Advisor, Zbigniew Brzezinski, at least had his priorities straight. He called me in early November to urge that I establish law and order first, and only then continue our democratization program. 

That call, too, has become famous, although at least it actually happened. What followed, however, was truly bizarre. I thanked Brzezinski for his expressions of support. The next day I sought confirmation of the message from Ambassador Sullivan. As usual the American envoy promised to cable Washington, but when I next saw him, he said gravely that he had received no instructions. This rote answer had been given me since early September and I would continue to hear it until the day I left the country, 

Since then I have often been asked why I did not seek the confirmation I wanted through other channels, perhaps by picking up the telephone and calling Washington. My answer is simple. In foreign affairs, I have always observed the protocal of international diplomacy. Thus, I never tried to establish direct contact with Carter or anyone else in the Administration because that is done through an ambassador. The fact that no one contacted me during the crisis in any official way explains everything about the American attitude. I did not know it then-- perhaps I did not want to know--but it is clear to me now that the Americans wanted me out. Certainly this was what the human-rights champions in the State Department wanted, and Secretary Vance apparently acceded. I say apparently because again I was never told anything: nothing about the split within the Carter Administration over Iran policy; nothing about the hopes some U.S. officials put in the viability of an “Islamic Republic” as a bulwark against communist incursions. Instead, my ambassador in Washington, Zahedi, reported the same thing I heard day after day from Sullivan in Teheran: The U.S. is a hundred percent behind you. I was ill-served by Ardeshir Zahedi's inaccurate reporting. He had been in Washington too long and was closely identified with the Nixon and Ford Administrations. He pretended to have access to the highest authorities but his reports could never be confirmed. His outgoing temperament was unsuited to the straight-laced Carter White House and I should have replaced him. In any event, protocol was not observed and I was not told the truth. 

On November 5 Sharif Emami's position had become untenable and I asked for his resignation. There was little choice left but a military government. None of the opposition politicians I had contacted showed any interest in forming a new cabinet. In fact, few civilians did. I therefore turned to my chief of staff, General Gholam Reza Azhari, an honest, loyal man who had always avoided politics. He accepted as duty the weight of this office. He, too, was anxious to prove his goodwill to the opposition, and he immediately arrested twelve highly-placed officials, among them Mr. Hoveyda, whom he put under house arrest. He told me that only a proper trial could fairly deal with the accusations against my former Prime Minister and the other arrested men. 

I was not totally convinced of the wisdom of this assessment, but Mr. Hoveyda, who still has my wholehearted esteem, was one of the favorite targets of the opposition. Knowing that in fact it was I whom they hoped to reach through him, I suggested that he go abroad for awhile, and offered him the Belgian Embassy. Too sure of himself, or perhaps too loyal, he refused. I have mentioned elsewhere the abominable way in which he was treated before his execution. 


During the first days of the Azhari government, we still had hope. Work began again and the daily production of oil, which had fallen very low, rose again to 5.3 million barrels a day. There were favorable reactions from the people. The general strike dictated from Neuphle-le-Chateau for Tuesday, November 12, was a failure. In all the big cities (Teheran, Isfahan, Meshed, Shiraz, and Tabriz), large numbers of young men, armed with clubs, confronted the Red and Black urban guerrillas. 

But we wanted peace and the reconciliation of all Iranians. We had liberated, on four or five occasions, several hundred political prisoners; in December, General Azhari reaffirmed the government's declaration of October 19, 1978, that a full amnesty would be granted to all Iranians provided they respected the Constitution. Furthermore, we did everything possible to discourage what might have amounted to counter-terrorism by our own partisans. We freed the last political prisoners; our jails then held only those who had been convicted of murder and other serious crimes. 

All too soon it was no longer a question of an opposition conspiracy against me; all the forces of destruction were united. Modern, progressive Iran was to be annihilated, and with it, by one means or another, the representative of a dynasty which had so often saved the country from ruin. 

Then began the strikes which were to bring the country to her knees. We had power cuts lasting several hours each day; water and oil were cut off. Transport workers struck; then banks and the most important ministries were closed one after another. Combined, these stoppages paralyzed the nation. Idle crowds thronged the streets, growing all the while more bitter. The ringleaders had threatened most workers, either personally or through their families. It is well known that a mere five or six people inside a big power station can halt supplies of electricity; this is also true of the oil-pumping centers. The effectiveness of such small cadres explains the success of the strikes. In two months, strikes at the oil wells and refineries caused incalculable losses. 

Our last effort to hold the country together was on the verge of failure. General Azhari did his best, I know. He felt that he had to placate the opposition. He would rise to speak in Parliament and begin with some verses from the Koran; he tailored his speeches to fit the religious mood and soon was talking like a priest. Soon people began calling him Ayatollah Azhari. Eventually, the strain proved too much and he suffered an incapacitating heart attack. 

Since leaving my country I have often asked, and indeed wondered myself, whether stronger action on my part could have saved my throne and my country. Certainly my generals urged me often enough to use force in order to re-establish law and order in the streets. 1 know today that had I then ordered my troops to shoot, the price in blood would have been a hundred times less severe than that which my people have paid since the establishment of the so-called Islamic Republic. But even that fact does not resolve my fundamental dilemma--a sovereign may not save his throne by shedding his countrymen’s blood. A dictator can, for he acts in the name of an ideology and believes it must triumph no matter what the cost. But a sovereign is not a dictator. He cannot break the alliance that exists between him and his people. A dictator has nothing to bestow for power resides in him alone. A sovereign is givena crown and must bequeath it to the next generation. This was my intention. Under my rule, Iran had reached a certain cultural, industrial, agricultural, and technological level. I still hoped to raise these levels before abdicating, so my son, Crown Prince Reza, could reign over a nation, strong industrially, militarily, and culturally, with a growing nuclear industry and no longer as dependent on oil. 

During those final weeks I spent most of my time on the telephone with local administrators. The instructions I gave were always the same: “Do the impossible to avoid bloodshed.” 

One day the mayor of Meshed told me with some embarrassment that mobs in his city were attempting to knock down my statue. I let him know that in light of the bloodshed and mayhem facing us, the forces of law and order were not to be wasted in defense of a statue. 

Throughout the period of turmoil and violence that marked the last months of my reign, I wanted to believe that my opponents were of good faith. Did they want more liberalization? I had already effected it. Did they denounce corruption? I had not waited for their demands to act vigorously in this area. 

A lawful solution without bloodshed remained my aim. If a climate of conciliation could be established, it might be possible, still, to form that elusive coalition government with members of the opposition. And such a government might succeed in claiming the agitators and their mesmerized mobs. Most importantly, such a government could put the country back to work. 

Soon after Prime Minister Azhari’s cabinet was confirmed, I resumed discussions with opposition leaders. I first contacted Dr. Sadighi, a member of the National Front, whom I considered a patriot. He agreed without conditions to try to form a coalition, but asked for a week of reflection to which I agreed. Then, acceding to the pressures from his party, he demanded that I nominate a Regency Council while remaining in my country. This condition was unacceptable because it implied that I was incompetent to perform my duties as a sovereign. (It should be noted, however, that Dr. Sadighi was the only political leader who begged me not to leave Iran.) Mr. Sanjabi and Mr. Bazargan had, on returning to Teheran from meetings with Khomeini in France, launched a virulent antigovernment campaign. As a result, Azhari had them arrested for anti-constitutional declarations. From his prison, Mr. Sanjabi asked to see me. He chose as his intermediary General Maghadam, the head of SAVAK, who incidentally was the same man who had brought me the message from the religious leader under the Amouzegar government. For his pains, Moghadam was executed soon after the so-called Islamic Republic was founded. 

Determined to do all within my power to effect conciliation, I obtained the freedom of both Mr. Sanjabi and Mr. Bazargan after a few days of detention. I received Mr. Sanjabi shortly thereafter. When he arrived he kissed my hand, declared his personal loyalty to me with great fervor, and announced his readiness to form a government, provided I left Iran for a “vacation.” 

For him there was no question of naming beforehand a Regency Council, which was constitutionally necessary, nor of asking for a vote of confidence from the Majlis before my departure. I had to reject this solution and seek new consultations which proved to be difficult since the situation continued to deteriorate. 

Were these politicians not aware that Iran teetered on the brink of an abyss? Did they not understand that it was no longer a question of safeguarding privileges, monopolies, or the supremacy of one political party over another, but of the life or death of our country? 

Economic disorder was everywhere and this was no less worrisome than the agitation in the streets and around the universities. Strike followed strike. Oil production which normally was 5.8 million barrels a day had fallen by December 25 to 1.7 million barrels, a disaster for our economy. Deliveries of natural gas to the Soviet Union were:seriously hampered. Clearly, such political and economic chaos could not continue for much longer. 

It was at this time that I increasingly questioned my allies’ actions: Did the U.S. still hold to our bilateral agreement that obligated them to come to our aid in case we were attacked by a communist country? Did they want it annulled? But as I have already indicated the only answer the American ambassador brought back to my question was the implacable “I have no instructions” or “we support you 100 percent.” And even though Sullivan came often with Anthony Parsons, the British envoy, and both made much of the fact that at this time--unlike in the Mossadegh period--the U.S. and Great Britain spoke as one, I found in these sentiments little solace, many doubts. 

The messages I received from the Americans continued to be confusing and contradictory. What was I to make, for example, of the Administration s sudden decision to call former Under Secretary of State George Ball to the White House as an advisor on Iran. I knew that Ball was no friend and I understood that he was working on a special report concerning Iran. No one ever informed me what areas the report was to cover, 

let alone its conclusions. I read them months later in exile and found my worst fears confirmed. Ball was among those Americans who wanted to abandon me and ultimately my country. 

Indeed, it almost seemed as if the Russians were more concerned about Iran than the Americans. 

The Soviet attitude toward the upheavals in Iran were succinctly stated in an article published in Pravda late in November. In effect, it told the U.S. and the West not to interfere in our internal affairs. 

Shortly thereafter, the U.S. indirectly acceded to the Soviet position and issued an official announcement which stated that under no circumstances would the United States interfere in Iran. What had become of our bilateral agreement? When the British and American ambassadors again came to assure me of their continued support, I wondered what I was to conclude from these mixed messages. The West still urged me to continue my liberalization program, while maintaining law and order. Unfortunately, liberalization with a gun pointed at one’s head has inherent limitations. I had been in favor of greater freedoms and had implemented policies designed to speed up the democratization process. But the heedless, pell mell rush toward anarchy, rather than toward democracy, could not but lead to disaster. 

In December, pressure began building for me to leave the country. There had been discussions for several weeks about my taking a vacation abroad as a pre-condition for forming a coalition government; but as I have already noted, opposition leaders could not or would not agree to put the necessary procedures in place. Now foreign visitors appeared in Iran urging me to leave and to reach an accord with the opposition. 

At about this time, a new CIA chief was stationed in Teheran. He had been transferred to Iran from a post in Tokyo with no previous experience in Iranian affairs. Why did the U.S. install a man totally ignorant of my country in the midst of such a crisis? I was astonished by the insignificance of the reports he gave me. At one point we spoke of liberalization and I saw a smile spread across his face. It seemed a strange interest for a high-ranking CIA official, ostensibly charged with security in the Middle East. Thus, I could only assume that his mission was to further our liberalization efforts rather than discuss security issues. 

After the failure of my negotiations with Sanjabi, General Moghadam asked if I would see Mr. Shapur Bakhtiar, another member of the National Front who had served as a junior minister in the Mossadegh government. I had already had some contact with him arranged by my former prime minister, Mr. Amouzegar, who even out of office remained a trusted advisor. Unlike Sanjabi who continued to make inflammatory speeches at a time that called for calm and reason, Bakhtiar behaved in a discreet and reserved manner, I agreed to receive him. General Moghadam brought him to the Niavaran Palace one evening. We had a lengthy conversation in which Bakhtiar profusely expressed his loyalty to the Constitution and the monarchy. He wanted to adhere to the Constitution and name a Regency Council before I left on holiday. He would seek a vote of confidence in both chambers. 

It was with some reluctance and under foreign pressure that I agreed to appoint him Prime Minister. I had always considered him an Anglo-phile and an agent of British Petroleum. His political base lacked depth; he had admitted to me that the entire membership of the National Front consisted of only 27 people. 

I finally decided to name Bakhtiar Prime Minister after my meeting with Lord George Brown, once Foreign Secretary in Britain's Labor Government. We were old friends. He took my hand and pleaded with me to leave the country. Just take a two-month vacation, he said. Then he strongly endorsed Bakhtiar. On December 29, Shahpur Bakhtiar was asked to form a civilian government. 

On January 2, 1979, I made my first public appearance in two months and expressed readiness to take that much-demanded--and if truth were told much needed--holiday, once Bakhtiar’s government had been confirmed and installed. 

Shortly after the first of the year, General Robert Huyser, Deputy Commander of U.S. Forces in Europe, arrived unannounced in Teheran. On January 4, President Carter began the Guadaloupe meetings with French President Giscard, West German Chancellor Schmidt, and British Prime Minister Callaghan on that French-owned island in the Caribbean. Giscard said they hoped to “evaluate the situation of the world,” with special emphasis on events in the eastern Mediterranean and the Persian Gulf. I believe that during those meetings the French and the West Germans agreed with the British and the American proposals for my ouster. These Guadaloupe meetings may prove to be the “Yalta of the Mideast,” with the notable absence of the recipient (U.S.S.R.) of the largesse. 


About the same time French President Valery Giscard d'Estaing sent a personal envoy to Teheran, a man very close to him. He too advocated apolitical” solution to the crisis, a euphemism for accommodation and abstention from the use of force. His second point, I still have difficulty understanding today: at all costs we must avoid any confrontation with the Soviet Union. I asked him what my internal affairs had to do witha confrontation between East and West, but it was a question he would not answer. 

It was against this background that I learned of Huyser’s mission. I had known him reasonably well. He had come to Teheran a number of times, scheduling his visits well in advance to discuss military affairs with me and my generals. I had always found him very helpful. 

The unannounced visit distressed me. I asked my generals about it but they knew no more than I did. Such a man would not avoid me without good reason. As soon as Moscow learned of Huyser’s arrival, Pravda reported: “General Huyser is in Teheran to foment a military coup.” In Paris, the International Herald Tribune wrote that Huyser had not gone to Teheran to “foment” a coup but to “prevent” one. 

Did such a risk exist? I do not believe so. My officers were tied to the Crown and the Constitution by an oath of loyalty. As long as the Constitution was respected, they would not falter. 

But the different American information services had perhaps solid reasons to think that the Constitution would be abused. It was therefore necessary to neutralize the Iranian army. It was clearly for this reason that General Huyser had come to Teheran. 

I saw Huyser only once. He came with Ambassador Sullivan about a week before I left Iran. Sullivan did most of the talking. The atmosphere was grim. My departure was no longer a matter of days, Sullivan said, but of hours, and looked meaningfully at his watch. Both talked of a “leave” of two months but neither seemed very convinced that I might return. 

Huyser succeeded in winning over my last chief of staff, General Ghara-Baghi, whose later behavior leads me to believe that he was a traitor. He asked Ghara-Baghi to arrange a meeting for him with Behdi Bazargan, the human-rights lawyer who became Khomeini's first Prime Minister. The General informed me of Huyser’s request before I left, but I have no idea of what ensued. I do know that Ghara-Baghi used his authority to prevent military action against Khomeini. He alone knows what decisions were made and the price paid. It is perhaps significant that although all my generals were executed, only General Ghara-Baghi was spared, His savior was Mehdi Bazargan. 

By the time Huyser left Iran, the army had been destroyed and the Bakhtiar government he had supposedly come to save was in shambles. At the travesty of a trial which preceded the execution of General Rabbii, the commander of the Iranian Air Force, the General told his “judges” that “General Huyser threw the Shah out of the country like a dead mouse.” 

Bakhtiar's government was approved by the Majlis on January 16 by a comfortable margin. Plans for my departure had been announced, interestingly enough, on January 11 in Washington by U.S. Secretary of State Vance. 

Those last days were days of heartbreak, of nights without sleep, the deplorable conditions in my country were naturally my preoccupation at every moment, and it was necessary to continue to work while knowing that my departure was imminent. 

I cannot nor am I willing to express fully the sentiments which I felt on January 16, 1979, when I took the road to the airport with the Empress and my children. I had in me a sinister foreboding for I knew all too well what could happen. 

I wanted to persuade myself that my departure would calm the people, decrease hatred, and disarm the assassins. I hoped that Shapour Bakhtiar would perhaps be fortunate and that the country could survive, despite the immense destruction being inflicted by the furious crowd. 

An icy wind, usual for this time of the year, swept the airport at Mehrabad, where rows of planes stood, immobilized by the strikes. At the foot of the plane, our national leaders were gathered to say farewell to us: Shapour Bakhtiar, the presidents of the two chambers of Parliament, ministers and generals. 

I advised them all to exercise prudence. As God is my witness, I had done everything within my power to protect those who had served me. 

The Imam Jom’eh who, during all my departures had been present to recite the traditional prayers, was not there. Perhaps some people misunderstood and gave his absence a significance which it did not have. The poor man was really ill and died a few weeks later in Geneva. But I had with me the copy of the Holy Koran that never leaves me. 

I was completely overwhelmed by the expressions of loyalty given to me when I left. There was a poignant silence broken by sobs. 


The last image which I carried of this land over which I had reigned for thirty-seven years and to which I had offered a little of my blood was that of the frightful distress on the tearful faces of those who had come to bid us farewell. 