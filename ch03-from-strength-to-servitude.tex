
LEGEND MAY HAVE EXAGGERATED Tamerlane’s acts of violence; it is, however, undeniable that his was a reign of terror. Persia seemed to be finally swallowed up, wiped out of history for ever. And yet the Safavid dynasty (1501-1722) witnessed a Persian renaissance. 

The first Safavid, Shah Ismail (1487-1524), was to unite the country against the Uzbeks in the east and the Ottomans in the west. He sealed the moral unity of the nation when a majority rallied to Shiism, which became the official religion of the state. 


Ismail was to retreat before the Portuguese under Alfonso d’Albuquerque who had taken possession of the island of Hormuz and of the neighboring coastland. It was the first time since the fall of the Roman Empire that the West had-attacked Persia. It was a sign of new times, modern times: The movement which drew us towards the East had come to an end; imperceptibly the attraction towards the West had begun. 

Ismail's great-grandson, Abbas the Great (1587-1629), made us powerful again, prosperous and to be dreaded. He could not regain Mesopotamia, which was lost for good.in 1534 to the Ottoman Empire, but he embellished his capital, Isfahan, attracting artists and poets to it, and made of it a magnificent city of 600,000 inhabitants, “one half of the world.” It was another golden age of architecture and painting. 

It was Abbas, too, who authorized the Dutch and English East India companies to establish trading posts in Persia. It is true that the East India Company helped him to chase the Portuguese out of Hormuz (1622): we were not to profit from the arrangement. 

From 1629 to 1736 came another century of retreats from Turkish and Russian invasions. By the agreement of 1724, these two divided our northern provinces between them, while the Afghan Ashraf occupied the whole of the east and pillaged Isfahan. Persia was dismembered and torn apart. All seemed to be lost once again. 

The sad destiny of our last Safavid king the unfortunate Soltan Hoseyn, was to be miserably surrounded in his own capital by Afghan brigands. The destiny of Persia seemed sealed forever. 


Nadir Shah, however, whom history has dubbed the “Persian Napoleon,” was to give us back brilliance and power, Having put down the rebels in the north at Meshed and Herat, beaten Ashraf and retaken Isfahan (1729), he quickly chased away the Turks and then turned round against the Russians, who preferred not to confront him and abandoned their conquest. 

Next, Nadir Shah attacked and took Kandahar and Kabul, crossed the Tsatsobi Pass--unlike the Khyber Pass, it was not defended--and came out to the rear of the enemy whom he then vanquished. Taking his prisoners with him, Nadir marched on Delhi which he entered in March 1739. Whereupon, writes an Indian historian, “the riches accumulated over 348 years changed hands in a moment.” The jewels that passed according to the fortunes of war would become Persia's crown jewels, the ornament of our nation in good times and the backing of our currency in bad times. I left them in the vaults of the National Bank. God only knows what the mullahs have done with them. 

On behalf of his second son, Nadir obtained the hand of Aurengzeb's granddaughter and, satisfied, he withdrew and offered Mohammad Shah his kingdom, minus the right bank of the Indus which had been part of the Achaemenid Empire. 

Nadir Shah has been compared to Napoleon because of the brilliant campaigns in which he was constantly victorious. But whereas Napoleon was finally beaten by the coalition, Nadir remained invincible. However, the French Emperor was an outstanding administrator, which Nadir was not. And Napoleon sought to help his family whereas Nadir's abominable cruelty caused him to put out his own son's eyes. 

Nadir Afsharid’s dynasty was succeeded by the Zand dynasty (1757-1794), which gave us Karim Khan (1757-1779) who came to be known as the "Serfs' defender.” The dynasty ended in a new civil war which resulted in the coming to power of the Qajar dynasty (1794-1925). 

It was then that Persia began to disintegrate into an anarchy which was all the more catastrophic for coinciding with the industrialization of the Western powers. Their expanding economic interests gave rise to Western colonialism, While the Western powers were invading the four corners of the earth--economically, militarily and politically--we lost our Caucasian provinces to the Russians by the treaties of Gulistan (1813) and Turkimanchi (1828). We also lost the province of Herat to Afghanistan, which Great Britain forced us to recognize by the Treaty of Paris, 1857, and Merv, in the northeast, to the Russians. Finally, in 1872, the province of Seistan was divided between Afghanistan and Persia. 

Fath Ali Shah, the Qajar ruler who reigned from 1797 to 1834, endeavored to recapture Georgia. He greatly admired Napoleon and he warmly welcomed the “Comte de Gardanne” and the military-diplomatic mission which the French emperor sent to Teheran in 1807. This mission studied and explored the Persian roads with a view to sending a powerful French expedition into India. 

In Egypt, Napoleon had carefully studied Nadir Shah's victorious campaign of 1739. Today his plan is seen as a mere dream. But the recorded correspondence of Fath Ali Shah, his son Abbas Mirza, and General Gardanne with Napoleon and Champagny, the French foreign minister, reveals the emperor's real intentions. He saw Persia both as the natural bastion of the West and its passageway to the East. He consequently felt “this strategic area of primary importance” should be put to use for offensive and defensive purposes. It was first a matter of containing the Russians: The Persians could be counted on “provided they had twenty thousand guns and a good artillery at their disposition.” 

Next it was necessary to train seriously and to make use of “the 144,000-strong Persian cavalry, troops of the first order” as the vanguard of the French expedition to India: “An expedition,” wrote Gardanne on January 26, 1808, “which everyone in Teheran is thinking about.” 

The French headquarters in Teheran, Isfahan and Shiraz, calculated that “the Indus campaign” would last from five to seven months depending on whether the Grande Armée advanced by road (through Aleppo, Baghdad, Basra, Shiraz, and Yazd) or arrived by the Black Sea and Trebizond. One part of the army would then advance through Erzurum, Hamadan, Yazd, and Herat, and the other through Tauros, Teheran, the Khorasan. 

Gardanne added: "A beast of burden will be needed for every two men. ... cannons, cannonballs, and gunpowder will be manufactured on the spot: Persia has a very fine quality of saltpeter....the Indus sikhs at war with the English can muster 50,000 cavalry.” 

Unfortunately, Fath Ali gained nothing that-he wanted from the French alliance. The Russian troops overran Northern Persia and Gardanne wrote to Napoleon: “An English mission is coming up the Persian Gulf, led by Sir Harford Jones who is a great blackguard and laden with gold.” 

At the time the emperor was in Spain chasing the British expedition with the intention of putting his brother, Joseph, on the abandoned Spanish throne. His plan for an expedition to India by way of Persia was dropped as a result of his armies having to fight on the Danube in 1809 and, three years later, on the Moskva. 
