WHEN I WAS SIX years old I was entrusted to the care of a French governess, Madame Arfa, who had married an Iranian. To her, I owe my ability to speak and read French as if it were my own language, as well as my interest in Western culture. I was also exposed to her violent hatred of all things German and listened to many tirades against the despised “boches.” 

Under her tutelage I began dreaming dreams unusual for a boy my age. But then I already knew that I would not lead the life of an ordinary boy. Thus, I dreamed of making my country’s peasants happy and having every man judged by just laws. These were dreams that never left me and which to a large degree I managed to realize in later life. 

Though my father had done much to improve Iran's educational system, much still needed to be done. As a result many Iranian children from good families in those days were educated abroad. So, after graduation from the elementary military school in Teheran in May 1931, my brother, Ali Reza, and I were sent abroad to obtain the kind of top-notch schooling Reza Shah wanted for us. In 1931, we embarked at Bandar Pahlavi, a small port on the Caspian Sea, on our way to Baku in Russia. From there we took a train across the Soviet Union and Europe to Lausanne. It was an enormous trip for anyone in those days, and even more dramatic for young boys. 

My father sent several other boys with us to make sure we would not lose contact with our people. Among them was a boy named Hossein Fardoust. I had known him at court since we were both six years old; he was to become one of my closest friends and advisors and ultimately betray me. He is now head of Savama, Khomeini’s Secret Service. Mehrpour Teymourtach, the son of my father’s court minister was also with me. Later his father fell out of favor and my friend left school. 

We spent our first year in Switzerland at the Ecole Nouvelle in Lausanne, and then moved on to LeRosey, an elite Swiss school that has educated the sons of the world’s best families. Not surprisingly, given my early childhood training with Madame Arfa, I was able to adjust easily to Swiss life. I was well read for my age and my ability to handle the French language was up to Le Rosey’s standards. By and large the next four years were happy and productive. I followed the scientific course, not Latin and Greek, and did well in most subjects. My favorites were history, geography, and science, though geometry was not a strong point with me. I was petrified every time the mathematics professor entered the classroom, as were most other boys. Yet looking back, I remember him as a very nice man. His wife was a writer and she was very sweet to me. 

My social life was varied enough, though I preferred the friendship of older boys. My popularity with the others was bolstered by a large supply of pistachio nuts which I kept in my room. Finally, for my last two years my younger brothers joined me at Le Rosey. 

Intellectually, my education confirmed my passion for history and for the great men who had important roles in shaping it. I admired the emperor Charles V, for example, for his military genius in establishing what was then the best infantry in Europe and for giving such prestige and strength to Spain. Peter the Great unified Russia and I found his accomplishments fascinating, though I was concerned about the human costs of his achievements. And Catherine the Great continued the course he had set. The French, of course, were closest to my thinking. I admired the great French rulers like Henri IV and Louis XIV, the Sun King, and most of all Napoleon, a truly extraordinary leader. And, with great reservation, I studied the history of the French cardinals who had advised and directed their kings. I never thought the Peacock Throne could stand a Richelieu or a Mazarin. I shudder to think of my reign with clerical advisors, especially in light of the results of today's so-called Islamic Republic. 


In 1936, at the age of 17, I returned home to Iran for the first time in five years. It was like visiting a different country. I recognized nothing. The sleepy port city of Pahlavi had become a modern Western town, My father had razed Teheran’s old walls. Streets were paved and asphalted. The city had begun to take on the look and style of a European capital. I saw it all at first as if in a dream. My father met me and we drove in an open car through the streets of the city. Thousands of young people lined our route, tossing flowers. One full bouquet hit me in the face and my father similarly received another. The welcome was overwhelming, and surely one of the most moving experiences of my life. I had little time, however, to savor the reception or to think deeply about the new Iran growing up around me. My father sent me to our military school where I studied under Iranian and French officers trained at St. Cyr. 

In the spring of 1938 I graduated as a second lieutenant and immediately took up my duties as an army inspector. My relationship with my father grew closer. I spent several hours a day with him and learned much about his vision of rule and about the practicalities of being a king. Soon I was the only man in the kingdom able to express my opinions to him even when they contradicted his own. We often debated land reform. My ideas had not yet crystallized but I knew that the lot of our peasants had to be improved and improved from the top. At the time I had a plan for a debt moratorium that would free farmers from payments to their landlords for several years and let them use the money instead to build houses and buy cattle. Eventually I wanted them to own their own land. I was troubled by the vast land tracts my father purchased. One day we discussed the issue and he explained that he concentrated his land-buying along our country’s frontier primarily for national security reasons. Although he had in mind a better life for the peasants, he knew it would take time and that national security had to come first. 

My father took me on his frequent inspection trips through the country. He wanted me to know the land and how to rule it. I studied his role as military leader and especially remember his vivid descriptions of the chaotic situation of our country during the terrible years from 1915 to 1921. He had an uncanny understanding of international politics and its implications for Iran. In the late thirties he sensed the advent of war. He worried that it would begin before Iran's armies were capable of defending our neutrality against all comers. His worst fears were realized. World War II broke out on September 1, 1939, long before we were ready to mount an effective defense. 


Much has been made of my father’s alleged admiration for Hitler and for the Germans. Certainly Britain and Russia used his supposed links to the Third Reich as a pretext for invading our country in 1941 and they are still cited in their history books as fact. This was not fact but myth. My father mistrusted Hitler from the very beginning, if for no other reason than as an authoritarian ruler he was deeply suspicious of another who used such brutal methods of rule. And he had learned from recent history: On his state visit to Turkey in the late twenties, he had listened intently to his hosts’ graphic description of life under the German boot during Turkey's World War I alliance with the Kaiser when Berlin dominated Turkish life and the Germans treated the Turks like dirt. He had been deeply impressed then and now was deeply suspicious of German intent. True, we employed a number of German technicians, but their employment had nothing to do with politics. They were quite simply excellent at their jobs and vital to our modernization drive-- as they would be again in the sixties and seventies when German technological expertise was utilized to implement my own White Revolution. In any event, when war came my father declared Iran neutral and hoped for the best. 

During the early years of World War II, it seemed as if we might be spared. That hope began to fade after the Axis powers invaded the Balkans in April 1941 and German soldiers stood distressingly close on the Bulgarian border. The Nazi invasion of Russia on June 22, 1941, filled us with new foreboding. My father promptly reiterated Iran's neutrality in forceful language. 

But once again history and geography conspired against us. Hitler's Blitzkrieg in the East threatened to bring Russia to her knees in a matter of months. It was clear the Soviets could not survive the German onslaughts without Allied support. Shipping supplies through Murmansk in the far north, although possible, was difficult and slow. Sending supplies through the Mediterranean that summer was out of the question, since the Axis dominated the air and British ships could barely reach Egypt. Turkey would not allow passage through the Dardanelles, even assuming supply ships eluded German bases in Greece. And once in the Black Sea, Berlin controlled the coasts of Bulgaria and Romania. The only safe and relatively rapid supply passage was through the Persian Gulf. Thus, Iran became an area of prime strategic and tactical importance. 

By July 1941, our neutral position had grown precarious. The first warnings came from my brother-in-law, Egypt's King Farouk, whose sister Fawzia was my wife. Farouk told his father-in-law, then ambassador in Teheran, to warn me that recent British troop movements might be directed at Iran. In turn, I alerted my father. Reza Shah immediately cabled his minister in London, Mr. Moghadam, and instructed him to determine Britain's real intention. Unfortunately, we received no reply. The Churchill government was not interested in seeking our permission to ship supplies across Iran. Moreover, the Axis gave Britain every pretext for intervention: There were reports that Italian planes were bombing areas around the Persian Gulf and that armed German merchant ships were plying Gulf waters. In Teheran, meanwhile, the British and Russian ministers were pressuring my father's government to expel the last German technicians. We were prepared to negotiate this issue too, and had already made concessions. Clearly the British and Russians were not interested: their major objective was the trans-Iranian railroad and on August 23, 1941, both countries invaded without warning. 


To the north, strong, motorized Soviet forces crossed the frontier at Azerbaijan; other units advanced in the east at Khorasan and along the whole frontier. Five British divisions came up from the southeast, the south and the west. The Royal Air Force bombed military targets such as Ahvaz, Bandar-Shapur, and Korramshahr, taking pains, however, to avoid petroleum plants. At dawn on August 25, a Royal Navy warship, the H.M.S. Shoreham, sank one of our frigates off Abadan; the Soviet Air Force bombarded Tabriz, Ghazvin, Bandar-Pahlavi, Rasht, and Fezajeh. 

Our ambassador in Moscow, Mr. Saed, protested to Molotov, and asked him why the Russians had agreed to participate in a military Operation against Iran instigated by the British. Molotov did not reply. But we now know that it had been decided to open the Iranian road to Russia when Churchill and Roosevelt met to sign the Atlantic Pact. . 

On August 28, Reza Shah ordered our forces to lay down their arms. He was notified that on September 17 the allied forces would enter our besieged capital. When he heard that British troops were approaching Teheran, he said to me, “Do you think that I can receive orders from some little English captain?” 

He had made up his mind to abdicate. He was too independent and proud a man to subject himself to foreign invaders. The army had stopped fighting and his generals had agreed to disband the armed forces. He knew that meant a resurgence of tribal warfare, for who would keep the country together once he and the army were gone? Above all, he knew how much the British feared and hated him. How could he have dealt with them under such conditions? 

The British still ruled the waves in those days. They had India, Iraq, and most of the Middle East. He and I had talked often of British treachery. His distrust of British intentions went back to World War I and to the 1907 partition of the country, which he now felt was being repeated. 

On September 16 he abdicated. The abdication act was read to Parliament by the Prime Minister, Furughi: 

\begin{chapquote}{Reza Shah Pahlavi}
I, Shah of Iran by the grace of God and the nation, have taken the grave decision to withdraw and to abdicate in favor of my beloved son, Mohammad Reza Pahlavi. ... 
\end{chapquote}

Parliament ratified the act unanimously. But how was I to reach Parliament in order to swear my oath and receive my investiture? It was not easy. Russian and British troops had just entered Teheran. But crowds of Iranians swelled the streets and thus ensured my triumphant entrance into Parliament. When the ceremony had ended, the Iranians, carried away by their enthusiasm, even wanted to lift up my car and bear it on their shoulders. In this hour of danger, a fantastic display of patriotism and popular loyalty to the dynasty had been shown. This moment, I shall never forget. 

The British and Russian ambassadors had remained absent from the ceremony. The British had been inclined to support a Qajar prince who was an officer in the Royal Navy. Only after three days did their governments recognize me. During that time popular demonstrations in my favor showed them that they had no alternative. 

The occupying powers hoped to find in me an obedient head of state. They imagined that it would be easy to manipulate so young a sovereign. Their aims were always the same: As in 1907, Iran had to be made into a neutral area, “maintained in a state of respectable anarchy.” 

My father hoped to go to Canada or to Latin America, but the British would not permit it. A virtual prisoner, he was taken first to India but was not allowed ashore at Bombay. Clearly, the British feared the tremendous impact his presence would have on the already restive subcontinent. Next, they decided to take him to the island of Mauritius and finally to Johannesburg, South Africa. The very last message I received from him in his exile there was on a phonograph record. “My son,” he said to me, “fear nothing.” I was never to see him again. 

When I learned of his death in Johannesburg in 1944, my grief was immense. I owed it to his memory to continue to the very end the task which he had undertaken. 

It seemed obvious to me that it was once more a question of the life or death of Iran: We were going back to 1920. But it was 1941 and I was only twenty-two. 

Once I was confirmed on the throne by popular support, the British abandoned whatever vague plans they had of putting a prince of the deposed Qajar dynasty in power. I then set about solving the massive problems that arose from the Soviet-British occupation of Iran. 

Clearly, the most urgent was some formal agreement with the occupiers that gave legal assurances for Iran's continued independent existence. In January 1942 we concluded a triple alliance with the British and the Russians that recognized Iran's sovereignty and political independence. Article V specified that the Allied forces leave Iranian territory within six months of the end of hostilities. Article VI contained crucial guarantees against any future division of the country between Russia and Britain. This legal framework was only the beginning of the struggle I waged from 1942 to 1946 to keep intact my country’s political and economic integrity. 

There was little I could do about the all-pervasive black market, a problem that has beset Iran in both war and peace. (Even in the 1970s I was still obliged to battle illegal profiteers and others who moved on the fringes of the economy.) I could merely limit black market operations but there was no hope of breaking its stranglehold on the Iranian economy. That economy, moreover, was further weakened by British and Russian requisitioners who took what they wanted and paid little attention to our needs. Caught between a virulent black market and the demands of the occupying powers, Iran's economy was in shambles. 

My leverage was limited but I used what I had. For example, one day the Russian Minister appeared with instructions to dismantle a rifle and a machine gun factory in Teheran and take the equipment to Russia. 

“Why should you do that?” I asked. “Why not place an order with us and we will manufacture the guns to your specifications?” Bargaining wasn't easy but in the end I prevailed, mixing stubborn questioning of their right to take the plants with a willingness to put the output at our “ally’s” disposal. For we were allies now, Iran had declared war on the Axis. 

However, “allied” status did not end foreign interference in our affairs. One day the British ambassador asked me to direct the Central Bank of Iran to print more money to help British and Russian troops defray their local costs. My government had balked, and if the impasse were not resolved, he said, London would lose confidence in the regime. It was, quite simply, a demand that we inflate our currency and add sharply higher prices to the crushing burdens our people already bore. I refused. It was not up to Britain to have confidence in Iran's government but to me, the Parliament, and the people. 

It was the first time I had defied an occupying power so dramatically and it helped strengthen my position both at home and with the Allies. However, the British knew how to manipulate Iranian politics and to operate beyond my then limited authority. They controlled elections to the Majles or Parliament--the British used to bring a list of eighty candidates to the Prime Minister in the morning and in the afternoon the Russians brought a list of twelve and that was that--and they got most of what they wanted, pushing our economy into even deeper trouble. 

British and Russian interference in our elections was another source of friction and led to my first efforts to bring a man into the government who later would almost destroy me--Mohammed Mossadegh. Thus, in 1943, [asked him to become Prime Minister. I hoped that a “nationalist” leader like Mossadegh would have the backbone needed to reform our electoral system and hold elections free of foreign interference. 

To my great surprise, Mossadegh replied that he would accept on three conditions: 

\begin{enumerate}
\item If the English would agree. 
\item If he could see me every morning and hear my advice. 
\item If I would give him a personal bodyguard. 
\end{enumerate}

I told him I had never asked the English for their advice and that if I were to do so now, I would have to ask the Russians also. I stuck to that position even though Mossadegh said that “nothing was ever done in Iran without the agreement of the British” and the Russians “did not count.” I sent my Court Minister, Mr. Ala, to the British and the ex-Cossack, General Yazdan-Panah, to the Russians. The Russians agreed to the proposals; the British did not. In fact, their ambassador told my emissary that “the king is gambling with his crown.” 

When told of the outcome of these discussions, Mossadegh refused my offer for the Prime Ministership and the farcical elections continued. The English always talk about the merits of democracy, but found it perfectly normal to dictate how Iranian elections should be held. 

Mossadegh’s reaction should not have surprised me, for he had played a role in Iranian politics long before my father came to power. Indeed, some say that the basis of his fortune, which was considerable, dated back to around 1910. He then, through British influence, was appointed by the Qajar dynasty as finance supervisor in one of our provinces. It is said that he used that position to steal substantial amounts of money. In addition, Mossadegh owned large tracts of land. He also practiced law, having studied in Switzerland and France. In the late 1920s my father made him governor of Azerbaijan province and later he served as ambassador to Iraq. However, we always suspected he was a British agent, a suspicion his future posturing as an anti-British nationalist did not diminish. Certainly my father had long suspected his British connections and in 1940 jailed him on espionage charges. Mossadegh’s friends then urged me to intercede for him; this I did and he was released. He said at the time that he owed me his life, but that did not prevent him from betraying me later on. 

In November of 1943, Roosevelt, Churchill, and Stalin held one of the key conferences of the war in Teheran, where many of the major decisions affecting the conduct of the war and the postwar future were taken. We were, of course, honored that Teheran had been chosen for this meeting, though we understood the logic behind the choice: Roosevelt and Churchill were already in Cairo, and Teheran was not far from Stalin's home in Georgia. 

I had met Churchill the year before when he stopped in Teheran on his way to Moscow. We lunched together in the garden and he talked to me about the importance of radio in the modern world and its usage as an instrument for governing. In fact, he encouraged me to set up a radio network of my own in Iran. We also discussed the war at some length. 

Despite my youth I offered my opinions on politics and military matters. I believed that the Allies should invade Europe from the south, that is, through her weakest points, with Italy and the Balkans the most promising targets. Sunk in his chair, with his usual grumpy look, he watched me with interest. He never took his eyes off me while I talked. When I finished he made no comment. Many years later when I read his memoirs, I noticed that the theories I had expounded that afternoon in Teheran coincided with his own. 

During the Teheran Conference Churchill lived at the British embassy, several hundred yards from the Russian compound where Stalin and Roosevelt stayed. Although I was technically the host of the conference, the Big Three paid me little notice. We were, after all, what the French called a quantite negligible in international affairs and I was a king barely 24 years old. Neither Churchill nor Roosevelt bothered with international protocol that required they call on me, their host. Instead, I paid courtesy visits to both in their embassy residences. Churchill did have some kind words to say about me in the House of Commons on his return to London, but we only talked generalities during our meeting. 

Roosevelt stood at the peak of his power that year. Imagine my surprise, when I heard this agreeable man asking me to engage him as a forestry expert in Iran once his term as U.S. President had expired. What could such a request mean? Did Roosevelt believe the future of Iran had been secured so he could already worry about future problems such as reforestation? 

While my calls on Churchill and Roosevelt were perfunctory and without real significance, my meeting with Stalin was entirely different. For one thing, he was the only participant who bothered with protocol and called on me, rather than summoning me to the embassy as the other two had done. For another, he was polite, well-mannered, and respectful, not even touching his tea before I had mine. What is more, he spoke about matters important to Iran. 

Although I would learn later that Stalin's statements were deceptive, at the time I did not know that. With Molotov sitting beside him and only interpreters present, he began our conversation by declaring: “Have no worry about the next fifty years.” Was this a guarantee comparable to the one the Tsars had given the Qajar dynasty, I wondered. As a young patriot king who had seen his country's army destroyed and deprived of weapons and material, I was burning to talk to him about our need for planes and tanks. 


When I did, he immediately offered me a tank regiment and a squadron of planes, with troop training and method of delivery to be discussed later. I thanked him warmly for his expression of support, believing that I had taken an important step toward establishing the independence and prestige of my country. It would prove to be my first lesson in Russian duplicity and the hard bargain Moscow drove for every concession, Weeks later the Russian ambassador brought Stalin's terms. They were draconian. Russian officers were to command both the tank regiment and the squadron of planes, at least for the duration of the war. The tank regiment was to be based at Ghazvin, west of Teheran, while the planes would be stationed at Meshed in the northeast of the country. Indignantly, I refused to accept the Soviet conditions. Moscow wasted little time in letting me feel its displeasure. Soviet radio broadcasts began to attack me, while the newspapers of the communist Tudeh party in Iran began to snipe at my handling of foreign affairs. 

Earlier than most statesmen, I learned that Moscow plays rough and plays to win. Stalin laid down the rules for the game. However, much as I opposed him and everything he stood for, the man was a colossus. The great victor of World War II was neither Churchill nor Roosevelt, for all their eloquence, but Stalin. He pulled the strings at Teheran, Yalta, and Potsdam, and he imposed a Soviet peace on the world that has now lasted for thirty-five years. 

Part of his secret was the willingness to bide his time, while never really pulling back. Moscow saw the moment of opportunity for political action in Iran the day my father was forced to abdicate and leave the country. Moscow helped found the Iranian branch of the communist party, the Tudeh. I say “helped found” because the British had a hand here, too, however difficult that fact may be for naïve people to believe. An employee of the Anglo-Iranian Oil Company and a known British agent, Mustafa Fateh, financed the Tudeh newspaper, Mardom (The People). Media ownership, of course, is crucial to any fledgling party’s political success and Tudeh was no exception. 

Why would the British help launch a communist party abroad? The answer is not that complex. First of all, the British meddle in everything. Second, it was and is their policy to have their people everywhere, hoping to exercise some control no matter what happens. This is a policy Britain has never abandoned, not even today. For example, look at Aden and the rise of communism there. The leaders of the trade unions were all educated in England, at the London School of Economics, I believe. The British thought that having several people who pretended to be anti-British would give them control of the nationalist movement. 

I told the British over and over that it would take only one bullet to put those people away. Once they are gone, what will you do? Who will supervise the trade unions for you? And that is exactly what happened in Aden. Once British agents were eliminated, the whole thing fell into communist hands. 

To return to the Tudeh party, the British thought that a hold on the Tudeh would give them similar leverage. Specifically, they hoped to infiltrate their agents among the workers in Abadan and in the refineries and oil fields of the south. Moreover, while Fateh was engaged in publishing Mardom, he served as an advisor to the British military attaché, General Fraser, and supported his efforts to remove me as commander-in-chief of the army, my greatest source of power. Even in those days the British went to any length to “clip my wings!” 

Tudeh did not have much impact at first. It was limited to Teheran teachers and industrial workers, whose economic lot was not very good in those days. Only a few intellectuals supported it. The party had little success in the villages and the countryside; that came only after the war, when Russian interference in our internal affairs mounted dangerously. 

At the Potsdam conference in July 1945, Stalin had assured President Truman that Russia would take no action against Iran. Soviet troops were to be out of my country six months after the war with Japan ended; this pledge was reiterated at the London Foreign Ministers’ Conference in September of that year. But Stalin had no intention of keeping his promises. Even as he made them, Soviet activities in Iran were being stepped up. Red Army troops stopped Iranian police from entering areas where the Tudeh was strong. Rebellion and unrest were fomented in Azerbaijan and other northern provinces. Allied efforts to make the Russians cease and desist proved futile. 

Indeed, Stalin and Molotov were so sure that Iran would fall in their Soviet laps that they rejected a British suggestion to make Azerbaijan, Kurdestan, and Khuzestan autonomous provinces. This proposal was made at the Moscow foreign ministers’ conference in December 1945 by Secretary of State James Byrnes and British Foreign Secretary, Ernest Bevin. At first Stalin accepted the proposition, but after Molotov told him to "wait one year and we'll have the whole country,” Stalin believed him and refused the Western offer. 

For many months it looked as if the Soviet gamble would pay off. Rebellions in Azerbaijan and Kurdistan grew and spread with Russian support. Soviet troops stopped our columns from moving into rebellious areas. In Tabriz, the Soviets pressured the commander of our garrison to surrender to rebel forces. The officer was later tried for treason and condemned to death. I commuted his sentence, however, because I was not sure that he had the means or the will to do anything else, especially since our troops could not get through to end the siege. Still, the fall of Tabriz allowed the rebels to proclaim the autonomy of both Azerbaijan and Kurdistan. 

March 2, 1946, came and went without any sign of Russian withdrawal, although they had agreed in London the previous September to leave Iran by that date. Truman sent a polite note to the Russians on March 6. It was ignored. A stiffer letter followed and on March 24 Moscow announced that Soviet troops would pull out. By May the Red Army had officially departed. But the rebellion was far from over and we still had to defend our independence despite the many treaties that guaranteed it. Attempts at secession were made in the west and trouble broke out in the south, notably at Abadan where the Tudeh was still powerful. As if by chance, other tribes rebelled in the Fars, the region of Shiraz and around Isfahan. Clearly, the British and Americans were ready to occupy the south and divide Iran again, should we fail to reclaim our northern provinces. The unrest in the southern regions of the country, however, proved relatively easy to put down so that we could soon turn our full attention to the critical north. 

Not surprisingly our relations with Russia during this period were tense. We did not know from one day to the next whether the Soviets would withdraw or turn and attack Teheran, and after they left, if and when they would return to support further rebellions. With our Soviet policy stalemated, my Prime Minister Ebraham Hakimi, an old man of integrity and strongly pro-British sentiments, but always a patriot, resigned. His successor, Ahmed Ghavam, had previously served as Prime Minister. Our relationship had never been an easy one, and it certainly wasn't in 1946. Immediately upon his appointment he left for Moscow and returned with an agreement providing for joint prospecting and extraction of oil, with a 51-49 percent split in Russia's favor. 

Fortunately the agreement required parliamentary approval and could thus be held in abeyance. Ghavam's second move, however, could not be ignored as easily. Soon after his return he began talks with the Azerbaijan rebels. One of his proposals called for the promotion of every rebel officer by two grades--making a lieutenant a major, for example-- as a price for agreement. 

"I would prefer,” I said, “to have my hand cut off than to sign sucha 
decree.” 

The rebels in Azerbaijan at the time were led by a man named Pishevari whom I knew on sight. He had once been elected to our legislature, but Parliament had refused to seat him because he was known to be a communist and a puppet of the Russians. The wisdom of Parliament s action in refusing him a seat was borne out in the rebellion. It now became a matter of taking military action while there was still time. The Russians were gone but not forgotten. Some of Pishevari's rebels had been sent to the Soviet Union for military training. They would return soon, I knew, with equipment, expertise, and Soviet advisors. If I waited any longer, my armies would lose. 

Both the government and most of my military leaders strongly advised me against attacking, lest an attack invite Russian intervention. Even the American ambassador, George Allen, who was my friend, warned that ‘the United States are one hundred percent in agreement with you, but we won't go to war with Russia for your sake.” Persuading the opposition to rally round my plans took some months. Ghavam’s efforts to settle the rebellion included taking three Tudeh members into the government. Two were given minor posts but one was named Minister of Education. That is and remains a crucial post with enormous influence and I had given my approval with great reluctance. American support for Ghavam complicated my political position that spring. It added to the limits placed on my freedom to maneuver in treacherous political waters and made dismissal of the communist ministers more difficult. Fortunately, they did not remain long in the government. Nevertheless, with Ali Razmara, the chief of staff, behind me, I gathered the needed support, including, in the end, that of the Prime Minister. Moreover, I knew the military operation would not be that risky. The rebels were no better armed than we were and did not enjoy much popular support. Volunteers rushed to our side, more of them in fact than we needed. I felt I had the whole people behind me and preferred to risk an honorable death in battle than to become a monarch of servitude and shame. 

The decision made, Razmara and I made repeated reconnaissance flights over enemy-held territory, often in old planes, sometimes in a small, twin-engined Beechcraft, and always without radio. As our troops took to the offensive, Pishevari's forces were divided into three columns and harassed increasingly by a loyal local population. Finally, they disbanded and Pishaveri and his acolytes fled across the border into Russia. 

This all happened very suddenly. I was in Teheran awaiting the visit of the Russian ambassador. Just before he arrived, I was told that the rebel forces had disbanded and fled. The Soviet envoy then entered, flushed and furious. He demanded an immediate halt to our advance which he termed a “threat to world peace.” 

I refused, pointing out that our army threatened nothing and no one. We were merely reestablishing the status quo and preparing general elections in territory that belonged to us. Then I simply added, “Besides, you must know that the rebels have just surrendered.” 

Thus ended the second attempt in the 20th century to erase Iran from the map of the world. The first had been made in the Anglo-Russian Convention of August 30, 1907, that divided the country between Russia and Britain with Russia taking the north and the British the south. It had remained divided until Reza Shah reunited Iran after World War I. 

The fact that the same plan was revived to meet Allied needs during World War II shows the continuity of certain western policies toward Iran. The Anglo-American offer to Stalin in Moscow in December 1945 was another step in that direction. Stalin's greed together with his failure to take into account my reaction, the fighting spirit of my ill-equipped soldiers, and the loyalty of our people to crown and country, prevented that plan from succeeding. 

Once outside efforts to disrupt our country had failed, politics was used as a means to destroy the state. Corruption and subversion then combined against the nation and the Iranian people whose unity it was my duty to preserve. 

Throughout the history of my country outside forces have continually tried to use such tactics. Their sole aim has always been the disintegration of Iran. Time and again such attempts have been made--in 1907 and 1945-46 they did not succeed. Sadly, today they seem to be succeeding. 