WHEN BRITAIN AND RUSSIA signed the 1907 treaty dividing Iran between them, my father was nearly thirty. He was a giant of a man, loved by the soldiers who served under him in a brigade of Iranian Cossacks, and feared by the brigand leaders who terrorized the country-side and were in the pay of large landholders. My father was the stuff of legends and there were legends of him aplenty. 

By World War I he had a national reputation and the nickname of Reza Maxim, a tribute to his prowess with the early machine gun at a time when there couldn't have been more than five or six such guns in the country. One contemporary photograph shows him standing behind his Maxim gun, just after enemy bullets had smashed the weapon. But neither Maxim gun nor reputation enabled my father to do more than watch with anguish as Persia became in 1915 a battleground between Germans and Turks on the one hand and Britons and Russians on the other. Our people became bystanders to their own history. 

The situation grew worse after the Anglo-Iranian treaty of 1919 which in effect made Iran a British protectorate. Meanwhile, the Bolshevik revolution rumbled on in the northern provinces and proclamation of a Soviet republic seemed imminent. 

During that time, my father and his Cossacks lived in the fields trying to preserve what fragmented order remained. There wasn't much. It was a time to despair of the nation. The central government was paralyzed while bandit leaders carved up the country. Then, as now, Iran had no law, no order, no army, no police. Ignorant and self-interested clergy dispensed what justice there was, “sharing” that task with bandits whose courts meted out swift and terrible punishment. Foreigners, however, even if recognized criminals, were immune from prosecution thanks to the post-war treaties imposed on Persia. No one went out at night, not even in Teheran, except for dire medical emergencies and then as often as not a doctor could not be found. People were robbed or murdered on street corners. 

Communications had all but collapsed. Travelers from Teheran, fearful of highway bandits, went through Russia in order to reach Meshed in the northeast. Those who had business in the province of Khuzestan to the southwest went by way of Turkey and Mesopotamia. 

At one point, the visible decay of our nation had so filled my father with disgust and despair that he deliberately exposed himself to enemy fire. While up north with his soldiers fighting yet another gang of bandits, he swung onto his white horse and galloped toward enemy lines. Frightened at first by the sudden apparition of this giant on his white steed, the bandits were too stunned to react. Then, when they opened fire, they shot at random without taking careful aim. As a result, my father was able to spur his horse through the barrage unharmed. 

Perhaps it was an omen. For, shortly after my father’s return to Teheran from a victorious campaign in the north, my twin sister Ashraf and I were born on October 26, 1919. Death having rejected him and glad for a son and heir, Reza Khan decided to live and to fight on. First he engineered the dismissal of the Russian officers, who were only nominally all "White Russian.” Next, he took the Iranian Cossacks in hand and in August 1920 found himself in command of a force of 2,500. By that time the situation in Iran had changed from bad to worse. Reza Khan realized that it was a question of survival for his country and decided to act. 

With the support of liberal and democratic groups, early in 1921 he left his headquarters at Ghazvin with 1,200 horsemen and marched on Teheran. His Cossacks surrounded the capital and on February 21, 1921, Reza Khan forced the ruler, Ahmad Shah, to appoint a new government. The coup was carried out with such dispatch and so few casualties that the commander of British forces in Persia, General Ironside, is said to have told friends that “Reza Khan is the only man capable of saving Iran.” 

One of my father’s allies was a young political journalist named Seyed Zia ed-Din Taba Taba'i, a revolutionary who was well equipped to bring political pressure to bear in the capital while my father applied military force. Zia ed-Din became Prime Minister, while my father became Minister of War. However, Zia ed-Din, instinctively the radical reformer, was unable to execute any of his plans. This together with his increasingly pro-British leanings, forced him, after only three months, to resign his post and leave the country. Seyed, incidentally, stayed away until my father was exiled in 1941. Then he returned to form an opposition political party, Erade-Ye Melli (National Will), a fact that did not prevent us from becoming friends later on. 

With Zia ed-Din gone, my father persuaded Ahmad Shah to form a new government in which my father retained the War Ministry. The government was no sooner in office than Ahmad Shah left for an extended stay in Europe. In 1923 Reza became a Prime Minister in order to more effectively unite our bitterly divided country. He had no intention then of dethroning the king. On the contrary, he repeatedly asked him to return and when he finally agreed, my father met him at Bandar-Bushehr on the Persian Gulf. But Ahmad Shah was weak and uninterested and eager to return to Europe. When he finally left again for France--where he eventually died--my father realized that halfway measures no longer sufficed. The absence of a strong, effective head of state had proved intolerable. 

Reza Khan had been greatly influenced by the reforms Kemal Ataturk had introduced in Turkey and for a time he hoped to emulate his neighbor's achievements and turn Iran into a modern republic. But those efforts ran into great opposition from the clergy, politicians, and merchants because Persia, unlike Turkey, was an empire in which only the crown united diverse ethnic groups with their different languages and cultures: Kurds, Arabs, Azerbaijanis, Baluchistanis and others. Accordingly, on October 31, 1925, our parliament voted the Qajar dynasty out of power. Election of a constituent assembly followed and with only four nay votes, it handed the 2,500-year old Persian crown to generallissimo Reza Khan. 

The crown allowed him to push with greater vigor the reforms he had begun soon after the 1921 coup. Just how successfully he did so, and how much admiration his efforts garnered from the man he had sought to emulate, Ataturk, came some years after his coronation when he went on a state visit to Turkey and the honor guard's standard bearer knelt before him. A first step after the coup had been conclusion of a friendship and non-aggression pact with the Soviet Union, which voided the privileges and conditions of earlier agreements. At the same time the Anglo-Iranian treaty of 1919, never approved by parliament, was denounced. Both moves reduced foreign power and influence over Iranian government action. 

Equally important was the re-establishment of domestic unity and the removal of foreign interference in Iran's internal affairs. The British, to cite one example, had sold shares in the Anglo-Persian oil company to some tribal chieftains, who in exchange agreed to maintain law and order in the oil-producing regions. The opportunity for British manipulation was obvious. My father solved that problem simply enough: he bought up all the outstanding shares and subjugated one tribe after the other in the center, south and southwest of the country. 

But creation of a strong army ranked first on his list of priorities. After his march on Teheran he complained, “if only I had a thousand guns of the same caliber” and he meant rifles, not field pieces. Without adequate armed strength, he knew enduring national unity could not be achieved. In the twenties he expanded our armed forces to one division, an autonomous brigade and special highway patrols. In addition, he built small forts along strategic highway crossroads. Safe passage within a country is a basic prerequisite of national unity. Next, he built a navy and an air force. Officers were imported from France and Iranian cadets sent to St. Cyr, Saumur and Saint-Maixent for officer training. Later on, I would be instructed by St. Cyr-trained officers. 

A rapid program of industrialization was launched to begin domestic production of basic manufactured goods. Agricultural reform was on his mind but something he never had time to implement. That was left to me and the White Revolution I began in the 1960s. 

The stranglehold of foreign monopolies on the economic and social life of the country was gradually broken, though it was no easy task. Belgium ran the customs services with the entire revenue used to pay off our external debts. The Swedes ran the police force; our banks were in the hands of Russians, Britons and Turks. The British printed and issued our currency and owned the telegraph system. My father moved boldly to give Persia its own money. He issued new banknotes backed by gold and the crown jewels, whose most brilliant stones date back to Nadir Shah's triumph in India. 

The British Crown owns the Koh-i-Noor diamond, the Mountain of Light, but we have the Daria-i-Noor or Sea of Light, perhaps an even more beautiful stone. Along with the other crown jewels this magnificent diamond is kept in the vaults of the Central Bank. Other coffers in the vaults are filled with pearls and diamonds. Our dynasty never stopped enriching this treasure with gifts and purchases of precious stones. And we have always considered this the property of the nation and the people, a point worth making today when Reza Shah's achievements are being deliberately eradicated from Iranian consciousness. 

What did the 15-and 16-year-old rioters know of my father when they knocked down his statues in our towns and cities after my ouster? The fact that he had pulled our country up from nothing--that he had built towns, schools, the first university, hospitals, factories, roads, ports and the first power station--seems to have been lost on these so-called revolutionaries. And, he began doing all these things before issuing his own currency, a sign of his grit and determination. Both qualities came to the fore again, when in 1927 he began construction of the Trans-Iranian railroad to link the Caspian Sea with the Persian Gulf. The thousand mile line was completed in 1938. He defied the international oil powers for the first time in 1932 by cancelling the Anglo-Iranian oil concessions, originally granted to d'Arcy in 1901, and then concluding a new deal with the British. When he took power in 1921, Iranian oil production was only 2.3 million tons. By 1938 output had risen to 10.3 million tons (later dwarfed by the 300 million tons Iran produced in 1977.) 

My father's coronation took place on April 25, 1926, and as the new emperor, he became Reza Shah Pahlavi. The name Pahlavi has deep roots in our country’s history: it is the name of the official language and writings of the emperors during the Sassanid era. It is the patronym which he left me and which I bequeath to my children. During the ceremony I was proclaimed Prince and Heir Apparent. I was only seven years old. 

My father loved us dearly and deeply. There were eleven of us, all told, some my half-brothers and sisters. Our love for him was full of admiration though we held him in respectful awe. Broad-shouldered and tall, he had prominent and rugged features, but it was his piercing eyes that arrested anybody who met him. Those eyes could make a strong man shrivel up inside. Eventually I was able to say to him what needed saying without fear of contradiction or censure. But that took a long time. He was a powerful and formidable man and the good heart that beat beneath his rough cavalryman's exterior was not easily reached. Yet even his enemies realized that he was one of those men sent by Providence through the centuries to keep a nation from slipping into oblivion. 

He was impetuous and determined, qualities he needed to overcome the terrible difficulties that marked his reign. He grew up illiterate and had to teach himself how to read. As a grown man he felt no shame in starting from the beginning. Each day at the end of his army duties, he would sit patiently at his studies in his barracks, learning to read and write with the help of one of his friends. He prepared his lessons by the dim lamps, and when he was tired, he would come out of his tiny room and stand gazing at the twinkling lights of Teheran in the distance. He literally pulled himself up by his own bootstraps, and his country with him. As a ruler, he was out with his people all the time. He inspected everything, not just the armed forces. Sometimes I thought he supervised every stone put atop another. He did not surround himself with the trappings of oriental monarchs. Instead, he saw his imperial duties as a kind of military service. 

He slept on a simple mattress on the floor and was up at five each morning. By 7:30 he had finished reading through daily reports and began receiving people. As a young man I would go to his study at around eleven each morning and talk to him for half an hour or so about the problems of the country or anything else that interested him or me. Then around 11:30 we would sit down to a family lunch. After lunch my father rested and at about 2 P.M. started his afternoon work. It consisted chiefly of inspecting the army and new civil projects and institutions, The Council of Ministers, as our Cabinet is called, often held afternoon meetings in his presence. Then from 6 to 8 P.M., after all his other work, he would study the reports he had received during the day. Exactly at 8 P.M. he took his supper, and he retired at 10 P.M. However, he used to say that even in bed his plans were revolving and forming in his mind; thus he was never idle. 

Apart from relaxation with his family, my father almost never devoted any time to recreation. Perhaps once or twice a year he would go hunting for two or three hours. Mainly he got his exercise through walking. In my memory it seems as if he were a/ways walking, either pacing up and down in his office or inspecting troops or projects on foot or, in the late afternoon, taking long walks in his garden. Often he would hold audiences while walking; those whom he received were on such occasions expected to pace up and down with him. And whenever Reza Shah was walking, he was also thinking. 

His labors were not limited to industrial and military modernization of our country for he was equally interested in social reforms. Thus in 1926-27 he introduced a judiciary system modelled on that of France. He mandated compulsory, lay primary education even though we lacked competent teachers. And in the process he suppressed the often inquisitional legal powers of the clergy. 

It is essential to understand the fundamental importance of this evolution which, moreover, occurred throughout the Islamic Near East. The institution by my father, and the development by myself, of a modern political regime, partly inspired by the West, have deprived the clergy of a large part of their erstwhile privileges. 

Some of the Shiite priests at first fell back on the ancient political position concerning the very nature of power: all temporal power whatsoever must be usurped. Instead they should have taken advantage of the situation to develop their spiritual life, thereby increasing and spreading their moral and civilizing influence. 

But it must be acknowledged that, had my father not curtailed political efforts of certain clerics, the task which he had undertaken would have been far more difficult. It would have been a long time before Iran became a modern state. Because my father had little respect for certain particularly fanatical and sectarian hierarchies, he was said, quite wrongly, to be irreligious. He was a deeply sincere believer, as I am myself. His faith was that of a courageous and honest man. 

The spiritual authority of the clergy remained incontestable and uncontested. The moral primacy of the spiritual over the temporal being indisputable and undisputed, it was a matter of bringing Iran into the twentieth century, whereas today’s efforts are towards turning the clock back. Reza Shah asserted that in the twentieth century it was impossible for a nation to survive in obscurantism. True spirituality should exist over and above politics and economics. Reza Shah was too sincere a believer to see God as a sort of superior electoral agent or chief engineer of the oil wells. 

He named all his sons after the Imam Reza--with a first name by which they were distinguished--because he had a particular veneration for this descendant of our sainted Ali. Reza Shah frequently went on a pilgrimage to the shrine of Imam Reza at Meshed. Imam Reza’s Foundation was abandoned and in debt when Reza Shah came to power. Reza Shah restored it which was no easy task. Created as a religious institution, it had in better times been the recipient of land and monies bequeathed the foundation for religious purposes or charity. Usually the ruling king would be custodian or director of that foundation, but a succession of weak rulers had left Imam Reza in as desolate a state as the rest of the country. My father put things to right and set the foundation on a new path to growth. (It is worth noting, perhaps, that the Imam Reza Foundation has nothing to do with the Pahlavi Foundation, which my family and I founded much later.) 

During my reign, the Imam Reza Foundation became one of the most important and most prosperous in all Islam. Gifts from the faithful, among them myself, have turned this foundation into an extraordinary religious complex, owning factories, mechanized agricultural cooperatives, hospitals and numerous charitable organizations. Let me say, in passing, that I also restored and enriched many monuments and mosques both in Iran and abroad. It is well known that donations to foundations are non-recuperable. Now the self-styled new government in Teheran has confiscated them. 

My father also took care to protect our religion against the propaganda of an intolerant materialism which demanded that the “mosques be razed.” But this did not mean that he accepted all the claims of men of religion who lived obstinately behind the times. 

He decided, then, that the citizens should finally abandon oriental dress--their wide trousers, turbans and bonnets. Some people did not agree with this. And when women were asked to give up their black veil, measures “taken in the simple name of good sense” were actively contested by part of the clergy. I was to be far less intransigent. During my reign, women and girls were perfectly free to wear, or not to wear, the chaddor. 

My father's reforms had reduced the clergy’s authority in secular matters. Thus, from 1926 a certain section of the ecclesiastical hierarchy was openly opposed to the Shah's reforms and to Iran’s metamorphosis into a modern nation. This opposition made itself felt again at the time of the 1952-53 uprising, in 1963 and in 1978-79. 


It was thanks to my father’s example that, at an early age, I was able to understand the power of prayer, which was never a mere recitation of formulas learned by heart. 

Numerous chroniclers have published more or less accurate accounts of my childhood. Shortly after my father’s coronation, I fell ill with typhoid fever, and for weeks I hovered between life and death. The worst was feared, until one night, in a dream, I saw Ali, who in our faith was the chief lieutenant of Mohammed (much as, according to Christian doctrine, St. Peter was a leading disciple of Jesus Christ). 

In my dream, Ali had with him his famous two-pronged sword, which is often seen in paintings of him. He was sitting on his heels on the floor, and in his hands he held a bowl containing a liquid. He told me to drink, which I did. The next day, the crisis of my fever was over, and I was on the road to recovery. 

A little later, during the summer, on my way to Emamzadeh-Daoud, a place of pilgrimage in the mountains, I fell from my horse on to rocks and passed out. I was taken for dead but I had not so much as a scratch. In falling I had a vision of one of our saints, Abbas, who cradled me as I fell. 

This dream and this vision were followed some time later by an apparition near the Shimran Royal Palace. It was Imam, the descendant of the Prophet who, according to our faith, must reappear on earth to save the world. Dream, vision, apparition: some of my Western readers may dismiss it as an illusion, or put forward some psychological explanation. But remember that a faith in non-material things has always been characteristic of peoples of the East. I have found that this is also true of many Westerners. 

Long after I had emerged from childhood fantasies (if any among my readers feel more comfortable to call them that), there occurred four other incidents which may help explain why my childhood faith has continued strong within me. 

The first was an airplane accident in 1948. I was flying a Tiger Moth near Isfahan, where work was being carried out on an irrigation dam. The general commanding the Isfahan division, a cavalry officer, was with me. Suddenly, in mid-flight, my engine went dead. We had to make a forced landing in a mountainous region in a ravine full of rocks and boulders. 

As every pilot knows, a plane has a stalling speed below which it will go into a spin. With the engine gone I had no throttle, nor could I maneuver within the narrow confines of the ravine; the only thing was to maintain my speed by going down then and there. Just before we struck, I pulled on the stick to raise the plane's nose and avert a head-on collision with a barrier of rock lying directly in front of us. The plane had barely enough speed left to clear the barrier, and could not surmount a big stone lying just beyond. When we collided with it the undercarriage was completely torn off, but at least that helped to reduce our speed. The plane started to slide on its belly over the rock-strewn ground. A moment later the propeller hit a large boulder, and the plant turned a slow and deliberate somersault, coming to a halt with the fuselage upside down. There we were, hanging by our seat belts in the open cockpit. Neither of us had suffered so much as a scratch. I remember that the scene amused me so much that I burst out laughing, but my upside-down companion didn’t think it was funny. 

Another airplane had been following us. It landed behind the village. Meanwhile some people in our party who were to meet us had reached us by car. They were somewhat concerned. I hastened to reassure them, and said: “Well, now I am going on in the other airplane!” I was surrounded by generals who protested vigorously. Seeing that I had no intention of giving in, they lay down in front of the aircraft: "No, sir you cannot leave!” 

So I finished my journey by car and I had the satisfaction of arriving in Isfahan in time for what I had to do. 

On another occasion, a similar accident befell me. I was at the controls when we entered a very narrow defile. I immediately realized that it would be quite impossible for us to cross the mountain pass. I was obliged to make a half turn just as the dials showed that we were losing speed and managed, with the wings vertical and the ground only a few meters away, to right my airplane, thus, against all expectations, defying the laws of gravity and of aerodynamics. In this desperate maneuver a certain death awaited us. Logically we should have crashed. 

The young pilot who was with me was so surprised to find us both still alive on landing that he wanted, there and then, to give me a demonstration of his own talents. I realized that he was eager not to be outclassed. He wanted to loop the loop, to skim upside down over the land and to right the aircraft before completing his aerobatics. 

Since I knew that he was perfectly capable of performing such a difficult feat of flying, I accepted. Having flown nose down, he unfortunately did not succeed in righting the plane, and crashed before my eyes. Such great cruelty of fate once again forced me to conclude that my hour had not yet come. 

The attempt on my life on February 4, 1949, convinced me once more that I was protected. That day, in the early afternoon, I was to attend the annual ceremony to commemorate the founding of Teheran University. I was dressed in uniform and was going to preside over the presentation of diplomas. 

I took my place at the head of the official procession just after 3 P.M. Photographers had lined up to take pictures when a man broke away from the group and rushed at me. Not ten feet away from me he pulled a gun and fired at point-blank range. Three bullets whizzed through my hat and knocked it off but did not burn a hair. A fourth bullet went through my cheek and came out under my nose without doing much damage. He fired a fifth time and I knew instinctively that the shot was aimed at my heart. In a fraction of a second, I moved slightly and the bullet hit my shoulder. One bullet remained. Then the assassin's gun jammed. Despite the blood dripping from my face, my enemies would say afterwards that he had used cotton bullets. 

Unfortunately my assailant, a certain Fakhr Arai, was killed immediately. Perhaps it was in someone's interest that he not be questioned. What little we discovered about him was strange enough to motivate efforts to silence him. Arai was involved with an ultraconservative religious group that was comprised of the most backward religious fanatics. We also found Communist literature and brochures in his home relating to the Tudeh, the Iranian Communist party. Significantly or not, the Tudeh happened to be holding its national congress at the time of the attempted assassination. And there was a third connection: Arai's mistress was the daughter of the British embassy's gardener. 

The British had their fingers in strange pies. They were always interested in forging links with diverse groups in nations they wished to control, and they had long exercised a good deal of control over Iran. There is little doubt that London was involved with the Tudeh in various ways and of course the British had ties to the most reactionary clergy in the country. All this happened thirty years ago, yet I cannot help but wonder if in the person of Fakhr Arai I had seen the first glimmer of what would later come to be known as “Islamic Marxism.” Of course the two concepts are irreconcilable--unless those who profess Islam do not understand their own religion or pervert it for their own political ends. Arai does not seem to have been a clever man and might have confused the two opposing dogmas, aided in that confusion perhaps by subtle British propaganda to which he may have been exposed in the embassy. 

As I say, all this is conjecture, but persuasive conjecture for me. The roots of my downfall grew deep and in many places. By 1949 I had announced plans to revise the constitution so the king would have the power to dissolve parliament. That would have destroyed the parliamentary oligarchy then ruling Iran and sharply increased royal power, the last thing Britain wanted. Her policy needed a malleable king. 

The fourth incident occurred many years later when a soldier armed with a machine gun broke into the Marble Palace on April 10, 1965. I was working in my study on the first floor at the time the shooting started. I thought that maybe a thousand rifles were attacking the house, because of the echo of that machine gun in the hall. Since I had no arms in my study in those days, I just stood there, not knowing what to do. One bullet came right through the door of my study barely missing me. Suddenly, the shooting stopped. When I opened the door, I saw two of my civilian guards and the would-be assassin dead on the floor. A gardener and a valet had also been wounded in the fracas. Upon investigation of this assassination attempt, it was learned that two intellectuals and this soldier were behind the plot. The intellectuals were pardoned. As a result, one of them became a loyal supporter of mine, who later was gainfully employed in the television division of my government. It is sad to say that in 1979 he was executed by Mr. Khomeini. 

The miraculous failure of these assassination attempts once again proved to me that my life was protected. I have always had the feeling that only “that which is written” can come to pass. 

My faith has always dictated my behavior as a man and as a head of state, and I believe that I have never ceased to be the defender of our faith. An atheist civilization is not truly civilized and I have always taken care that the White Revolution to which I have dedicated so many years of my reign should, on all points, conform to the principles of Islam. 

I believe that the essence of Islam is justice, and that I followed the holy Koran when I decreed and organized a national, communal solidarity, when our White Revolution abolished privilege and redistributed wealth and income more equitably. 

For me, religious beliefs are the heart and soul of the spiritual life of all communities. Without this, all societies, however materially advanced, go astray. True faith is the best guarantee of moral health and spiritual strength. It represents for all men a superior protection against life's vicissitudes; and for every nation it constitutes the most powerful spiritual guardian. 

Our people had an opportunity to live under the banner of the most progressive religious principles possible: I am referring to the sacred principles of Islam, which at each stage of individual or societal progress marks the forward path. All those who participated in our Revolution and believed in it could justly take pride in the fact that it was inspired by the basic spirit of Islam. 

My desire to make the roots of this spirit penetrate ever more into the soul of our people was not accompanied by any animosity towards other religions. On the contrary, history will one day show that one of the characteristics of my reign was tolerance. Iran since the time of Cyrus has always been a land of refuge--except during periods of trouble when the central authority had collapsed before factions. Such is the case today. 

We had the greatest respect for all those who, living in our country, professed other faiths, for they were a part of Iranian society and also because all faith imposes respect upon the beholder. 