ON SEPTEMBER 16, 1979, in Mexico, I finished the first draft of my Answer to History, never realizing that there would be so much more to add. Completing that earlier work had seemed like a race against time. Over the previous few months my health had grown steadily worse, with high fevers, chills, and much pain. Doctors who examined me in Cuernavaca at first guessed at hepatitis and malaria, and finally that I might be suffering a recurrence of the lymphatic cancer I had fought in Iran for six years and that had seemed dormant. Special isotopes would have had to be flown from Houston every few days for my treatment if I had remained in Mexico. Unable to reach a conclusive diagnosis, the Mexican, French, and American doctors I had consulted urged me to travel to the United States to undergo a thorough examination that only one of the great hospitals of Houston or New York could provide. 

I was not eager to go. Since my departure from Teheran in January 1979, Washington had not been enthusiastic about my coming to the United States. But the Americans had always made one thing clear: I would always have access to the United States for medical treatment and assistance should my safety be threatened. These assurances had last been repeated by the American ambassador to the Bahamas when that government, under what I believe to have been British pressure, refused to extend my tourist visa. By then I no longer wanted to live in the United States. I am not a man to go where I am not wanted. But I was entitled to educate my children in America and to have access to the superior achievements of American medicine. In power, I believed that my alliance with the West was based on strength, loyalty and mutual trust. Perhaps that trust had been misguided. 

In October the decision to go to New York for treatment was made. I was too sick to hesitate any longer. My staff made the necessary arrangements and on October 22 I found myself walking across the tarmac of Mexico City airport to the waiting Gulf Stream jet. The United States Consul General in Mexico City waited near the plane to prepare the necessary entry documents. I noticed his surprised expression when he saw me. This is not how he had imagined the Shahanshah, that violator of human rights and oppressor of peoples depicted by the media for so long. Clearly, I was a very sick man. Once the formalities ended, my small entourage boarded the plane. We were to stop at Fort Lauderdale for entry formalities and then fly directly to LaGuardia airport and drive from there to New York Hospital. My staff and I hoped that by taking care of the entry procedure in Florida, we would avoid the inevitable media circus in New York and the related security problems. 

Nine months had passed since I left Iran, months of pain, shock, despair, and reflection. My heart bled at what I saw happening to my country. Every day reports had come of murder, bloodshed, and summary executions, the death of friends and of other innocent people. All these horrors were part of Khomeini’s systematic destruction of the social fabric I had woven for my nation during a 37-year reign. And nota word of protest from American human rights advocates who had been so vocal in denouncing my “tyrannical” regime! It was a sad commentary, I reflected, that the United States, and indeed most Western countries, had adopted a double standard for international morality: anything Marxist, no matter how bloody and base, is acceptable; the policies of a socialist, centrist, or right-wing government are not. 

The Western inability to see and understand clearly the grand design of Soviet expansionism had never astonished me more than in the first months of my exile. I had lived as neighbor to the masters of the Kremlin my whole adult life. In forty years I had never seen any wavering of Russia's political objectives: a relentless striving toward world domination. Moscow had time. It could wait fifty years, accept a step or two backward, deal, accommodate, but never lose sight of its final aims. I favored détente and accommodation, but not from the position of weakness and indecision that marked the policies of the American and European governments. If these policies are not reversed, Europe could be “Finlandized” in three years. Détente only makes sense if the West negotiates from a position of strength, or at least of parity. The 1980s promise to be a decade of harrowing danger. Russia will reach the apogee of her strength in 1983 and if current trends continue, the U.S. will reach its nadir, weaker than it has ever been as a world power. 

Power and its application within the geopolitical structure of the world are not popular subjects for political analysis these days. Certain prominent theoreticians today speak contemptuously of the ‘coaling station’ mentality of those who see national security as the linchpin of international discourse. But they forget that when the British navy had coaling stations for its ships in every ocean, the world was a far safer place. And even in the missile age of nuclear confrontation, conventional power remains a necessary ingredient of national policy. That power must include bases abroad and firm foreign alliances. I had believed in both. After Britain withdrew her forces from east of the Suez in 1968, I had gladly shouldered the burden of protecting the Persian Gulf. In order to meet our new responsibilities, Iran had to become a top-ranked military power, with our own bases and facilities, and the ability to protect them. I was confident that our American and British allies strongly supported these endeavors. How misplaced that confidence had been! 

Even in the first months in exile I was convinced that the Western governments had some plan in mind, some grand conception or overview to stop communist expansion and xenophobic frenzy in an area so vital to the free world’s welfare and prosperity. I had pondered that in Aswan in the first few days after my departure and had discussed it with President Sadat. But the hectic events of the first few months of exile, in Egypt and later in Morocco, had allowed little time for lucid analysis. 

I had intended to go to the United States soon after leaving Iran, but while in Morocco I began receiving strange and disturbing messages from friends in the U.S. who were in touch with the government and from sources within the Carter Administration. The messages although not unfriendly were very cautious: perhaps this is not a good time for you to come; perhaps you could come later; perhaps we should wait and see. 

About a month after my departure, the tone of the messages became warmer and they suggested that I could, of course, come to the the United States if I were so inclined. But I was no longer so inclined. How could I go to a place that had undone me? Increasingly, I began to believe that the United States had played a major role in doing just that. 

This belief was affirmed in the weeks and months following my move to the Bahamas. I had gone there because it seemed a place that offered a brief vacation while not imposing any further on friends like President Sadat and King Hassan II of Morocco. I was anxious to make my own way. And although my decision to leave Morocco and visit the Bahamas had been abrupt, the move went smoothly enough, at least at first. For the journey to the Bahamas, King Hassan had kindly put an aircraft at my family's disposal. When we arrived at Nassau International Airport, we found everything had been arranged through my new advisors, former associates of my good friend Nelson Rockefeller. After greeting representatives of the Bahamian government, my family and I walked over to three waiting helicopters that were to take us to Paradise Island. I felt relaxed and confident. We landed on a golf course where the landing site had been marked with large white Xs. Our vacation home was the comfortable ocean-side villa of the Chairman of Resorts International, the owner of the compound, that included a hotel, tennis courts, and swimming pool. The three bedroom house was far from lavish, but quite adequate for our needs. After the obligatory photo session and brief chat with waiting reporters, I went in for a rest. 

Our stay in the Bahamas was not an easy time. I spent much of it listening to the depressing news coming over Radio Teheran. Khomeini's revolutionary courts had swung into action. Every day stories of new atrocities flowed from my country. My friends and colleagues were being executed by firing squads. The homes of my supporters were sacked and robbed, their bank accounts looted, automobiles and personal belongings stolen. The execution a few weeks earlier of my long-time Prime Minister, Amir Abbas Hoveyda, had distressed me deeply, but it was only the starting point for Khomeini’s hangmen. Just how savage they could be was driven home time and time again by vivid daily accounts of gratuitous cruelty and abuse that was meted out routinely to members of my government and their families by this so-called Islamic Republic. 

When I appeared outside of my house I was mobbed by tourists. Crowds were friendly, Many requested my autograph and expressed their support for me and my family. Through speculative news stories rumors began circulating that I intended to buy into the Islands. My staff was kept busy denying one erroneous story after another. 

Those were difficult weeks for me and my Bahamian vacation was anything but a holiday. 

My contacts with the United States in the Bahamas were minimal. American Ambassador Schwartz called on me only near the end of my stay. My staff, however, was in contact with various other members of the embassy. Through various channels I was assured that my family was always welcome in the U.S. and that I could always go there for medical treatment. But increasingly Washington signaled some uneasiness about my presence. Perhaps that feeling was transmitted to the Bahamian government. Relations there were correct but distant. Three weeks before our visas expired my staff inquired about extending them. Officials said they would get back to us in a week. The applications were referred to higher officials and ten days before the visas expired, we learned they would not be renewed. We had ten days in which to leave the Bahamas! There was no explanation, no expression of regrets, and no further discussions with Bahamian officials. 

I now have my own theories about their change in behavior. Then, they were only vague suspicions as to why we were asked to leave. Although the casino interests are the financial mainstay of the Bahamas, British influence in this former territory has remained strong, as it has elsewhere in the crown's colonies. I have a longstanding suspicion of British intent and British policy which I have never found reason to alter. With the U.S. distant and cool, and the British, as always, hostile, Bahamian Prime Minister Pindling wanted me out--despite the enormous sums I spent there for my ten weeks stay. 

Two days before we were to leave, a senior official of Resorts International contacted Mark Morse, one of my aides, to inquire about my interest in remaining in the Bahamas. This kind of shifting, double-minded policy I would encounter often in exile. It had mired my last months in Iran when I never knew from one day to the next what U.S. policy was, or how reliable it was. 

The immediate question, however, was where next? We did have one firm invitation, ironically from Panama. Gabriel Lewis, formerly Panamas ambassador to the U.S., came to visit me in the Bahamas and invited me to Panama. For various reasons, I was not interested at that time. However, I did send my son, Crown Prince Reza, for a visit. He met with General Torrijos and was given a brief tour of the country, including Contadora Island, which would eventually be my place of residence. 

Mexico was first on my own list of preferred places of exile. While in power I had visited the country and had enjoyed its scenery and people and became friendly with the then Financial Minister Lopez-Portillo. Crash efforts were begun to explore the possibility of a Mexican haven. Several of my friends in the U.S. helped. Henry Kissinger contacted President Lopez-Portillo, as did a number of other people, including Carter Administration officials. Two days before our visas expired in the Bahamas we were invited to visit Mexico. Aides flew ahead to look for a house and found a place on a small street in Cuernavaca, an hour and a half drive from Mexico City. It was a large house that had not been lived in for several years. Lush gardens fell to a river several hundred yards below the house and rolling countryside extended along the other bank. It was a beautiful setting, although the area was thoroughly infested with mosquitoes. 

On June 10 we flew to Mexico and drove from the airport in a small motorcade to Cuernavaca. President Lopez-Portillo had provided much of the necessary security. Privacy was important to me also, for I had enjoyed little of it in the Bahamas, where we were the focus of much public interest and constantly surrounded by people. My health was good at that point. The people I met were friendly, the atmosphere relaxed. I paid a courtesy call on President Lopez-Portillo and resumed a quiet social life. I now had the time and solitude to ponder the geopolitical aspects of the recent events in Iran and to reshape my philosophy on the free world’s future in light of what was happening in my country. 

Many friends visited, too, and helped the process. I was touched and grateful that President Nixon and Secretary of State Kissinger visited me. Both are old and treasured friends, and their visits showed how much they still cared, not only for me but more importantly for the problems we had fought together for so long to solve. I had long discussions with both men and found that our views on geopolitics still coincided, as they had during our common years in power when relations between the U.S. and Iran were so close. 

My friendship with Richard Nixon dates back to 1953 when he was Eisenhower's vice president. Our ties were strengthened both as friends and allies when he became President of the U.S. With regard to foreign affairs, President Nixon has a remarkable vision and understanding of men and events. His policies of disengagement in Vietnam and normalization of relations with the People’s Republic of China were based on reason, common sense and prudence. His rigorous conception of the balance of world power gave the U.S. very definite prestige. 

Before he became President, we had lengthy discussions in Teheran on many geopolitical issues and discovered we had a great deal in common. For instance, we both agreed that a nation must search for alliance with “natural allies,” countries with which it will remain allied by virtue of common and permanent interests. Care must be taken to avoid doubtful alliances which are potential troublespots. A sure and solid ally is worth more than a number of partners who may weaken at the decisive moment. Richard Nixon is precisely one of those Americans who by coming to visit me in Cuernavaca has proven his loyalty to old friends. 

I knew Henry Kissinger under many different circumstances, first as director of the National Security Council and subsequently as President Nixon's and then Ford's Secretary of State. He is an accomplished statesman whose breadth of understanding of American and international affairs is extraordinary. Always faithful to his principles, he served his country with an unremitting awareness of America’s responsibilities for the maintenance of an honorable world peace. His superior intellect is complemented by two qualities unfortunately lacking in many of the luminaries of this world: an ability to listen and a very fine sense of humor. 

We all agreed that the myth of Eurocommunism as a viable alternative to a communist takeover in the West had begun to fade. In France George Marchais had loosened ties to his Socialist partners led by Francois Mitterand. The move did not surprise us. Late in 1977 I had visited Poland and Czechoslovakia and in both Warsaw and Prague listened to anxious communist officials express concern that the united French left might actually win the March 1978 French Parliamentary elections. Nothing was less desirable, I was told. Communists had little interest in power shared with others. They wanted total control. Neither alliance nor coalitions such as those forged by France's Marchais, Italy's Berlinguer, and Spain's Carillo would satisfy. Only the uncompromising stand of Portugal's Cunhal, who had closed in on power and then refused to bargain for it, was acceptable. 

Communist intransigence was to be witnessed elsewhere. More than 40,000 Cuban mercenaries had begun the bitter process of carving up Africa by fomenting war, rebellion and communism from Angola to Ethiopia. Africa is the continent of the future whose raw materials are now vitally demanded by the industries of the West. They will become increasingly important as time passes--and increasingly threatened. Rival tribes disguised as nation states have riven the continent and already the dagger of dissension and race war has plunged into South Africa. China was still industrially weak. Japan unwilling to invest even a fraction of her vast new wealth in self-defense. Tokyo still relied on the American umbrella. I know from personal experience how porous that protection could be. 

Meanwhile, Iran was sliding into chaos. The nation I had led to the threshhold of progress, power, and self-confidence lay torn and bleeding. A worn, fanatic old man was repeatedly telling the Iranian people how mean, miserable, and poverty-stricken they were. The women’s rights I had so painfully established on a basis of dignity and pride were ground into the dust of the Middle Ages. The many projects I had begun but been unable to complete lay fallow. But then what interest could the mullahs have in the nuclear power reactors I was planning and building? The first two under construction would have added 2,500 MW capacity to Iran's power grid and their location in the northern deserts would have assured safe disposal of radioactive wastes. What would happen to schools, hospitals, universities, and other social institutions, to the new factories we were building? I had had plans to make Iran the world’s largest producer of artificial fertilizer. A litany of shattered dreams! 

These dreams haunted me with special poignancy the night we left Mexico for medical treatment in New York. By the time we reached Florida's Gulf Coast, it was dusk. The plane crossed the peninsula and touched down at the Fort Lauderdale airport--unfortunately at the wrong Fort Lauderdale airport. My staff had made arrangements to land at the Executive air field but instead we arrived at the executive jet section of the city’s International Airport. It took the officials who had been alerted to meet us at the smaller airport about an hour to get across town to meet our plane. Meanwhile, a U.S. official approached and asked the pilot if we planned to dump garbage and whether we carried plants. He proved to be an agricultural inspector with no idea who we were. 

During the two hours wait, the Empress climbed down from the plane and walked briefly nearby. I was tired and feverish and uncertain of our reception in New York. The story of my departure had been leaked in Mexico and there had been photographers at the Mexican airport. With the formalities in Florida finally completed, we were able to continue our flight to New York, arriving at LaGuardia before midnight. The plane landed in a remote part of the airport, away from other traffic. Strong security measures had been taken. New York police were everywhere. A small fleet of cars lined up on the tarmac. No TV cameras had been set up and I was grateful for that. 

I was relieved to be in New York and thus assured of the best medical treatment and opinion available anywhere. We entered the cars and drove off. Originally, I had intended to stop at my sisters house in Manhattan and greet the staff there. However, when word came over the police radio that a photographer was waiting in front of the house, I decided to drive directly to New York Hospital. I was taken to two rooms on the seventeenth floor. The surroundings were familiar. Almost thirty years ago, in 1949, I had checked into those same rooms for a routine medical examination during a visit with President Truman. 

The medical team that treated me, led by Dr. Benjamin Kean, who had examined me in Mexico and recommended the move to New York, came in the next morning for a thorough examination and a series of tests. Some 24 hours later, on October 24, I underwent surgery. 

After the operation I issued a statement about my medical condition. In this report, I acknowledged my six-year treatment for lymphoma. In the best interests of my country, I had previously withheld this information. Recently, I had developed intermittent obstructive jaundice--the cause of which could not be determined. The question of whether the two conditions were related or not, demanded sophisticated study and analysis. My physicians at New York Hospital determined that the jaundice was due to gallstones. At 8 A.M. that morning my gallbladder and stones were surgically removed. A stone was also discovered in my common duct, which was semi-surgically removed two weeks later. 

Two days after the operation I celebrated my 60th birthday with my family around me and felt myself regaining strength. The outpouring of affection for me around the world had been heartening. My rooms resembled a florist shop. Finally the overflow had to be put in a waiting room in another building and I asked that these flowers be distributed to other patients in the hospital, hoping their beauty would cheer them as well. Calls, letters, and telegrams poured into the hospital. Thousands of letters accumulated. Many were from average Americans wishing me well and offering me help. I remember one man wrote and offered me a cottage on a lake "where you'll be safe.” There were many similar offers. Others urged me to "feel welcome in this country.” 

The friendship shown me by American citizens has always pleased and amazed me. It is in such stark contrast to media accounts, and, alas, government policy. Demonstrations against me, when they occurred, were always magnified, while shows of support were dismissed or ignored. I remember the protests during my last state visit to the U.S. in 1977 when about 50 people demonstrated against me in Williamsburg while 500 were demonstrating support for me. The media switched the numbers and asked rhetorically who had paid to bring the Shah's supporters. No one bothered to answer my own question: Who had paid the anti-Shah demonstrators to come? Surely they were paid, for it was hardly an Iranian demonstration. The crowd was dotted with black faces and blond manes, rarely found in Iran. 

During my stay in New York Hospital there was little contact with the U.S. Administration. President Carter never phoned or sent a message, neither did any other high U.S. official. As I gained strength, however, a stream of visitors came to see me. I watched some television and wondered again at the media's obsession with the small groups of anti-Shah demonstrators that paraded near the hospital. As usual, little note was taken of expressions of support: one that cheered my staff involved a small plane flying up the East River floating a streamer that said “Long live the Shah.” 

On November 4, two weeks after my arrival in New York, militant fanatics in Teheran occupied the American embassy and seized more than fifty hostages. There is little I can say about that act of villainy, allegedly committed to “punish” the United States for offering me a medical haven. Any detailed comment would be inappropriate, even today. Nevertheless, the incident had a profound impact on my own life. Although Washington still did not communicate with me directly, the signals were unmistakable. The Administration wanted me out of the country just as quickly as was medically possible. For my part, I had no desire to stay any longer than absolutely necessary. 

Thus, on November 8, I publicly expressed my willingness to leave the United States in hopes of freeing the Americans being held hostage. However, my doctors’ position was that any travel for me at that time could well be fatal. I reiterated that my friendship with the U.S. remained unimpaired, and pointed out that during my reign 45,000 Americans had lived in Iran in “peace, tranquility and prosperity.” 

The first reaction to my statement came not from Washington but from Cairo. President Sadat dispatched Ashraf Ghorbal, Ambassador to the United States, to the hospital with an offer to return to Egypt for further medical treatment in Cairo. I was touched, of course, but unwilling at that time to impose once again on the kindness and generosity of my friend. The house in Cuernavaca was perfectly adequate for our needs. Although my visa to Mexico expired on December 9, I foresaw little trouble in renewing it. President Lopez-Portillo had told me personally on two occasions to “consider Mexico your home. You are welcome here.” And according to newspapers accounts, the Mexican government had quietly informed the U.S. that there would be no problem about my return. Again, I believed what they said. For all my growing disillusionment with the West, I still had faith. 

I refused to criticize the West then and only do so today with great hesitation. It is increasingly obvious that Western policy in Iran, and indeed around the world, is dangerously short-sighted, often inept, and sometimes downright foolish. I base these conclusions upon many recent observations. To cite just one, a television broadcast of UN Security Council proceedings on the hostage crisis. There on my screen was Anthony Parsons, now the British Ambassador to the United Nations. A year before he had been Her Majesty's envoy in Teheran. I could hardly believe my ears! I can only recall the gist of his remarks. “Let these people come,” and explain their revolution to us--and he meant the members of the Revolutionary Council who had already massacred so many innocent people. This was the same Parsons who told me in the fall of 1978, when I planned free elections, that if I lost them--and my throne--I would go down in history as a ruler who had lived up to his democratic ideals. 

This performance was a classic example of the West's double standard. As anally, I was expected to live up to the West's idea of democracy regardless of its infeasibility in a country like mine. But this so-called Islamic Republic which makes a mockery of all Western ideals was cordially invited to the UN forum to educate the delegates in the new “morality” of the so-called Islamic Revolution. 

As I watched Parsons’ incredible performance I began to wonder if there had ever been any coherence to Western policy toward Iran beyond a successful effort to destroy me. The British hand has lain heavily on Iran for most of this century. This did not really change after the American entry. Western support of my rule had always been tempered by a need to exercise a sufficient amount of control. True, the definition of “sufficient” varied with changes on the international scene, but Western efforts to “clip my wings” go back to Mossadegh’s day. They were revived whenever I struck out on my own. 

The international oil companies were long-time adversaries. After Mossadegh’s defeat I roused their anger by negotiating an agreement with Italy’s Enrico Mattei. He had built the Italian oil company Ente Nazionale Idrocarburi (ENI) into a major but maverick competitor of the international giants. Our agreement, in and of itself, was not large but its terms were significant. Instead of splitting profits fifty-fifty as we had been doing, Mattei agreed to take only 25 percent for himself with Iran receiving 75 percent. Shortly afterwards I made the same arrangements with Standard Oil of Indiana. The fifty-fifty principle had been broken and Big Oil never forgave me. By 1959, two years after the ENI agreement, the first student demonstrations against me were orchestrated in, of all places, the United States. I suspected that Big Oil financed the demonstrations and that the CIA helped organize them. I know this sounds contradictory since both of these powerful interests had also supported my rule. But I do believe now that the West created an organized front against me to use whenever my policies diverged from theirs. I should have believed it twenty years ago when my Prime Minister, Sharif Emami, warned me that the U.S. was behind the student agitation both inside Iran and out, and was busy fomenting other trouble as well. 

At the time Prime Minister Emami took the brunt of the animus, as was his legal responsibility. The U.S. wanted him out and its own man in as Prime Minister. This man was Ali Amini, and in time the pressure became too strong for me to resist, especially after John F. Kennedy was elected president. John F. Kennedy was never against me. I considered him a friend although we had little direct contact. I remember so well my first meetings with the Kennedys at the White House: Jacqueline Kennedy spoke of Amini's wonderfully flashing eyes and how much she hoped I would fame him Prime Minister. Eventually I gave Amini the job. There have been rumors that Kennedy offered me a \$35 million aid package as an inducement. These rumors are totally unfounded for it was Amini who obtained this money from the United States after he became Prime Minister. But he mismanaged affairs so badly that he was soon asking the Americans for another \$60 million, which was refused, 

After Amini's failure, I launched the series of reforms known as “The White Revolution,” and for ten years the West muted its agitation against me. However, it erupted in full force again after the 1973 oil embargo and my decision to raise world oil prices. Throughout the seventies opposition mounted and in the end created a strange confluence of interests--the international oil consortium, the British and American governments, the international media, reactionary religious circles in my own country, and the relentless drive of the Communists, who had managed to infiltrate some of Iran's institutions. I do not believe that this convergence of forces represented an organized plot against me in which each part meshed with the others. But clearly all the forces involved had their own reasons for pushing me offstage. Throughout 1978, the oil consortium refused to sign a new agreement with Iran to purchase oil. This coordinated action--or inaction--had incredible significance. I believe they somehow had foreknowledge of the events that were to take place later that year. I also believe that members of the Carter Administration--especially the McGovernites in the second echelon of the State Department--were anxious to see me leave in favor of this new so-called “Islamic Republic.” Their strategy, if indeed they have one, appears to assume that Islam is capable of thwarting Soviet ambitions in the region. I wonder with what? The media for its part focused on those human rights activists who deplored my rule and kept pressing for the reforms that ultimately led to disaster. 

I think all this would have been easier to bear if there had been some coherent policy behind the confused and contradictory actions taken by my friends and allies. For many months I believed that such a plan existed, I have repeatedly pondered the question of Western intent and Western policy without reaching any logical conclusion. Despite the evidence, I find it difficult to believe that the Iranian disaster was simply the result of short-sighted or non-existent policy and unresolved conflicts within the American government. Yet analysis both of the past and of events since the seizure of the hostages does not allow any other conclusion. Consider my own fate! 

By the end of November 1979 the U.S. wanted me out of their country at almost any cost, and I was as eager to go. By the 27th my doctors reported that radiation treatment on my neck had been completed, and an attack of cholangitis with high fever brought under control. A stone in the common bile duct had been crushed and the particles removed. Though my doctors called the outlook “guarded,” I was recovering. I wanted to return to Mexico as soon as possible. 

Two days later, on November 29, the Mexicans dropped the next bombshell. That morning my aide in Mexico received confirmation from the Mexican authorities that my invitation was still valid. While making final arrangements for my arrival, he contacted my New York aide to verify the confirmation. However, in New York my aide was shortly thereafter informed by the Mexican Consul General that these plans were changed. Within less than three hours, the Mexicans had done an about face and rescinded my invitation. Foreign Minister Jorge Castenada then made the announcement official at a press conference in Mexico City. My return would be contrary to Mexico's “vital interests. He did not explain the nature of the vital interests. Press reports later quoted Mexican officials as fearing militant attacks on their embassies in the Middle East and Europe. The explanations seemed weak. I still don’t know what motivated Mexican policy. They have plenty of oil for their needs and therefore had little to fear from Mideast producers. Perhaps the government hoped to play a larger political role in the councils of the Third World and feared my presence would dash that hope. I have heard accounts of a deal Cuba offered: bar the Shah and Castro would give up deadlocked efforts to win a Security Council seat and throw his support to Mexico. This theory has some plausibility. Cuba dropped out of the race and Mexico was elected. 

What next? I had no quarrel with the Carter Administration about leaving, but my options were few. In addition to my own hesitation about returning to Egypt, the U.S. government feared my presence would be detrimental to President Sadat’s relations with other Arab states, a groundless concern as later events would show. Panama was a possibility, as was a return to the Bahamas, although neither seemed attractive. Thus, Washington offered to let me recuperate at Lackland Air Force Base near San Antonio, and I accepted. On December 2 a U.S. Air Force plane flew us to Texas. Our departure from the hospital resembled a "getaway" scene from a 1930s gangster film. About fifty heavily armed FBI men guarded all the doors and exits and were posted in the street and the inside hallways. 

Lackland Air Force Base is a training facility where a number of Iranian pilots underwent flight instruction. It is probably one of the least secure bases in the U.S. Daily, 30,000 people come and go through this military base as they would through a shopping center. There are no fences, few restricted areas. In making arrangements for my arrival the Pentagon had issued minimal instructions. The base command had little idea of how ill I was or what security requirements were necessary. When we arrived, therefore, we were put in a hospital van and driven to the most secure part of the hospital--the psychiatric ward, rooms with barred windows and locked doors. It appeared as if we had been imprisoned. The Empress grew claustrophobic. We simply could not remain in those quarters. The base commander was apologetic and friendly. He readied the visiting officers’ quarters for us. 

Once we were installed in the officers’ quarters, things improved. General Acker, the base commander, and his top officers went out of their way to be friendly and helpful. The weather was good. My health improved. I went for walks and several times had dinner with the general and his aides. The Empress, who is a physical fitness enthusiast, enjoys tennis, and General Acker recruited some worthwhile competition for her. Some of the military on the base had served in Iran so we had an opportunity to renew old acquaintances. There was little Opposition to my stay. Much of the pressure we had felt in New York began to subside, 

True, we still had no place to go, but it was now Washington's problem to assist us, All the options previously explored were repeated. Austria and Switzerland were asked to take us and both again said no, even though my relations with Chancellor Bruno Kreisky of Austria had always been good and I had owned a house in Switzerland for many years. South Africa was discussed, as was Britain. Shortly after I left Iran I was informed that Margaret Thatcher had assured us we would be given asylum in England should she win the upcoming British elections. After she became Prime Minister, we were told it would be awkward for her to have us come. That position did not change. 

The Carter Administration came up with an alternative, ironically one which I already had--an invitation to Panama. White House chief of staff Hamilton Jordan arrived at Lackland one day in December directly from Panama City, He had discussed a visit for me with General Torrijos and found him receptive to the idea. I discussed the proposal with my staff and decided to send my aide, Robert Armao, and my Iranian security chief to Panama with Jordan for further discussions. They visited a distant mountain resort four hours from Panama City, a location in the capital, and Contadora Island. They found the mountains were lovely, but too isolated, and Panama City too crowded and noisy. Contadora now seemed the best choice, and my aides returned with a hand-written note from General Torrijos assuring me of a warm welcome. 

Jordan, joined by Lloyd Cutler, White House special counsel, then met with my staff and me to work out the necessary arrangements. My attorney, William Jackson, Dr. Kean, and my aides, together with the White House officials, finalized the plans for my departure. More importantly, an oral agreement was reached. It guaranteed full U.S. support for me should any medical and safety problems arise: Cutler and Jordan promised full White House support for the move; I would be assured access to Gorgas Hospital in the former Canal Zone, an American military installation with modern and up-to-date equipment. Dr. Kean had some doubts about how well the hospital had been maintained but acknowledged that it had once been among the finest. What is more, the Administration's men said, the Panamanian hospital at Paitilla was also first rate. And, in case of any real emergency, I would still be able to return to the U.S. Jordan assured me of President Carter's full support. The U.S. still maintained a strong military presence in Panama, another guarantee of my safety and care. 

Both Jordan and Cutler were friendly and courteous throughout our discussions. The case they made for Panama became persuasive. In addition to the U.S. presence and the availability of U.S. medical facilities, Panama was sympathetic to the West. It had no diplomatic ties with Iran and presumably was immune to any threats from Khomeini. Since we had very few other options, it seemed a good solution. 

After we reached an agreement with the White House, President Carter telephoned. He warmly wished me good luck and reiterated the assurances of his aides. It was the first and only time I had spoken with the President since wishing him farewell on New Year's Day 1978 when he visited Teheran. 

We had but one day to pack. Before dawn on Friday, December 15, our small motorcade drove out of Lackland for nearby Kelly Field where a U.S. Air Force transport plane waited to fly us to Panama. Shortly after 7 A.M. we took off. The American promises were still ringing in my ears. 

Our first weeks on Contadora were pleasant enough. The compound somewhat resembled Paradise Island. We were ensconced in a four-bedroom beach house owned by Ambassador Lewis. The island was 30 miles off the Panamanian coast in the Pacific and our house commanded a splendid view of the ocean. The heat and humidity were good for my throat, which was sore from the radiation treatments. I saw several Panamanian leaders: General Torrijos and President Royos invited me and my family to their homes and visited us on several occasions on Contadora. Both men were frequent luncheon guests on the island. 

David Frost brought a television crew to the island to film an interview arranged months earlier in Mexico. I enjoyed the challenge and the intellectual stimulation which the interview afforded. Over the years I have always enjoyed friendly combat with the media. Few correspondents of major media who spent any time in Teheran ever had much trouble obtaining an interview. Talks with foreign journalists, no matter how outrageous their questions, always gave me the opportunity to explain my views on major issues. Besides, argument and debate sharpen the mind and help clarify one’s own thinking. Frost is a skillful interviewer. 

Too often the media, especially American, came with set notions of what Iran ought to be, rather than what it was really and more importantly what it was becoming. Iran had been propelled abruptly from the Middle Ages into today’s technological world. To compare such a nation to countries with centuries of democratic traditions and histories of literacy and learning is much like comparing apples to oranges. They are simply not comparable. Glib answers to complex problems are worse than no answers at all. 

America’s postwar history is an uninterrupted demand that the rest of the world resemble America, no matter what the history--political, economic and social--of other nations might have been. The example of Vietnam haunts me still. Unlike the French, who had a sense of what could and could not be done, the U.S. set out to build a new nation in Vietnam modeled on itself. Ngo Dinh Diem refused to bend his policies to an unrealizable democratic ideal as propounded by dogmatic young journalists. It was apparent that the Kennedy Administration ordered Diem’s removal. It is worth noting that on the day he died, Diem was on the offensive against the communists, and that on the day afterwards the initiative had passed to the Vietcong and the North Vietnamese. Over the next twelve years the Americans and the South Vietnamese never regained it. 

Democracy is an historical process that cannot be imposed by fiat, either from the bottom or the top, though my own experience suggests that gradual introduction from the top that allows time for adjustment is more effective than violent upheaval from below. My harshest critics no longer even suggest that the barbaric regime of the mullahs in Iran today is more democratic, more just, or more effective than my own. The world has rarely witnessed such a demagogic regime. In fact, I am struck by how few comparisons are made between us in the West. Few see the contrast. The Shah is put in one box, Mr. Khomeini in another, and it seems as if we ruled different countries. 

By the time the Frost interview was aired in the U.S. on January 17, 1980, the early tranquility of my stay on Contadora had begun to fade. U.N. Secretary General Kurt Waldheim had undertaken his disastrous mission to Teheran and earned only contempt and ridicule for his pains. Khomeini and his henchmen never took seriously the Waldheim offer: a trade of the American hostages for a United Nations investigation of my alleged crimes. Nevertheless, Waldheim persisted for two months in chasing his illusions, giving up only when the mission he had dispatched to Teheran in March returned empty-handed to New York. They had never seen the hostages, let alone freed them. But Ghotzbadeh and the others had displayed a myriad of lies before the panel and to the world via television. It never occurred to any of the media to investigate whether or not any of the children allegedly tortured and disfigured by my police were in fact victims of other tragedies--accidents or birth defects. 

On January 12, Iran’s new rulers opened the next phase of their unrelenting war on history and on myself: they demanded that the Panamanian government arrest me. The move did not surprise me but the hesitation shown by my hosts did. Instead of treating the demand with the contempt it deserved, Panamanian authorities began contributing to rumors that they were indeed in contact with Teheran, and negotiating better arrangements than the U.N. had offered. It was the beginning of a strange and ominous double game. For even as the stories circulated, the Panamanians simultaneously hastened to assure us--in secret--that there was no way I could be extradited since such extradition would violate Panamanian law. A week after the Frost interview Ghotzbadeh fired the next salvo. He claimed I had been placed under house arrest in Panama. On January 24 the Panamanian government issued an official denial. Yet days later the government waffled on the denial. There were accounts in the press of a “technical possibility” of my extradition. Again, we were given private assurances to ignore these reports. 

This charade continued into February. On February 7 Panama's Foreign Minister said I was a “virtual” prisoner because I was not free to leave Contadora without Panama's permission. Gradually government pressure was increased, directly and indirectly. My staff noticed security growing lax and the tapping of our phones continued. On one occasion an aide was discussing on the telephone the high costs of our stay. The next day the Panamanians complained that too many details about money were being discussed on the telephone. A \$400 tape recorder-- for which we were forced to pay--was set up in the house to record all our telephone conversations. Money pressures were anything but subtle. My staff complained about bills that seemed much too high. Friends of General Torrijos let me know that Contadora was for sale and had a \$10 million price. We were shown property, as we had been in the Bahamas, and again at inflated prices. But there was nothing we would consider buying. Both the White House and Torrijos had assured my staff that we would not be victimized by price gougers. Torrijos had gone out of his way to direct any complaints to his office. 

Increasingly I began to sense that an effort was underway to isolate me from the rest of the world. On one occasion I flew to Panama City for a secret meeting with the American Ambassador who said he had a message from President Carter. My advisor, Robert Armao, planned to come with me to the capital, but Panamanian officials refused to let him attend, The U.S. was probably as eager as Panama not to let me be seen with my advisors--because they were American. Carter's message was that I should not worry, that everything was under control. Again “reassurances.” 

Pressure mounted against my American staff. Panamanian officials charged them with leaking false stories to the American papers. But they had nothing to tell. Any stories about me at that time probably were generated by U.S. government officials. Thus, after we had complained about overcharges, a story appeared in the Washington papers detailing my interest in returning to the United States. I had no such plans, of course, and certainly not after my recent experiences there. But press speculation continued and so did gentle harassment by our Panamanian hosts. I could see that Panama was not a permanent haven. 

The Panamanian government was still playing extradition games with Iran. Two Paris-based lawyers, a Frenchman and an Argentinian, were employed by Khomeini to handle the legal work. They drew up a 450-page brief and journeyed around the world seeking support for this document which would have been laughed out of court in any civilized country. However, there was some doubt as to whether the Panamanians would dismiss the charges that easily. Khomeini’s Panamanian lawyer argued that I could be extradited if the Teheran government promised not to execute me. 

Early in March one of my American aides, Mark Morse, was arrested by Panamanian authorities and held for several hours before U.S. Embassy pressure obtained his release. He had been charged with interference in Panamanian security. In fact, authorities were angry at Morse’s insistent complaints of overcharging and bill-padding. 

The die was finally cast by my illness. The cancer had flared up again in February and spread to my spleen. Dr. Kean visited several times from New York to examine me and to consult with my French physician, Dr. Flandrin. Both recommended surgery and Dr. Michael De Bakey of Houston was requested to perform the operation at Gorgas Hospital, the U.S. military facility in the Canal Zone. That possibility, of course, had already been discussed during the negotiations between my staff and Cutler and Jordan at Lackland Air Force Base. Then, other problems arose. The U.S. ostensibly had always held Gorgas Hospital open to us. However, the Panamanians now demanded that the operation be performed at their Paitilla Hospital. 

The next act in this bizarre drama resembled a medical soap opera. My medical team and advisor met with Dr. Garcia, General Torrijos’ private physician and part owner of Paitilla Hospital. Garcia was adamant that the operation be done at Paitilla. Dr. Kean argued the facilities there were not adequate, that proper blood equipment and laboratories were at Gorgas Hospital. Blood would have to be taken from Paitilla and rushed to Gorgas for analysis. The discussions grew heated. The Panamanian doctors were emotional and hot-tempered and resented the foreign doctors’ presence. We found it difficult to understand why any medical professional would let false nationalist pride override the welfare of a patient. Dr. Garcia bluntly stated that “we're just following President Carter's orders” and that he didn't care what we thought and that we were to do what we were told. Finally, Dr. Garcia ended the discussion with an ultimatum: “This is it. You have no choice. You either go to Paitilla or head for the airport. You are going to use our hospital.” 

My staff was outraged but at that moment there seemed little we could do. It was clear the U.S. wanted to keep us in Panama in order to keep playing games with Iran using me as bait for the release of the hostages. I would remain a "beloved American prisoner” on the beautiful Island of Contadora. 

On March 11 we decided on the inevitable surgery. Dr. De Bakey had agreed to perform the surgery at Paitilla. Arrangements were made for moving my entourage from Contadora to Panama City. Three of my sisters flew to Panama to be with me. On March 14 I checked into the hospital. 

A short time later, Dr. De Bakey, Dr. Kean, and their U.S. medical team arrived. Unfortunately, once again what appeared to be, was not. The Panamanian doctors now refused to let Dr. De Bakey operate. They claimed he was merely an itinerant surgeon and this “routine” operation could be performed by Panamanian doctors with no problems. Their pride appeared at stake. I would not be operated on by American surgeons and that was that. 

I considered their attitude insane. My life was in jeopardy, and I was not about to lose it to the personal insecurities of the Panamanians. My doctors counselled against an operation at Paitilla, and agreed that surgery could be postponed for two weeks without imminent danger. The next morning I left the hospital and returned to Contadora. 

At that point worldwide news reports began to detail my plight. Fortunately, Mrs. Sadat had called the Empress to express concern for my health and safety. She invited us on behalf of President Sadat to Egypt for my much needed medical care, with any doctors I desired. President Sadat would send his plane immediately. 

I decided to accept President Sadat's kind offer--a standing invitation since the day I left my homeland. I have always considered him to be a noble friend and a man of honor. During those difficult days for my family and me, these sentiments were emphatically confirmed. He and Mrs. Sadat called a number of times during my stay in Panama. The message was always the same: “Why don't you come to Egypt? You are welcome here.” 

On March 21, Hamilton Jordan arrived in Panama. He telephoned on arrival to say that he was coming to see me. However, Jordan never came. At the same time, he contacted Lloyd Cutler who later flew to Panama with Arnold Raphel, a top aide to U.S, Secretary of State Vance. Upon his arrival in Panama, Cutler called to inquire if he could come to Contadora with a message from President Carter. We agreed, even though I knew that the Empress had already spoken with Mrs. Sadat and indicated my interest in moving to Egypt. 

When Cutler arrived he insisted on seeing me without my aides present. I reluctantly agreed. I had dealt with Cutler during our Lackland negotiations, where I had found him a man of substance and tact, sure-footed in the byways of diplomatic intercourse. Now in Contadora he outlined the U.S. position with skill and detachment. My trip to Egypt, he said, could endanger Sadat's position in the Arab world, particularly the Mideast peace process. Houston was a possibility for my surgery, of course. The terms of the Lackland agreement had been specific on that point. Nevertheless, any such move during the delicate negotiations now underway could endanger resolution of the hostage crisis. The best solution, clearly, for Carter, was for me to remain in Panama. Now the concessions came quickly. Of course, the operation could be performed at Gorgas; Panamanian doctors were anxious to apologize to Dr. De Bakey. Cutler was persuasive but my decision was virtually made. Still, I agreed to consider his proposal and to see him again the next morning. 

I did not seriously consider the American offers. For the last year and a half, American promises had not been worth very much. They had already cost me my throne and any further trust in them could well mean my life. When Cutler returned to my house the next morning he looked at our packed bags and understood. He did not bother trying to change my mind. Instead, he called the White House to discuss the necessary logistics. He suggested it would be wiser to use a U.S. chartered plane than to wait for Sadat’s aircraft with its refueling and landing rights’ questions. More calls followed. Cutler and his aides managed to charter an aircraft. The plane would meet us at Panama City International Airport. 

After my decision to leave had been made, Panamanian demeanor changed. They were extremely helpful. Guards were now eager to assist us. Medically, time was of the essence: I was running a fever; my blood count was dangerously low; the blood platelet count dropped to less than 10 percent of normal. Thus, the trans-Atlantic flight would be risky, for if I cut myself at so high an altitude, I might well bleed to death. 

We boarded the plane in Panama City. It was a long, uncomfortable flight, for the seats were narrow and cramped. We landed in the Azores to refuel. A Portuguese general and the American consul were waiting at the airport. Though sick and feverish, I rose, straightened my clothes and prepared to receive them, as protocol required. 

Finally we landed in Cairo where President Sadat and Mrs. Sadat waited to greet us in the bright sunshine. An honor guard was posted behind them. I walked from the plane to President Sadat and his wife and warmly embraced him. 

“Thank God you're safe,” Sadat said, and I was safe indeed. 

From the airport we flew by helicopter to Maadi Military Hospital on the Nile just outside Cairo. Doctors began at once to work on bringing down my temperature and building up my blood count. A few days later Dr. De Bakey and his medical team arrived in Cairo with their sophisticated medical equipment. Surgery was immediately necessary. X-rays revealed that my spleen was dangerously enlarged. 

The surgery was performed on March 28. The removed, cancerous spleen weighed four-and-a-half pounds and resembled an oversized football. A normal spleen is the size of a man’s fist. 

When I was finally well enough to leave the hospital I joined my family in the Koubbeh Palace, six miles north of Cairo. This is the Egyptian residence for all visiting heads of State. It is set in the midst of a large park with fruit trees and gardens, encircled by a wall and well secured, This lovely home offered the first peace, quiet, and security we have known since leaving Iran. Its serene beauty has been a source of strength during my convalescence. Not since my stay in Mexico have I enjoyed such time for reflection and thought. 

Of course, I continue to focus on events in my homeland, past and present. Certainly, I had made mistakes in Iran. However, I cannot believe they formed the basis for my downfall. They were rectifiable with time. My country stood on the verge of becoming a Great Civilization. 


The forces against me, however, proved stronger, although they were gathered without unified motive or larger purpose. 

Much of the opposition aroused in the West seems to have been triggered by ignorance, and a warped view of what Iran should be. The West has never understood my country. We were ignored for centuries. When we re-entered modern consciousness, it was only as a geographic cross-roads. We were merely a guardian of trade routes to the East, a savage and barbaric land of no intrinsic worth, whose importance lay in political realities. I have never denied those realities but have never understood British and American inability to recognize Iran as a truly independent nation. We live in that trapezoid of land linking the Near East to India and we are the Western wedge against Russia's centuries-old dream of warm water ports in the Persian Gulf and the Indian Ocean. 

Part of the answer, I think, lies in the West's lack of interest in Iran's history and its failure to understand the differences between Persia, both ancient and modern, and itself. My own answer to history, therefore, must begin with the history of my country, the 3,000 years of Persian civilization that, misunderstood, has led to the defeat of Iran's attempt to enter the twentieth century, perhaps presaging an even greater defeat of the countries I considered friends and allies. 