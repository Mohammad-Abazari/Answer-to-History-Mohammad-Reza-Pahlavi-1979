NONE OF THE GOALS of our White Revolution could have been achieved if we had not been able to maintain peace with our neighbors in a strategic region of the world of which Iran is the focal point. 

My essential objective in foreign policy was therefore to have good neighborly relations with all countries of the region. 

Although we had no belligerent intentions toward others we made it known that Iran would resist aggression from whatever quarter. 

Once our position was well understood, we were able to satisfactorily negotiate a number of issues with the Soviet Union. We reached agreement on matters concerning our common frontiers. We divided the waters of the Arax River according to international law and constructed (on a 50/50 basis) a dam on this river which produced electricity and permitted irrigation of vast agricultural areas. We were working on similar 50/50 projects with the Russians to provide additional irrigation and electricity from the same river. Our common production of electricity would have reached one million KWH in the eighties. 

The considerable volume of trade with the Soviet Union made it one of our principal partners. We furnished the Soviets with gas while they built an important steel mill for us in Isfahan. The Americans, under Eisenhower, had been undecided on undertaking this project. The Russians also helped us to prospect and discover iron and coal mines north of the desert (south of Khorasan) and near Kerman. The volume of Western merchandise shipped through Russia to Iran reached enormous proportions and became an important element in our trade relations. 

Quite naturally, I had many occasions to meet with Leonid Brezhnev in the Soviet Union and Iran. Ideological differences aside, I sincerely admire Mr. Brezhnev. He is a superb diplomat who has brought his country to the apogee of power we see today. The Soviet Union is the first nuclear power of the world and soon will be the first naval power. As for ground and airborne forces, its superiority is so great that it defies comparison. 

I must add that we purchased large supplies of military equipment from the Russians as well as the Czechs. I was always cordially received in Czechoslavakia and the other socialist countries of Eastern Europe, and made great friends with leaders of several of these nations. 

Yugoslavia is the only country, along with Iran, to have stood up to Josef Stalin under difficult if not anguishing circumstances. It was not easy to unify various ethnic groups and modernize a country like Yugoslavia. One must understand that Marshall Tito accomplished an extraordinary task. May God make his successors show themselves to be as capable as he was. 

President Ceausescu of Rumania impressed me with his intransigent patriotism and his fierce, independent will. He is a strong leader capable of insuring his nation’s independence. Through our meetings and conversations, we became good friends. This relationship deepened as our mutual trade endeavors grew. His loyal friendship has been sincerely appreciated during these difficult times. 


Since my father's time Turkey and Iran have been faithful friends. For Iran, the prosperity and greatness of Turkey were of fundamental importance. Today I fervently pray for the happiness of this valiant people. 

Our alliance was cemented by the 1956 Baghdad Pact. My meeting with Khrushchev took place shortly afterward. He was not pleased and welcomed me by saying “This pact is aggressive, directed against us!” 

I pointed out to him that in political and journalistic circles there was talk of a line of defense passing through the Zagros mountains. And I asked, 

“Where are these mountains, in Russia or in Iran?” 

“In Iran,” 

“It is then surely a defensive pact,” I responded. 

Kruschev added, still in a truculent vein, “Don't make me laugh with your pacts... ! You know perfectly well that we could flatten England with seven atom bombs and Turkey with twelve!” 

He was not a man for conversation--sometimes gruff, always obstinate. But his peasant upbringing, which made him alternately good natured and cunning, also made him likeable. Together we agreed on a policy of neighborliness which good sense dictated and we held resolutely to it. 

Later Khrushchev declared that the Baghdad Pact would break up like a soap bubble. Twenty-five years later one has to admit that he was right. 


During the Algiers oil conference in 1975, I spoke at length with Saddam Hussein, the president of Irag. We agreed to bury our differences and succeeded in ending the misunderstandings which colonialist influences had maintained between us. 

President Hussein agreed to negotiate the question of the River Chatt-El-Arab according to international law. As in the case of the Arax River, the waters were divided midway between Iraq and Iran. All our land disputes were also settled. 

In principle, I told President Hussein that the happiness and prosperity of Iraq were important to the security of Iran. 

This same principle led us to aid Afghanistan when it was facing certain economic difficulties. Subsequently, a change of government and of political orientation took place in Afghanistan. The Western powers were not in the least concerned. 

We recognized the new Afghan regime and continued our economic aid. I was puzzled by the West's indifference and concerned about the implications for Iran. Was it a question of a policy change of the superpowers in this region? 

I was the first head of state to visit Pakistan when it gained independence. We were always faithful allies of this new republic which we aided economically and militarily. We worked for the establishment of peaceful and lasting relations between Islamic Pakistan and India. An Indo Pakistani confrontation always appeared to me to be terribly dangerous. 

This is why I wanted to take advantage of the presence in Persepolis of the then president of Pakistan, General Yahya Khan, on the occasion of the 2,500th anniversary of the Persian Empire. I hoped to arrange a meeting between him and the President of the USSR, Podgorny, and thus to help avert the impending conflict between India and Pakistan 
over Bangladesh. 


We also had friendly relations with our neighbors on the other side of the Persian Gulf--Kuwait, the Emirates, and especially Saudi Arabia. 

I had traveled on several occasions to Saudi Arabia, a country whose integrity and independence are sacred for all Muslims. Twice I had the great joy of making the supreme pilgrimage. As a faithful Muslim and Defender of the Faith, I hope that Saudi Arabia will always remain the guardian of these holy places, Mecca and Medina, where millions of pilgrims travel every year on the path to God. 

History has recorded the stature of Ibn Saud, founder of Saudi Arabia. He was wise and brave and an excellent administrator. When one considers the fatal events for which Iran is now the theater, one cannot but rejoice at seeing Saudi Arabia still free and independent. One can only pray to God that it remains so. 

In 1973, at the request of the Sultanate of Oman, I provided this state with military aid. Oman was then threatened by the Zofaris, who were supported by Southern Yemen, the communists, and the Chinese. Our troops in Oman intervened in a vigorous way until the Sultan, who is my friend, succeeded in dominating the situation. 

China retired from this conflict after establishing diplomatic relations with Iran. China was not playing a double game; at the time of President Hua Kuo-Feng’s visit to Teheran in August 1978, the Iranian crisis was reaching its peak. I was impressed by the Chinese leader's integrity and very sound knowledge of international politics. After talking to him, I realized China was among the few nations interested in a strong Iran. 

I felt that a policy of solidarity was necessary among the Persian Gulf countries, who were faced with increasing dangers. 


Similarly, all of the countries vitally interested in the Indian Ocean-- Iran, Pakistan, India, Sri Lanka, Bengladesh, Burma, Malaysia, Thailand, Singapore, Indonesia, Australia and New Zealand and all the countries of East Africa--should combine for their collective security. A Southeast Asian pact was supposed to provide this protection but it was practically dead. 

I deeply felt that the creation of a zone of peace and stability around the Indian Ocean would serve the cause of world peace. It could have succeeded. However, would it have been tolerated? It would have precluded both Russian and American intervention. Could the Soviet Union and the U.S. admit that their armed presence in the Indian Ocean was not necessary? 

During a visit to Australia in 1974, I proposed, the creation of a common market for countries bordering on the Indian Ocean. I made the same proposal in Singapore and in India. This concept was everywhere well received. 

Under my plan, after studying the possibilities and needs of the member countries, a program of exchange and mutual assistance would be put into action. I had declared, for example, that Iran would be ready to contribute to the industrialization of India, and the development of its mines and agricultural lands, 

Without waiting for the realization of the common market, we furnished economic aid to the Sudan, Somalia, and even to Senegal and other countries in West Africa and in the interior of Black Africa. I intervened with leaders of South Africa in an attempt to find an acceptable solution for the Namibian affair. I also met with all Black Rhodesian leaders in order to encourage an equitable and peaceful solution there. This move for peace brought me at that time the thanks of the Anglo-Americans. 

Iran, which is only separated from Africa by the Arabian Peninsula, the Red Sea, and the Indian Ocean, was concerned to see communist penetration into Africa along three axes: The first, going from Libya toward Chad, the Sudan, and Somalia, is the Mediterranean-Red Sea-Indian Ocean axis; the second aims to link the Mediterranean to the Atlantic by land; and the third cuts Africa in two from Angola to Mozambique. The axes of Communist penetration in Africa are real dividing lines. Both the longitudinal and transverse axes sever the African continent, This penetration is a vast strategic movement which threatens to destabilize the whole of Africa. Tomorrow what is called Black Africa could become Red Africa. (In an effort to thwart such actions, I had dreamed of contributing financially to a modern railway line linking the east and west coasts of Africa.) 


It was always with great pleasure that I conversed with the emperor of Ethiopia, Haile Selassie, who displayed great patriotic force in resisting the Italians in the mid-thirties. Our conversations were frank and animated and I sometimes allowed myself to suggest certain reforms to him. 

I was a young student when I heard him defend his country without success before the high tribunal of the League of Nations in Geneva. The League was powerless in those days, as is the United Nations in our times. What has happened to Ethiopia today? 

We were envisaging giving increased aid to the Ivory Coast presided over by Houphouet-Boigny and to Gabon and Senegal. 

Especially strong bonds linked me with President Senghor of Senegal, who influenced the development of my African policy. A diplomat of international prestige and a remarkable administrator, Leopold Senghor is a master of the French language and a true poet. We spoke at length concerning the doctrine of “negritude” which is truly a spiritual synthesis. I empathize with this doctrine, for throughout my life, my Persian roots have been an integral part of me. 


Egypt, Jordan, and Morocco happily are in a stable condition thanks to three exceptional men, President Sadat, King Hussein, and King Hassan II. 

We must not forget that having succeeded Nasser, Anwar al-Sadat found himself at the head of a country which was not only crushed but humiliated. In Egypt, public opinion had been led astray by deceptive slogans. Sadat took up the fight again and thanks to Soviet arms, gained a first victory. However, he felt the cost of this victory was excessive and it was in order to preserve peace that he thanked the Soviet advisors and established a policy of independence which was of benefit only to the Egyptian people. 

To affect such a reversal of policy, one must be endowed with a great political sense as well as great courage. Sadat has already entered into history as one of the most authentic political geniuses that Egypt has known. 

I shall never forget with what generosity he welcomed me under dramatic circumstances. It is with fervor that I pray for this great Egyptian and his people.

As for King Hussein of Jordan, I can never praise him enough. To me, he is not only a friend but a brother. He is a strong, good-hearted man in whom great courage unites with a deep love for his country. In numerous instances, Jordan's Hussein has known how to face adversity, and he has done so with resolution. Jordan's strategic position has been independently preserved through King Hussein's political wisdom and decisive spirit. I remember an incident which is illustrative of his valor. A coup d'état, fomented by Gamal Abdel Nasser and led by a Jordanian general, took place in Jordan. There was an uprising in one of the garrisons. King Hussein went alone into the insurgent barracks and addressed the troops firmly and reasonably. When he finished speaking, the soldiers cheered and threw themselves at his feet. 

As a crown prince my friend Hassan II proved his patriotism. Certainly he must possess the baraka to have been able to escape different assassination attempts and to have succeeded in his famous Green March into the middle of the recovered Sahara. 

He is a sovereign with a rare intellectual elegance. At once descendant of the Holy Prophet and doctor of law from the University of Bordeaux, he is a perfect incarnation of two cultures, the Koranic and the European. 

It is superfluous to add that I pray for him and his faithful people. Words fail me to express here my thanks for the attitude he displayed toward me and my family after the sinister events of January 1979. For his part, it was a challenge to all those who have forgotten the very exalted lessons of the Holy Prophet. 


As for our relations with Western countries, Iran placed itself ideologically squarely in the camp of the Western democracies. My country has for many years enjoyed excellent relations with France, which has never shown any imperialist tendencies toward us and his contributed much to our culture and to our economic development. It was as a very young sovereign that I first met General Charles De Gaulle. I was immediately captivated by his extraordinary personality. In listening to him speak of France, I heard echoes of the hopes and dreams which I nurtured for my own country. He had an eloquence that made his faith in the future of his country contagious. 

Naturally we were to meet many times either in France or in Iran. Each time our bonds of friendship were strengthened. These ties were furthered by our ongoing correspondence. Upon his death, I attended the funeral services at Notre Dame and was later received at Colombey by Madame De Gaulle. I was deeply touched by this attention to me during those painful hours. 

This ardent patriot was a guide for me as a sovereign. As the measure of his greatness, one has only to compare the conditions of France in 1956 with its conditions at the time he retired from public office. 

Our relationship with France was exceptionally strong. However, our economic exchange with the other countries of the OECD, in particular the U.S., was considerable and constant. In addition, we had recently increased our commercial relations with other countries in the Americas. 

I have stated clearly that our foreign policy served only one country, Iran. But it was equally clear to us that our own interests were best served when our immediate neighbors were at peace. That is why we always practiced a good neighbor policy and, to the extent of our means, a policy of mutual assistance. This is also why we maintained a just balance in our relations with the West, the U.S.S.R., eastern European countries, and communist China. 

The profound and genuine solidarity of all the peoples on earth is not a fantasy to me. It is a reality which, in my opinion, should dominate the foreign policy of all nations without exception. I have always aimed with all my country's might toward the achievement of this ideal. 

One needs only a glance at a map of the world to realize that of the 150 countries of the earth only 28 are considered democracies in the West's eyes. The common features among all these countries, whether they are predominantly industrial or agricultural, are a prosperous economy and high standards of living. 

Outside of this “club of the privileged,” the Indian sub-continent, for example, is ruled by governments which are theoretically democratic. But they have done little to remedy the illiteracy of their people, to change poor health care, and to curb civil strife. They refuse to modernize. Their misery stems from a weak economy. 

We are touching here on the heart of one of the gravest dilemmas of our epoch. It weighs on all countries which find themselves outside the “club of the privileged.” The question is to evolve or not evolve, to remain a docile member of the Third World or to attempt to cross the threshold and enter into modern civilization. 

These countries expose themselves to enormous, if not insurmountable, difficulties. They often have only limited means at their disposal, and almost always lack experience and qualified professionals. They must compete on the world market and vie with intransigent industrialized countries. The privileged have never had to cope with such problems during their evolution. They did not encounter fearsome competitors and could therefore peacefully prepare for democracy. They built solid economies based on their scientific and technological capability. Today's developing countries are not in this enviable position. 

It seems that in our epoch no country can pretend to political independence if it lacks a solid economic structure. This is a condition sine gua non. For today it is economic power that guarantees freedom and sovereignty for a nation, and furnishes it with the conditions for a true democracy. 

Thus, in 1973, I proposed that the twelve industrialized countries join the twelve members of OPEC in creating an International Fund for Aid, each contributing 150 million dollars to it. The council of this Fund would include, besides the 24 founding nations, twelve representatives from Third World nations. The members together would study projects submitted by the developing countries. These plans would aim towards the gradual economic independence of the aided countries. 

The World Bank and the International Monetary Fund were to serve as advisors and provide facilities for exchange and financing, agreeing to loans on the basis of 20 years at 2.5 percent a year. Robert McNamara and the president of the International Monetary Fund favored the project. 

This program provided a framework in which an international bank for development would aid any Third World country which was in difficulty. This organization would have to be absolutely neutral, independent, politically free from obligation, and open to everyone. 

The number of OPEC countries has grown and my earlier proposal should be financially increased. The rises in the price of oil should also be taken into account. 

My initial proposal on contributions would have produced three billion dollars. It would be fitting today to collect twenty billion dollars. 

The Fund would have played the role of an international cooperative, or better yet the role of an economic United Nations with executive power. 

These proposals for international economic cooperation had been made in the framework of a tentative global solution for the problems of energy and as such were taken up in an integral fashion in the article in Le Monde Diplomatique already cited.\footnote{see last page of \cref{ch08}.} 

Without doubt this project was bold, but it was not chimerical. It made a direct appeal to international solidarity and could have constituted an element of politico-economic stabilization of great efficacy. 

I had many conversations with Valery Giscard d'Estaing, the president of the French Republic, concerning this conference. Our views were in absolute agreement. We agreed that the great politico-economic problems, and particularly those of energy, had to be reconsidered in their totality on a world-wide scale so that international solutions could be found for them. World peace can only be gained at such a price. 

I am happy to see that a similar program is finally being given serious attention under UN auspices by the Brandt Commission. This endeavor is long overdue and I pray that Mr. Brandt's efforts will meet with success. 


I should not end this chapter without a few words about our armament policies. I have often repeated that Iran harbored no expansionist notions. We had no wish to impose our political, social, or economic concepts on anyone. Our actions in the United Nations were always in the interests of moderation, conciliation, and peace. 

However, our policy of strict independence made military strength a necessity. This need had been graphically illustrated throughout Iran's ancient and modern history. When our armed forces were weak, our nation was overrun. When we were strong, our nation was saved from foreign invasion, Often military might alone had been our sole guarantee of survival. 

My father, Reza Shah, was keenly aware of this and of the necessity for national armed forces. He knew that to control the rebellious tribal chieftains in the outlying provinces a strong, centrally controlled army was needed. Without domestic peace, Iran could never survive, much less hope to modernize and industrialize. 

In 1920 my father took the command of the Persian Cossacks out of the hands of the Russians and began to build our modern military forces. Reza Shah focused most of his attentions on shoring up our domestic security. At that time “Britannia ruled the waves,” and thereby protected our nation from foreign aggressors. In the late forties and early fifties, the U.S. under President Truman bolstered our defense system through their Mutual Security Program. On July 25, 1949, President Truman in a special Message to Congress stated: “In Iran the use of surpluses of U.S. military equipment has aided in improving the defensive effectiveness of the Iranian Army....It is now necessary to provide certain additional items... to strengthen the ability of Iran to defend its independence.” In all our discussions I always found Truman to be a straightforward, honest statesman. 

Under President Eisenhower, limited military aid was continued through the Mutual Security Program. And in the early sixties, President Kennedy, too, acknowledged the strategic importance of Iran to the West. On a visit to the U.S. in April 1962, Kennedy stated that he considered me to be“... a vital force in maintaining the independence of [my] country. ...So when we welcome the Shah here we welcome a friend and a very valiant fighter. . ..” Later during that visit he reaffirmed our bilateral agreement concerning the maintenance of independence and territorial integrity in Iran. We also agreed on the necessity for collective security arrangements. 

However, in January 1968, Great Britain announced that it was withdrawing its troops from the Persian Gulf. Three months earlier Mr. Goronwy Roberts, sent by the British Foreign Office, had assured Iran that the British had no intention of leaving the Gulf in the foreseeable future. Their about-face took me by surprise and I can only assume that their prospective entry into the Common Market prompted this reversal. Shortly after, President Nixon declared that the United States would no longer maintain its role as the “world's policeman.” Thus, our security could be assured only through our own efforts. 

That is why in 1971, on the eve of the British departure, we occupied the islands of Tomb and Abu Musa. In Bahrain, where only one-sixth of the population was Iranian, we agreed to self-determination. Bahrain opted for independence. 

By that time Egypt’s Nasser had already succeeded in fomenting dissension in the region, and Libya’s Qadaffi and the PLO’s Arafat continued these incendiary activities. I knew then that as Iran continued to grow and prosper, we would become an increasingly attractive prize for foreign predators of every ilk. 


All these circumstances prevented Iran from remaining complacent. The security of our borders required constant vigilance, not only along the Gulf coastline but also to the East, where we faced possible incursions. Afghanistan, Pakistan, and India have all been subject to domestic and foreign strife, quite apart from the sorties of our neighbor the Soviet Union. 

Our lifeline was and is the Persian Gulf. We have no oil pipeline to the Mediterranean as do Iraq and Saudi Arabia. The stability of the Gulf of Oman and the Indian Ocean were also of vital concern. Defense of the Straits of Hormuz required that the nations on the Arab side remain our friends. Our forces had to be strong enough to prevent these friendly but poorly-armed governments from being overthrown. Guerilla groups could be deterred only if they knew that Iran was prepared to move rapidly and forcefully to protect these nations. 

Outside the Persian Gulf, the sea lanes through the Gulf of Oman, the Arabian Sea, and the Indian Ocean were vulnerable to submarine attack. This aspect of our defense required a substantial investment in our naval capabilities. 

Many times I have voiced my firm intention to avoid the use and even the possession of nuclear armaments. Our conventionally armed forces would have been among the finest in the world by 1982. These forces would not only have safeguarded our interests in the Persian Gulf but would have preserved the stability and peace of the Indian Ocean as well. 

How much armament is sufficient to assure such security? For the historical record, a detailed list of the armaments Iran would have possessed by 1982 are given in Appendix 4. Our military projects were never kept secret. Quite the contrary, they were well known to all. President Carter had reiterated his support of our endeavors at our meetings in Teheran in December 1977, when he had called our nation an "island of stability” in a very troubled part of the world. Iran truly was the only nation capable of maintaining peace and stability in the Mideast. My departure has changed all that. The Soviet invasion of Afghanistan and the terrorist attack in Mecca demonstrate this all too well. 

Our assemblage of a formidable military force in the Mideast has resulted in charges of megalomania and of careless spending of Iran's money on arms while my people are deprived of basic needs. 

The question of the adequacy of our military forces is subjective. To my knowledge, no military leader of world stature has criticized my arms policy as excessive. As for robbing the Iranian people of their living essentials in order to pay for armaments, nothing could be further from the truth. After paying for these armaments, Iran had a reserve of \$12 billion in foreign currency. 