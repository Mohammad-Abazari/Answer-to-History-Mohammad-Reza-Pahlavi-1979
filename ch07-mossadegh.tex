THE AZERBAIJAN AFFAIR WAS truly a landmark in modern Middle East history. It was in Azerbaijan that the post-war intentions of Stalinist Russia were first exposed. What the Soviet Union did in Azerbaijan, as related in the dramatic debates of the United Nations Security Council, shocked the Free World. It was then that free men everywhere first began to awake to the threat of Communist imperialism. 

I think it is fair to say that the cold war really began in Iran. There were of course signs of it elsewhere as well, but the lines were first clearly drawn here. It was in the course of the Azerbaijan affair that America for the first time in history began to play a leading role in the Middle East. Azerbaijan led straight to the Truman doctrine which saved Greece and Turkey from Communist imperialism. It also paved the way for the later Eisenhower doctrine. 

After the battle for Azerbaijan, Iran felt a resurgence of true nationalist sentiment. All over the country my people seemed to sense the larger meaning of what had happened. Everywhere they went out of their way to express their loyalty to me. 

In 1947 I went to Azerbaijan. My reception in the reunited province--and on my return to Teheran--bordered on hysteria. I shall never be able to forget the enthusiasm and loyalty of the huge crowds that greeted me. The crush of the crowd in Teheran was such that it took us four hours to cover three kilometers. 

The Iranian people celebrated our regained national unity. It was the end of a defensive phase in our history and marked the beginning of a constructive one. Elections in July gave Prime Minister Ahmed Ghavam a solid majority in the Majlis and in October he moved to overcome a major obstacle to our freedom of action: the oil agreement he.had brought back from Moscow in 1946 when he still hoped negotiation and compromise could end the Azerbaijan uprising. 

The agreement called for establishment of a joint Russian-Iranian oil company in which Moscow would retain a controlling 51 percent interest. Fortunately, a 1944 law mandated parliamentary approval of any such accord, allowing us to procrastinate--legally. That Fall, however, Russian patience had run out. The Kremlin pressed hard for passage. On October 22 therefore, Ghavam went before Parliament detailing the agreement and proposing legislation to void it. This frustrated the Russian plan to establish themselves in the north, in the same manner that the British had in the south. The proposal! was approved by an overwhelming 109 to 27 vote, with, of all people, Mohammed Mossadegh leading the opposition to a law designed to strengthen Iranian national interests. 

The law specified that: 
\begin{enumerate}
\item All past discussions and negotiations concerning eventual oil concessions to the Soviet government were declared null and void. 

\item Inthe future the Iranian government would refrain from granting concessions to foreign powers for prospecting or extracting oil, no matter what the circumstances. 

\item Negotiations with the Anglo-Iranian Oil Company over a larger Iranian share of company profits should begin as soon ag possible. 

\item If new oil deposits were discovered within the next three years, Iran would be willing to negotiate oil sales to the Soviet Union. 
\end{enumerate}

Not surprisingly, Moscow was outraged, and brushed the fourth point aside as insignificant. The situation grew so tense that the Russians threatened to break diplomatic relations. I did not take the threat seriously, Moscow does not easily break such ties. My judgment was confirmed dramatically twenty-five years later when the Kremlin did 



not respond to President Sadat's expulsion of 16,000 Soviet military advisors. The danger of renewed military intervention faded. Proclamation of the Truman doctrine projected more American power into the Middle East and extended the U.S. military umbrella over Iran. Finally, in early 1948 we began to receive from the United States “light armaments with a view to safeguarding Iran's threatened security.” 

The stage was set to put Iran on the road to economic growth. Oil production was the key to such development. Despite Tudeh agitation in the oil fields, output expanded steadily: 17 million tons in 1945, 19 million in 1947, 25 million in 1948, with the bulk, of course, sold abroad. I pushed for the drafting of a seven-year development plan. For \$3 million we hired an American firm, Overseas Consultants, Inc., to formulate the proposal. They came up with an ambitious \$656 million investment scheme geared to agriculture and oil. Expenditures were to be divided as follows: 

%\begin{table}
\begin{tabular}{lr}
For the general improvement of social conditions &28.6\%\\ 
Agriculture &25.0\% \\
Transport &23.7\% \\
Industry and mining &14.3\% \\
Petroleum plants &48\% \\
Communications &3.6\% \\
\end{tabular}
%\end{table}


A large part of this budget was to be devoted to hygiene and education. In every province a hospital was planned with between 500 and 700 beds; improvements were to be made in sanitation. Five thousand primary schools, 150 lycées, 26 professional schools, as well as technical centers, were to be built. Three new universities were planned for the provinces. A million children and 175,000 adults per year would be attending schools. 

Our agriculture was to be mechanized. Canals were to be built as well as about ten dams and hydroelectric power stations. In the field of industry, efforts were to be made to develop metallurgy, textiles, cement and brick works, chemicals, and mining. 

More than 3,000 kilometers of new roads, repairs to a further 6,700 kilometers of existing roads, and the building of new railway lines from the capital to Tabriz and from Meshed to Yaza was to improve communications and trading. Our ports on the Gulf and on the Caspian were to be developed, airports were to be built and the postal, telegraphic, and telephonic networks extended. f 

As usual, politics stymied progress. The Ghavam government fell on December 10, 1947, having failed to survive a no-confidence vote in the Majlis. Over the next thirty months five cabinets came and went. Neither prime ministers nor deputies showed much activity. Time was wasted in fruitless debates which did not produce the money needed to begin our development plan. 

In 1948 I made my first trip abroad as a sovereign. I visited Great Britain. King George VI and the royal family showed me such kindness that the trip remains a very happy memory. I had a long conversation with Mr. Bevin, then minister of foreign affairs. As we were speaking of the natural wealth of Iran and I mentioned the region of Kerman, Bevin exclaimed, “Perfect, Kerman! In our zone...” “But, it seems to me,” I told him “that all of Iran is in the zone of free countries.” “It is precisely what I wanted to say,’ Bevin hastened to answer. 

In February 1949 my life was miraculously spared when the mysterious Fakr Arai fired five bullets at me from point-blank range and succeeded only in wounding me in the face and shoulder. Politically, the failed attempt produced positive results: the Tudeh party was outlawed; support for the crown surged. Iranians realized that my death would mean the nation’s sinking into murderous chaos, as it has today. Even the religious community, so long at odds with my dynasty and ultimately so influential in my downfall, rallied to my cause. The most eminent doctors of Koranic law called my survival a “true miracle.” Since my political leverage had improved, I was able to obtain the Majlis approval for implementation of the seven year development plan. But getting started wasn't easy. We didn’t have the money, not even the \$25 million our American planners said was needed in start-up costs. In late 1949, therefore, I traveled to Washington for the first time to plead for increased economic and military aid. I received a friendly reception, but returned home empty-handed. In part, the failure of my mission was our fault. The Americans realized we were not handling our internal affairs with sufficient firmness. The recent collapse of Nationalist China had strengthened U.S. determination to aid only those countries who were willing to clean house. 

When I returned, I set to work with new energy: I dismissed corrupt officials; I began a program of royal land distribution among peasants. Long-standing defects in our 1906 constitution required correction. The constitution's primary aim had been to establish a constitutional monarchy. Unfortunately, many of its provisions had never been implemented. Thus, by the time I assumed power, my government was operating like an oligarchy: No Senate had been established; the members of the Majlis were able to control the elections and perpetuate themselves in power. In 1949, the Senate was created and met for the first time in 1950. We were subsequently to amend the Constitution to provide the king with power to dissolve the Parliament and order new elections. This enabled us to break the power of the oligarchy. In June 1950 I named my long-time chief of staff, General Ali Razmara, Prime Minister, and he negotiated the first agreement for modest U.S. aid. It wasn't much: some Point IV program money, a small Export-Import Bank loan, nowhere near what we needed to implement our plan. Indeed, a shortage of funds brought most plan activities to a halt. Our American economic advisors left for home. Gradually, popular unrest mounted over the stagnating economy, especially as our efforts to negotiate a better royalty split with the Anglo-Iranian Oil Company were prolonged interminably, The stage was being set for Mohammed Mossadegh. Unhappily, Razmara’s Prime Ministership did not work out as well as I had hoped. He could not or would not bring negotiations with the Anglo-Iranian Oil Company to a conclusion. His performance in Parliament was terrible. He failed to articulate clearly the government's positions and was widely perceived as an ineffective parliamentarian. Mossadegh meanwhile, built up his reputation as a spell-binding orator, or more accurately, as a theatrical performer. Had Razmara been sharper, Mossadegh might never have reached the pinnacle he did. Mossadegh, the orator, is difficult to judge as a politician because of the constant contradiction between his words and acts, and because of his sudden changes of mood from elation to depression before one's very eyes. Whatever his opinions, he was always certain of them and expressed them in hysterical speeches marked with tears and sobbing. He had frequent “diplomatic” illnesses and he played out macabre comedies in which he would exclaim "I am dying...” His fainting spells have become a legend but few know that when he lay on the ground in one of his fits and a doctor opened his vest to see what was wrong he found Mossadegh’s hand clutching not his heart but his wallet. He has been compared to Robespierre and to Rienzi, and perhaps most aptly to a character from the Comedia del’ Arte. 

He called for the nationalization of oil time and again, enflaming the hearts of a people long angered by foreign domination of their affairs. This was an issue on which I had no quarrel with him. But I knew that to function effectively, nationalization had to be coupled or preceded by an agreement with the British. The agreement proved elusive, as we have seen. The Anglo-Iranian Oil Company refused to grant us a fifty-fifty royalty split, similar to the American oil companies’ arrangements with Saudi Arabia. 

On March 7, 1951, Prime Minister Razmara was assassinated by a member of Fedayeen Islam, a terrorist group of the extreme right, while attending a religious ceremony in the Great Mosque. I can't prove it, but my impression is that Razmara had the agreement with the Anglo-Iranian Oil Company in his pocket when he died. Even if he had, it was too late, for popular indignation was at the boiling point. 

I named Hussein Ala, an experienced diplomat and government official, as prime minister. Sadly, it was already too late for moderation and experience. By March 20, both houses of Parliament had passed legislation calling for nationalization of the oil industry. Street riots followed demanding immediate seizure of the installations. Hussein Ala’s position rapidly became untenable. Pressure mounted that I name Mossadegh in his stead. On April 28, 1951, I bowed to the inevitable. 

It was not an easy decision. I had long been wary of Mossadegh’s strident nationalism and violent anti-British sentiments, suspecting that both hid pronounced pro-British statements. His earlier refusal to assume the Prime Ministership during the war, and his support of the Soviet oil treaty had done little to enhance his nationalist credentials. However, he did believe that Britain controlled Iran, his anti-British rhetoric notwithstanding. Still, I worried about that rhetoric and its political effects. Mossadegh was 73 years old and had thirsted for power all his life. Now he had it and he wanted it to be absolute. When we discussed his nomination, I urged prudence and moderation, Our path to political and economic independence was full of pitfalls, I said, and we had to beware of moving too fast. He promised to be cautious. His subsequent actions were, of course, anything but cautious. 

Two days after his investiture, Parliament approved a decree seizing the nationalized oil facilities. I approved and signed it, believing Mossadegh would now begin negotiations with the British on a new settlement. This he refused to do for two long years. The initial British reaction was hostile. Paratroopers were ordered to Cyprus. The cruiser H.M.S. Mawritious was even then berthed at Abadan. Rumors abounded of naval activities directed towards the Persian Gulf and troop movements along our frontier with Iraq. 

“You must realize that I will personally lead my soldiers into battle against you if you attack Iran,” I told the British ambassador. My firm stand probably had something to do with the fact that the rumors of war never materialized into action. 

Mossadegh refused to bargain, even after the Anglo-Iranian Oil Co. had accepted the idea of nationalization and was ready to negotiate. In fact, when the British government sent the so-called Stokes mission to Teheran it agreed to the fifty-fifty royalty split that had eluded us for so long. Mossadegh refused the offer, just as he later refused the Harriman mission's proposals, and offers of mediation from Churchill, Truman, the World Bank, Eisenhower, and that of an independent tribunal. 

For years his political program was based on something he called “negative balance,” which ruled out granting any concessions on anything to any foreign power. He hoped to reverse the trend of recent Iranian history, which had been based on the “positive balance,” ie., balancing a concession made to one power with another to a second. The principle was interesting but could not be applied with monomaniac fury to the entire political spectrum. Mossadegh insisted that it could. 

He was convinced that the world could not do without Iranian oil, and he was equally certain that Iran could sell her oil without help from abroad. He was wrong on both counts. In response to Mossadegh’s intransigence, the Anglo-Iranian Oil Company closed its doors. Royalty payments to the Iranian government were cut off and AIOC blocked sales to others of Iranian oil in which it said it had majority holdings. That meant just about all our oil. 

Production at Abadan ground to a halt. Our new organization, the National Iranian Oil Company, had huge oil reserves, but was not equipped to transport or sell them. Small wonder. We didn’t own a single tanker, nor did we have even the rudiments of a marketing organization. The British took the case to the International Court of Justice--where the British judge voted for us and the Russian judge absented himself--but Mossadegh refused to recognize the court's authority, even before it ruled that it had no jurisdiction. The dispute then went to the UN Security Council which also failed to resolve it. Mossadegh's messianic and mystical misrule threatened to plunge Iran back into chaos and poverty. 

By July 1952 I felt that I could no longer support a man leading the country to its downfall. We had sold no oil since nationalization. There was no agreement in sight and the seven-year development plan was all but abandoned. Opposition to Mossadegh was mounting. On July 13 he demanded extraordinary powers, including the War Ministry. I refused. On July 17 Mossadegh resigned. With some misgivings I named Ahmed Ghavam, a former Prime Minister to succeed him. Although he favored taking strong measures against Mossadegh’s left-wing support, he had aged since his earlier government service. He was now an old man, sick, tired, someone who often fell asleep during policy meetings. 

Upon Ghavam’'s appointment, the Tudeh Party and other Mossadegh supporters took to the streets. Mob rule prevailed and Ghavam's government seemed powerless to cope with it. Ghavam then exacerbated the situation with the denunciation of oil nationalization. As the rioting continued, the threat of civil war mounted. I refused to order my troops to fire and was forced to recall Mossadegh and meet his conditions: I named him Prime Minister and Minister of War. 

Unfortunately, from then on Mossadegh found these powers more and more convenient to his personal ambitions. He muzzled the press and arrested newspaper editors. Because some of the members of the National Assembly now had the courage to criticize him, Mossadegh reduced that body to impotence, not only by relying on his plenary powers but also by ordering his followers to stay away from the Assembly, thus depriving it of a quorum. Dissenting legislators were also threatened in their homes and on the streets by Mossadegh’s hoodlums. 

Mossadegh, who had always preached about the danger of depreciating the currency, printed millions of dollars of paper money without any increase in the gold or foreign exchange backing of the currency. He appointed military commanders personally loyal to him, and he allowed--many would say encouraged--the further infiltration of the army by Tudeh communists. He extended martial law. He had Parliament set up a seven-man committee of his followers to study ways and means of curtailing my powers as Commander-in-Chief of the armed forces. The committee actually prepared a detailed report on the subject, which Mossadegh demanded be brought before the whole Parliament. But Parliament refused to pay any attention to the report or to Mossadegh's request; even Kashani and other of Mossadegh’s former supporters in Parliament refused to countenance his behavior. 

Mossadegh dissolved the Supreme Court. He suspended elections for the National Assembly. Angered because some members of the latter body had been brave enough to oppose him, he announced a national referendum to decide if the current National Assembly should be dissolved. Its members could not help recalling that at the opening of that same session Mossadegh had made a little speech in which he said that 80 percent of them were true representatives of the people. 

And for that referendum Mossadegh, the champion of free elections, arranged that those in favor of dissolution and those against it should vote in separate plainly-marked booths! Everyone understood that if a man had the courage to vote against dissolution he would probably be beaten up by Mossadegh’s toughs or by those of the Tudeh--actually the two groups by this time were almost indistinguishable. The results were all that Mossadegh--or Hitler before him--could have desired. Dissolution won by over 99 percent of all votes cast. In one provincial town where the entire population was about 3,000 people, 18,000 votes were announced as favoring dissolution. Both in that and in other towns, it seems that the dead rose up to vote! 

But in another and grimmer sense many of the dead had voted or tried to vote; for in this and other of Mossadegh’s rigged elections literally hundreds of people were killed. By the time of his overthrow, he had had 27 gallows put up on Sepah Square to hang his enemies in public. Some of his intended victims were former members of his own party. 

During all his years in Parliament, Mossadegh had posed as a champion of constitutional principles, representative government, and due process of law. He had railed against the idea of marital law and had eulogized free elections and freedom of the press. 

But now Mossadegh had in a few months abolished the Senate, dissolved the highest court of the land, and claimed a mandate from the people to eliminate the National Assembly. He had stifled the press, in effect abolished free elections, extended martial law, and tried his best to weaken my constitutional position. What had become of our hard-won constitution of 1906? 



According to Mossadegh all of our economic setbacks were part of our fight to free our oil from British domination. He never addressed the fact that the English were still in possession of our oil, uselessly stored, and their possession of it no longer brought us any rent. They simply bought more oil from Iraq and Kuwait, where it cost them less, nine cents per barrel, I believe, compared to thirteen cents for Iranian oil. Thus, Iran lost while Great Britain profited from Mossadegh’s negative equilibrium. It began to appear as if Mossadegh’s real aim was contrary to what he said. (It must be added that Mossadegh was abandoned by his English friends as soon as he ceased to be of any use, which was as soon as a world oil cartel seemed possible without him.) 

As Mossadegh became increasingly entwined in the webs of his own intrigues, he never lost sight of one major aim: the ouster of the Pahlavi dynasty. He had familial ties to the Qajars and had opposed my father's coming to power in 1925. Moreover, he knew that confirmation of his own power depended, in the final analysis, on the loss of mine. 

He had a parliamentary commission study ways of curbing my powers as Commander-in-Chief and was furious when Parliament failed to implement its findings. In February 1953 he attacked more directly, suggesting I leave the country for a while. I felt he should be given the opportunity to implement his own policies and welcomed a respite from his intrigues, so I agreed. The wily Mossadegh then suggested I not leave by air; he felt that crowds protesting my departure might block the airport runway. Instead, he recommended that I drive to the Iraqi border incognito and go from there to Beirut. 

But somehow word of our planned departure reached the people, who immediately took to the streets to demonstrate their loyalty to the crown. Heartened by this show of support, I decided to stay. 

By mid-year the national mood was changing. Old supporters of Mossadegh dropped away as they realized his policies were opening the way to communist domination of Iran rather then removing British influence. The final crisis broke in late July when he tried to dissolve the Majlis and called for new elections. Political chaos mounted rapidly. The Tudeh again dominated the streets. An uncertain Mossadegh pondered proclaiming a republic with himself as president. 

He had already taken more and more power upon himself. A dozen tanks guarded his town house in Teheran while my summer palace at Saadabad had only four, leaving it vulnerable to Tudeh mob attack. 


Queen Soraya and I went alternately to live in my father’s villa on the Caspian Sea near Rasmar and the hunting lodge at Kelardasht. On August 13, 1953, I decided the time for firm action had come. I signed a decree dismissing Mossadegh and appointing General Fazlollah Zahedi as Prime Minister. Zahedi, a former Mossadegh colleague, had fallen into disfavor and was now in hiding. 

I asked Colonel (later General) Nematollah Nassiry, commander of the Imperial Guard, to deliver the messages. His subsequent adventures resembled those of the Three Musketeers. 

First, he had to find Zahedi, a man who had been on the run from Mossadegh’s police for months, spending each night in different “safe” houses. However, Nassiry quickly located him and delivered my order naming him Prime Minister. Zahedi accepted. Mossadegh was next. Nassiry first arrested three of his close advisors to get some ‘‘feel” of his attitude on his dismissal. Soon news of my counter-measures leaked out. Before Nassiry could deliver my message to Mossadegh, communist newspapers hit the streets warning that the Colonel was planning a military coup. At eleven that night Nassiry and two of his officers boldly drove up to Mossadegh’s house. It was surrounded by tanks and guards. The colonel and his aides strode past the gun muzzles of the tanks, confident the troops knew him too well to shoot. He was right. Once inside he demanded a meeting with Mossadegh. His aides refused; however, they agreed to deliver my order of dismissal and obtain a signed receipt. Nassiry waited for an hour and a half; then he was handed the receipt. He glanced at the handwriting. It was Mossadegh’s. Before he could leave the house, Mossadegh’s chief of staff, General Riahi, had him brought to the War Ministry and arrested. 

I had already made contingency plans with the help of my American friends, who in those days included Kermit Roosevelt of the CIA and the U.S. ambassador in Teheran, Lloyd Henderson, We had agreed that should Mossadegh use force to resist his ouster, I would temporarily leave the country. We felt my departure would crystallize the situation by forcing Mossadegh to show his true colors and thus rally public opinion behind the throne. To facilitate matters, we had established special radio communications between the Saadabad palace and my two hideaways at Kelardasht and Ramsar. Thus, when Nassiry’s driver reached the palace with news of his arrest, the message went out, but for some unexplained reason, the transmission was delayed. 


I vividly recall that for two nights I hadn't slept. Well before dawn Mossadegh’s radio came on the air, proclaiming that my plan to supersede him had failed. Only a few minutes later Colonel Nassiry's radio message arrived, telling of his imprisonment. 

We were at the hunting lodge at Kelardasht when this news reached us. The small airstrip there could only accommodate very small planes, so Queen Soraya and I flew first to Ramsar, twenty minutes away. There we boarded a twin-engined Beechcraft for the flight to Iraq. We were accompanied by two aides--the master of my horses, who had insisted on coming with us, and Major Khatami, who later rose to command our air force and married into my family. I was at the controls. Hours later we landed in Baghdad. Our unexpected arrival was cordially received. The king of Iraq and I were old friends. Nevertheless, Mossadegh's foreign minister, Fatemi, cabled instructions to our ambassador in Baghdad to have me arrested. Incredibly, the man tried to follow these instructions but, of course, to no avail. We spent two days in Baghdad visiting the holy shrines, and then took a commercial flight to Rome. I remembered that I kept a personal car at our embassy there. It is perhaps a measure of Mossadegh’s methods that the Charge d'Affaires actually refused to give me the keys to my own automobile. Nevertheless, a trusted embassy employee brought them to me anyway. We stayed in Rome only two days. 

The tide had begun turning on August 18. Anti-Mossadegh newspapers managed to publish my decree naming Zahedi Prime Minister and several statements I had made in Baghdad. Later that day the first anti-government demonstrators took to the streets. Nationalists and soldiers moved to break up Tudeh demonstrations. The prisons were stormed. Colonel Nassiry was freed and quickly took command of the Imperial Guards. The next day, August 19, the anti-Mossadegh tide gathered new force. Crowds stormed government ministries and Radio Teheran. At 2 P.M. the radio broadcast information of the Zahedi government. Fighting now erupted around Mossadegh’s house. Finally, the prime minister's defenders broke ranks. By nightfall Mossadegh, still dressed in his pajamas, fled over the garden wall of his house into the neighboring garden and took refuge in a cellar belonging to the director of postal services. 

I returned to Teheran where I was greeted with popular enthusiasm. Throughout Iran the people were undeniably behind the crown: Before, I had been no more than a hereditary sovereign, but now I had truly been elected by the people. 

In front of his judges, Mossadegh continued to play his part: he was at times pitiable; he fabricated stories and behaved extravagantly. He continued to make a spectacle of himself in front of the international press. I knew that he would certainly be condemned to death, for he had been convicted of treason. I therefore instructed the court not to take into account his actions against me. 

After three years in prison, Mossadegh was freed. He went into retirement on his large estate at Ahmad-Abad to the west of Teheran and died there in 1967. I was unable to prevent the execution of Hossein Fatemi, Mossadegh’s Foreign Minister, because he was a communist. However, I personally have provided for the financial well-being of his family. This support was suspended when Mrs. Fatemi, who had been living in London, recently returned to Iran in support of Khomeini. 

The trials which followed the end of Mossadegh’s rule revealed the events of 1951-53 in a strange light. For instance, it turned out that when Mossadegh took control of the War Ministry in 1951, only 110 officers were members of the Tudeh Party, whereas 640 officers were members by the time he fell in 1953. 

The Communists had planned first to use Mossadegh to topple me. Second, uncovered Tudeh papers revealed that Mossadegh was to be eliminated two weeks after my departure. I have seen postage stamps printed in the name of the People’s Iranian Republic which was then to be proclaimed. The uprising of the masses in my favor took the conspirators by surprise. Aware that the people were not behind them, the Tudeh went underground. It should not be forgotten that Stalin had died a few months before and that the Soviet strategy was to undergo some considerable changes. 

Since there was no doubt that the Russians had supported Tudeh politically and financially, the world media credited Great Britain and the U.S. with financing the overthrow of Mossadegh. But the most accurate documentation proves that at the time of these events the CIA had spent no more than \$60,000. I really do not think that such a sum is enough to make a whole country rise up in a few days. 

In May 1957, President Eisenhower, in an address to the American people, affirmed the threat of communism that my nation faced under Mossadegh. He stated: “Under the courageous leadership of the Shah, the people of Iran met that danger. In their effort to restore economic stability, they received indispensable help from us. ... Iran remains free. And its freedom continues to prove of vital importance to our freedom.” 

It took nearly thirty months for the Iranian people to see Mossadegh as the prototype sorcerer’s apprentice, incapable of controlling or dominating the forces of destruction which he himself had unleashed. 

In the beginning he had served his country well. In his negative way he had crystallized our people's anti-foreign sentiments; with his own interests at heart he had jumped on the bandwagon of xenophobia. Oddly enough, his real usefulness to the country ended with his appointment as Prime Minister. In any country a head of government to be effective must do something positive. Mossadegh--let us hope unintentionally--betrayed the common people of Iran by promising them a better deal and then sabotaging his own promises. The people lived for a time on these promises. Then they realized that no matter how dramatically promises are put, you cannot feed your children upon them. They also saw that their native country was disintegrating before their eyes. So the people, especially the common people, rebelled. Mossadegh left the country ruined and in debt. The damages suffered by our economy amounted to hundreds of millions of dollars and three wasted years. 

We shall see in the following chapter how the oil question was settled. 