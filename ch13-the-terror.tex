THOSE WHO “GOVERN” IN Teheran since my departure have demonstrated their impotence and irresponsibility. Shapour Bakhtiar wanted to govern, but could not. Mehdi Bazargan, the pseudo-prime minister of the so-called Islamic Republic, never governed over anything, for Iran no longer had a constitution or a parliament. He was a chameleon who consented, contradicted, and retracted any and all of his policies to please the mullahs and their advisors, ignoring the needs of our nation and its people. 

On January 23, 1979, Mr. Bazargan declared, “The Islamic Republic which we shall proclaim will not resemble Libya or Saudi Arabia, but rather the Islamic government which existed during the first years of the Caliphate Ali'.” On March 30-31, a pseudo-referendum was organized to put the Islamic Republic to a popular vote. It was a grotesque farce. People over the age of 15 voted. A green ballot meant a vote for the “Islamic Republic” and a red indicated a no-vote. Since this public election was held under the surveillance of the Guards of the Revolution, is it surprising that 98 percent of the voters cast green ballots? The Iranian media announced that the “Islamic Republic” had been voted on by approximately 23 million Iranians. But half of our population, nearly 18 million people, is under 15 years of age. Even if one assumes that everyone voted, that would make at least 5 million votes too many. This republic was declared on April 1, 1979, and Iran thereby returned to the Middle Ages. 

Ironically, Mr. Bazargan, who was the former president of the Association for Human Rights, now presided over a reign of terror. What mockery that these executioners pretend to render justice “in the name of God.” This blind fanaticism has established in our country a reign of terror, folly, and stupidity. 

What is difficult to understand is the Western media's double standard--my government was continually characterized as a tyrannical, repressive regime which trampled on Iranians’ freedoms and liberties, but Khomeini’s government seems to be glorified by some as a new revolution. Since many earlier revolutions have freed the oppressed, I feel that perhaps it is difficult for the Western mind to face the reality that revolution is not always a positive force for mankind. Neither Castros Cuba nor Khomeini’s regime have freed their peoples. Moreover, those international human rights agencies which at one time gave us so much good advice on humanizing our criminal justice system appear to have melted away with the advent of this so-called Islamic Republic. 

In this republic's inquisitional process, the essential accusation remains irrefutable. The prisoners are accused of being “corrupters of the earth.” This does not at all signify, as certain Western reporters seem to think, that the accused have permitted themselves to be corrupted by bribes, nor does it mean that they were accused of betrayal of trust, thievery, or any other peculation. The term “corrupter of the earth” is an expression from the Koran which denotes all individuals whose iniquities, vices, and sins offend God. 

No penal code in the world has employed such a vague expression to define an offense or a crime. It is obvious that one can be a “corrupter of the earth” in the eyes of inquisitors for numerous reasons. The accused is proclaimed ipso facto to be impure before God and from that moment must be removed from the earth which he blemishes. 

The “Islamic Tribunals” disdain the most elementary rights of defense. In the minds of the religious “judges,” the accused are obvious criminals solely because they participated in the political, social, and economic life of Iran during my reign. Those who naively protest their innocence and observe that during all this time of corruption the mullahs also lived very well, only succeed in aggravating their cases. To offer testimony and pleadings before these tribunals is useless. 

At the beginning of February 1979, Bazargan promised that the politically accused “would be judged publicly by regular tribunals” and declared himself to be fiercely opposed to an “expeditious procedure.” A few days later General Nematollah Nassiri, who had been the head of the State Security, was removed by the militia from the prison where he was awaiting trial. Beaten and tortured, he appeared during the evening of February 12 before television cameras, his face swollen, covered with bandages, hardly able to express himself. However, he did speak and in the midst of his executioners, declared that he had not ordered any murders and that he had never received any orders to torture prisoners. The Committee shot him during the early morning hours of February 16 along with three other generals who proclaimed their faithfulness to Iran and their sovereign. 

On February 28 Bazargan threatened to resign if the discretionary powers of the committees were not defined and limited. On March 8, “firm assurances” were given to him on this subject. But in fact arrests, followed by firing squads and summary executions, doubled. Almost all the generals who were division commanders were put to death because they had committed the crime of exercising command under my reign. A senator who was more than a hundred years old was executed because of his fidelity to the monarchy and a number of accused who were more than seventy years old were also killed. Everyone was guilty; everyone was a potential victim. Among the published names of the assassinated were ministers and secretaries of state, diplomats, parliamentarians, governors of provinces, mayors and municipal counsellors, numerous generals and officers from the army, junior officers, ordinary soldiers, police officers and officers of state police, journalists, editors and radio reporters, magistrates, lawyers, religious authorities, doctors, professors, sports figures, and businessmen. All were condemned and received summary executions “in the name of God.” 

Thousands of citizens escaped by fleeing Iran, as did Shapour Bakhtiar. Since February 1979 even some religious authorities have been assassinated and persecuted. Until now the terror exercised by the militia has prevented certain ayatollahs from publicly declaring their severe judgment of the massacres ordered by the so-called Islamic tribunals. During revolutionary torments the vast number of those terrorized follow the ringleader. In Le Monde on March 8, 1979, an “open letter” to Khomeini was published by Bakhtiar’s son, which said in part: 

“You need heads, Mr. Khomeini, and at this hour when I cry out, many have already fallen, through your divine will, without knowing at all exactly what their count of indictment was.... But under what name do you qualify, you who through cassettes sent by intermediaries have sent thousands of fanaticized young people joyfully to their deaths? 

“Do not hope for history to flatter you. You have prevented Iran from attaining its chance for democracy. And you are responsible for too many martyrs to ever become a hero.” 

This letter was written before the militias of the so-called Islamic Revolution had become tightly organized. Many “suspects,” especially in the provinces, could still escape the raids of the vigilantes who provided the “Islamic tribunals” with fodder, or indulged in summary executions followed by pillaging. 

These gangs multiplied during January and February when the insurrectionists stripped the barracks and arsenals. To my knowledge they never returned their arms. The best known and the most powerful of these vigilantes are the “Revolutionary Committees” which have at least ten thousand active militants about whom nothing is known. One such group, the Guardians of the Revolution” is a paramilitary organization founded by Ibrahim Yazdi, the famous advisor at Neauphle-le-Chateau --a peculiar figure who traveled with an American passport and became Bazargan’s deputy and foreign minister. 

During this terrible month of March many other innocent people were “judged” and executed. The executed had no knowledge of the accusations made against them; they had no time to prepare their defense; no lawyer; a trial behind closed doors; anonymous judges-- these were the innovations of so-called Islamic justice. 

On March 11, the speaker for the Bazargan cabinet, Amir Entezam, declared to the press: "The government in general and the Ministry of Justice in particular cannot exercise any control upon the deliberations and decisions of the Islamic revolutionary tribunals.” 

Once more Bazargan threatened to resign. But he remained. On March 16, Bazargan had asked for a suspension of the trial of Amir Abbas Hoveyda, who had been my Prime Minister for thirteen years. This man of rare merit had but one fault: his extreme courage carried him to temerity. 


Here I want to make certain clarifications which seem to me to be necessary. During the fall of 1978, the most diverse methods were employed to accuse the administration. Amir Abbas Hoveyda served as the scapegoat. 

A very clever intrigue was mounted, and Hoveyda, whose frankness and loyalty were known to me, did not perceive the dangers which threatened him personally. Hoveyda was promised a fair trial with all the constitutional and legal guarantees which would have justified him and others who had long been a part of his government. He was not afraid of a just trial. 

Arrested on November 8, 1978, he was freed from his prison by the riots a few months later. Instead of escaping, he went to the house of a friend where he telephoned the Committee in order to inform them of his whereabouts. They returned him to prison where he undertook to prepare a statement in his defense. 

However, his inquisitors refused to let him finish his defense. He was grossly mistreated. Men who had seen him before his reimprisonment said he appeared to have lost more than 40 pounds. At dawn on Thursday, March 15, he was taken before a self-styled “revolutionary tribunal” and condemned to death without any opportunity to defend himself against their charges. When accused of having "battled against God,” he reproached his anonymous judges in these terms: "I did not ever declare war against God--how could I have done so?--I am a believer and have gone to Mecca. If you have decided that I am guilty, then I am so. Do what is necessary. But here we all lived under the same laws, the same system of government.... All our laws were promulgated by the parliament and each person approved it.” 

On the evening of April 7, an executioner killed Amir Abbas Hoveyda with a burst of machine gun fire. But it seems that he was already dying from the ill-treatment he had received. 

When I heard of his execution, I shut myself up for a whole day and prayed. Hoveyda's death had been an assassination. It could not be disguised or ignored, and a real cry of horror and indignation arose from the press in the free world. The governments of the West officially expressed “their emotion, consternation and grave concern.” At the United Nations, Mr. Kurt Waldheim could not but note the “indifference with which the new Iranian authorities have confronted the calls for mercy and justice.” 

Meanwhile the executioners continued their work. At two o'clock in the morning of April 11, a “tribunal” condemned eleven officials to death, after a brief deliberation. They were assassinated half an hour later. The first of these victims was General Hassan Pakravan. His sole “crime” was to have headed SAVAK fifteen years earlier. He had a reputation for probity and goodness. He had often appealed to me on behalf of opponents sentenced by our courts and particularly on behalf of numerous mullahs. Four generals were executed with him. 

The International Commission of Jurists, in Geneva noted that the “Islamic tribunals” which judged and condemned in Iran “deliberately violated the international conventions of the United Nations concerning civil and political rights, conventions which Iran had signed.” 

Mr. Khomeini's response was brief. From Qom on May 4, he declared: “The Revolution must cut the hands of the rotten. ... Blood must be spilled. The more Iran bleeds, the more will the Revolution be victorious.” 

In late February, the committees of the Military Wing of the Revolution had decreed “the complete renovation of the Army.” This meant that a veritable popular army was to be achieved through the elimination of commanders with too well-known backgrounds. From February until June the violent repression continued but in a more organized manner. During this period more than twenty generals, superior officers of the general staff, colonels as well as air force and naval officers, were assassinated. On May 8 in Teheran, following a summary trial, the “Guardians of the Revolution” killed 21 people. This group of victims included political as well as military figures. 

On May 10, General Fazollah Nazimi, who had commanded a brigade of the Iron Guards, was killed. Later “the great burning” took place in Kerman. In a helter-skelter fashion, officers, gypsies, and women were condemned. One hundred fourteen people were found “corrupt.” Assassinations of generals, colonels, battalion commanders, and officers of the state police continued until mid-June. In addition, some 250 officers of the general staff or superiors of the different forces were imprisoned, transferred, or dismissed. In this way, the Army lost its leadership. 

At the beginning of September, the international press published the names of some 575 people officially executed since February 16, by decree of so-called Islamic tribunals. No communist leader was on this casualty list. The press, however, appeared not to have noticed this detail. Khomeini's tribunals assassinated only true believers, “in the name of God.” Victims of guerilla groups and militia bands who roamed the provinces remain unaccounted. These insurgents had stripped our arsenals in early January; they killed and pillaged with impunity in Teheran as well as the provinces. 

At the end of March the “Guardians of the Revolution” had arrested and imprisoned Amir Hossein Atapour, a 78-year-old retired general. His son, Fariborz Atapour, was an editor of the Teheran Journal and wrote a courageous article in which he revealed that “at feast twenty thousand political detainees are stagnating in improvised jails and prisons.” Why are these horrors passed over in silence, ignored by the world? During my reign there were never more than 3,200 so-called political detainees, and most of these were in fact terrorists. Since February, it is certain that many tens of thousands of men and women have been and are being arrested and imprisoned, frequently under inhuman conditions. They have sometimes been beaten or tortured. The number of people who have died in prison is also unknown. 

During my reign, members of the International Red Cross could visit freely all the penitentiaries of the country. Our prisons remained open to all qualified investigators. The lawyer of every detainee had knowledge of the file of the accused and time to prepare his defense and to summon the necessary witnesses. Finally, the condemned could make an appeal and seek an annulment, after which I often exercised my right of pardon. This is no longer the case. These so-called Islamic tribunals are an insult to the exalted principles of the Holy Koran. 

That spring the International Red Cross was forbidden to visit or aid any prisoners. On April 1, Mr. Khomeini alluded to the fate of the innumerable men and women who had been imprisoned. I quote: “All these people should have been killed from the first days instead of crowding the prisons. It is not a question of people who are accused but of criminals. Only those who were notorious criminals were killed and now we judge these people according to documents whereas they should not be judged but killed. It is deplorable to acknowledge to what an extent Westernization rages still among us.” 

On May 13, the Agence France-Presse sent this dispatch: “Teheran The Ayatollah Khomeini has addressed a message to President Giscard d'Estaing to thank ‘his French friends’ for their welcome but deplores the fact that they throw human rights in his face for a few criminals and thieves.” 

I was astounded to learn that Andrew Young, defender of human rights, had declared the author of this barbarism a “saint.” 

And what of the so-called Islamic Constitution which institutionalizes clerical supremacy? Its critical feature is the supreme power it gives to its leader, which parallels the “fuhrer” principle. Why is this ignored? In terms of human rights, this constitution severely limits freedom of speech, assembly, and press. Where are the human rights advocates now? The duly elected Bani Sadr actually sits only at the pleasure of the leader, currently Mr. Khomeini, for under this Islamic Constitution Iran's new führer has the power to suspend the president at any time. 

What is more, this so-called Islamic Revolution, through its medieval methods of public torture and death, is brutalizing its populace. And what can I say about Khomeini's seizure of over 50 innocent Americans on November 4, 1979, in their own embassy? He leads the people into a daily deification of terrorism. 

This so-called “saint” succeeded in bringing Iran to its knees. The very fabric of our nation is in ruins as a result of Khomeini's theocracy, which affords the government almost limitless powers over all aspects of Iranian life. The popular despair his reign of terror and stupidity created can only lead to communism. 
