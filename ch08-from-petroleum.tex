THE DEVELOPMENT OF THE oil industry constitutes the most tumultuous aspect of modern Mideast history. It is an unending series of intrigues, plots, political and economic upsets, acts of terrorism, coups d'etat and bloody revolutions. To understand the upheaval in Iran and other parts of the Middle East, one must understand the politics of oil. 

The world petroleum story is one of the most inhuman known to man: in it, elementary moral and social principles are mocked. If powerful oil trusts no longer despoil and humiliate our country it is not because these predators have become human, but because we have won a hard-fought battle which has been waged since the beginning of the century. 

In 1901 the rights for the exploration, prospecting, exploiting, refining, transporting, and selling of oil in all of Iran, with the exception of five northern provinces, were accorded to William Knox D'Arcy. D'Arcy's company was to refund 16 percent of the benefits to the Iranian government. Our government could not intervene in the company's affairs and only unskilled Iranians were to be employed by the company. 

D’Arcy’s company went through several reorganizations in order to remain financially solvent and in 1909 became the Anglo-Persian Oil Company (later known as the Anglo-Iranian Oil Company [A.I.O.C.] and then as the British Petroleum Company). War clouds had been gathering in Europe, and as early as 1912 the British government had become much concerned about petroleum supplies. Winston Churchill, then First Lord of the Admiralty, recommended that to meet naval needs his government acquire a 51 percent interest in the Anglo-Persian Oil Company. A law to this effect was approved by the British Parliament only six days before the outbreak of war in August 1914. In 1920 the Qajar government worked out a new agreement with Anglo-Persian which represented a slight gain for my country. Interestingly enough, Sir Sydney Armitage-Smith, a British treasury official, acted as negotiator for the Persians. 

In 1933, the Iranian government managed to repeal this agreement. A new accord was concluded which guaranteed somewhat more revenue for Iran. More importantly, the surface area covered by the concession was reduced to a hundred thousand square miles and the company was required to employ Iranians in preference to other nationalities. 

Profits from the Anglo-Iranian had soared. The initial investment of some 100 million dollars had been completely recovered by the beginning of the twenties. Subsequently, according to the most reasonable estimates, the company's income reached twenty-five times this sum. Iran was getting nothing from the prodigious wealth drawn from her soil. It all went to the company's shareholders, of which the largest were the British Admiralty--the Royal Navy was run on Iranian oil--and the British Treasury. So by 1950, Iran had received 45 million dollars of royalties for her own oil, while the A.I.0.C. paid 112 million dollars in income tax to the British government. This gives some idea of the size of the dividends paid to shareholders. 

The company’s attitude was, furthermore, curiously discriminating with regard to our country since it paid higher royalties to other countries, including both Iraq and Kuwait. In addition, the gas extracted with the oil was entirely wasted and burned on the spot. The clauses of the new 1933 agreement were not respected: the company failed to train Iranian technicians and consequently refused to reduce the number of foreign employees; they paid miserable salaries to the Iranians and failed to lodge them decently and in accordance with the agreement. 

Whereas American companies had already signed a 50-50 contract with Saudi Arabia, the royalties paid by the Anglo-Iranian represented less than 30 percent. Finally, a large part of the profits gained at the expense of Iran were invested in prospecting and digging for oil in other countries so that humiliation was added to injustice. 

In May 1951 the law to nationalize the oil industry was ratified. I was one of the most ardent supporters of this nationalization. However, I felt that this act had to be followed or preceded by negotiations. Unfortunately, the very opposite occurred, due to Mossadegh’s recalcitrance. 

Great Britain protested, sent the dispute to the International Court of Justice in The Hague, withdrew its 4,800 technicians from Abadan and decreed an oil blockade of our ports. The National Iranian Oil Company (N.I.O.C.), which was formed immediately after nationalization, could not sell a barrel of oil even at half price due to this blockade. Thus Iranian oil remained in the barrels and the reservoirs for three years, constituting a liability rather than an asset since we had to maintain the closed installations. After Mossadegh was deposed, we were able to resume sensible oil negotiations. 

I had no intention of tampering with the 1951 oil nationalization law. Mossadegh and I had agreed in principle on the nationalization of oil, but had disagreed on its implementation. One of my post-Mossadegh aims was the relaxation of British control of our oil industry. A more conciliatory approach in 1951 might have achieved this. But by 1954 this effort had been immensely complicated by the British stranglehold on our oil supply lines. Only as a result of lengthy negotiations were we able to reach a basic agreement with a Consortium of the eight largest oil companies in the world. Our National Iranian Oil Company, as owner, employed the Consortium as its contract agent for the operation of the fields and sale of the oil. The agreement was valid for 25 years (with an option for three five-year extensions) and gave Iran 50 percent of the profits. At that time, President Eisenhower sent me a letter expressing his personal appreciation for my efforts in resolving the oil problems caused by Mossadegh’s government. 

Through the passage of the Iran Petroleum Act in 1957, we were able to reduce the Consortium’s power in our oil industry. This act allowed more foreign companies into the country and expanded the activities of the N.I.O.C. into every phase of oil production in the country. The Irano-Italian Oil Co., formed with the E.N.I. of Enrico Mattei, and later the Iran Pan-American Oil Company (I.P.A.C.), formed with the Pan American Oil Company, were examples of these arrangements--both concerned new areas for prospecting. 


These arrangements, which provided for a 50-50 participation, accomplished two important goals. First, they enabled N.I.O.C. to share for the first time in the management of our oil fields. Second, they enabled us to obtain a larger share of the profits, since the foreign company's profit was subject to a 50 percent income tax. Iran's true share was 75 percent. These innovative accords changed the course of the oil industry in the Mideast as well as other areas. They paved the way for the exploited nations to gain greater control of their own wealth. 

The great bulk of our oil production, however, was managed by the consortium under our 1954 agreement. In 1958 I undertook to change this. My aim was that N.I.O.C. take control of our oil fields, in fact as well as in name, by assuming full management responsibility. These efforts consumed a fifteen year period--a period in which I would come to understand the perils that await those who tamper with the oil magnates. Finally, I won out--72 years of foreign control of the operation of our oil industry was ended on July 31, 1973. 

On that date the Consortium became in effect a simple buyer of such crude Iranian oil as we wanted to sell them. The N.I.O.C. took command of all activities including running the refinery in Abadan and it was henceforth up to the N.I.O.C. to plan the exploration of geological beds in the interest of the nation. 

From the instant in 1957 when I set Iran on the course of independence, strange happenings began to take place. Deplorable events followed this truly revolutionary step in the history of oil since enormous interests were at stake. 

I believe that Enrico Mattei, the general director of the A.G.I.P., was among the first casualties. When I knew him, engineer Mattei was a man in his fifties with extraordinary dynamism and an admirable knowledge of the petroleum world. He was fully aware of the risks he was taking in defying the great international cartels. And he always said that he did not have time to be afraid. To save time, he constantly traveled by plane or helicopter. His two-motor Marane-Saunier 760 was always piloted by the excellent and very prudent Major Irnerio Bertuzzi. 

On October 27, 1962, at 5:25 P.M., the plane took off from the Sicilian airport of Catania and was to land in Milan at 6:57. William McHale, the head of the Italian service of the American journal Time, accompanied Mattei. At 6:55 the control tower of the airport received a last message from Bertuzzi, who was preparing to land, and nothing after that. 


About 10:00 P.M. it was learned that the plane had fallen in flames near Bascape in the province of Pavia. There were no survivors. The accident was Officially termed as one “due to lack of visibility.” 

I have never believed that Mattei's death was an accident. Earlier that month, during an inspection of the Marane plane, an explosive device had been found hidden in one of the motors. 

From the moment that Iran became the master of its own underground wealth, a systematic campaign of denigration was begun concerning my government and my person in certain of the mass media. It was at this time that I became a despot, an oppressor, a tyrant. Suddenly malicious propaganda became apparent; professional agitators operating under the guise of “student” organizations appeared. This campaign, begun in 1958, reached a peak in 1961. Our White Revolution halted it temporarily. But it was begun anew with greater vigor in 1975 and increased until my departure. 

In response to my call the ministers of OPEC met in Teheran on December 22 and 23, 1973. This assembly decided to raise the price of each barrel of oil from 5.032 to 11.651 dollars. 

Earlier that year during the Arab oil embargo, we had sold oil on the spot market for \$35 a barrel. That told us something: Demand for oil was so strong that price was no object. Oil had been underpriced for far too long. It was time to move firmly and with dispatch. I was also convinced that in the long run the world economy would be healthier when oil sold at a price which would foster exploration of other forms of energy. 

Early in the decade, President Nixon and Secretary of State Kissinger wrote urging that we rescind the announced increases. This correspondence continued through the end of President Ford's administration. 

It was to me an economic aberration, that oil remained much cheaper than Evian mineral water, for oil is a noble product from which some 70,000 different products are derived. Among these, many thousands have been developed in such a way that the cost of petroleum is only a small percentage of the total cost. In my opinion, oil should, therefore, become the primary substance of an increasingly diversified and sophisticated petrochemistry. The use of oil for heat, light, or railroads is a wasteful policy with little thought for the future. This is a philosophy of Apres nous les déluges! 

As I explained in 1973, the sale of oil at its equitable price is in the real interest of all industrialized countries. By raising oil prices in stages to levels which would allow competition from other costly forms of energy, one would achieve a more sophisticated use of petroleum and at the same time an augmentation of the energy reserves of the world. 

A policy of petroleum at its just price requires not only the periodic revision of the price but also cooperation with consumer countries, especially with the Organization of Economic Cooperation and Development (O.E.C.D.), in order to avoid the creation of an inflationary spiral of international prices. Through negotiation one should be able to fix periodically, by common agreement, the price of energy upon which the industry of the future world could be constructed. 

International pressure groups immediately launched a malicious campaign against me through the mass media. I was accused of attempting the disintegration of the West's economy. Not surprisingly, little attention was paid to my full program. They ignored the proposal I had made to the consumer states: to limit their taxes on oil to 100 percent of the purchase price. For, in effect, the public treasuries of the consumer nations have collected a higher tax per gallon than the sale price of the producer countries. 

During my remaining years in the country, the National Iranian Oil Company grew and prospered. Oil revenues topped \$22 billion in 1977. We engaged in building refineries in Asia and Africa and in various offshore ventures. 

Far from being the OPEC price hawk depicted in the West, within the cartel I counseled moderation towards an ordered and rational growth. After 1975-76 I repeatedly attempted to keep oil prices in check and to persuade my partners that price advances should be gradual and fitted to world economic conditions. 

In 1977, for example, I agreed with Western requests to freeze oil prices in 1978. U.S. Secretary of State Cyrus Vance and British Foreign Secretary David Owen approached me during a break in a CENTO meeting in Teheran and asked me to hold the line. Interestingly enough, Vance urged a freeze extending beyond 1978. Owen did not. Confident that North Sea oil would soon make Britain independent of imports and quite probably a net exporter, he was only anxious for a bridging agreement that would take his country--and presumably his Labor government--through the next year. We were alone in a room of my palace during a reception when the request was made. I told them sure, why not? And that was that. 

Iran then was the world’s fourth largest producer, with a daily output of around 6 million barrels, and the second largest exporter. In addition to the National Iranian Oil Company, we had a National Society for Petrochemical Industries with eleven companies and factories. The National Society of Iranian Gas handled drilling and distribution of our huge natural gas reserves--the equivalent of the Soviet Union’s--and continued to look for new deposits. We know that many gas deposits in Iran still wait to be discovered. 


Not surprisingly, many of the ideas I put forward in 1973-74 have now been accepted by major Western institutions. 

In March 1979, the French newspaper Le Monde, which had consistently attacked my policies, published in its “diplomatic” supplement a long study entitled “The Crisis of Energy and the Price of Oil.” This article, coming five years and three months after the Teheran conference, took up nearly all the arguments which I had advanced in favor of a just price for petroleum, arguing that it was a question of necessity which all countries have been too slow to admit. The article pointed to the scandal that 120 billion cubic meters of gas continued to be burned off each year as sheer loss in the world. It established once again the fact that the part appropriated by public treasuries from the price of the sale of oil to consumers is more than the cost of the oil. Finally, it arrived at the same conclusions concerning the necessity for an international agreement. 

In all of these studies there is not a single word, not a single reference to the 1973 O.P.E.C. meeting in Teheran. The author relies first of all on a study made in 1976 by the Continental Oil Company, establishing the fact that the price of a barrel should rise from 24 to 27 dollars (1975) in order to allow the five sources of energy to enter into competition with oil; and secondly on a report made in 1978 by the Rand Corporation for the CIA. This report emphasizes that “the increase of the price of oil up to ‘thirty dollars’ (constant) per barrel would allow the proven oil reserves of the world to be doubled.” 

I should add that at the beginning of August 1978, Mr. James Schlesinger, U.S. Secretary of Energy, declared that the price of oil could rise to 40 or 50 dollars a barrel. 

Thus, my policy, which had been denounced as that of “destabilization,” and “shameless blackmail," when it was proposed, became the only world-wide policy possible. 