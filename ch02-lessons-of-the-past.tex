
OURS IS A VERY old country: the history of Persia goes back into the mists of time. Situated in that part of the Middle East which was the cradle of the great Western civilizations, we find ourselves at the crossroads which unite Europe and Asia, the Indian sub-continent and Africa. Our shores are washed by three seas--the Caspian to the north, the Persian Gulf to the southwest and the Gulf of Oman to the south--and we are only separated by Syria and Iraq from the Mediterranean, which was for centuries the center of the civilized world. This is the strength of our position. It allowed us, during the great moments of our history, to conquer, trade with, influence and civilize neighboring countries. The weakness of our position is that the center of Iran is a vast plateau, on a northwest-southeast axis, with steppes and salt deserts. The plateau is surrounded on all sides by chains of mountains: the Elburz mountains more or less cover the northern frontier, the Zagros mountains lie to the west and the Baluchistan mountains to the southeast. With the exception of a few large towns (Isfahan, Kerman) the center of our country is empty and barren, and the population, activity, wealth and culture are concentrated in the surrounding provinces. That is why throughout the centuries Persia had as its capital one or the other of the great peripheral cities. Before Teheran, there were during the period of the Persian Empire, Susa, Ectabana, Persepolis and Ctesiphon, and during the Safavid period, Tabriz and Ardebil in Azerbaijan and later Isfahan, which, as a large oasis in the center of the country with unique geographical features, was an exception. 

During our greatest eras, the energy, ambition, intelligence and sometimes the firm wisdom of a single leader kept us united. Other epochs were, on the contrary, marked by attacks, launched either openly or not for foreign interests, aggravated by internal complicity more or less conscious and more or less organized. 

A brief review of these legendary events will make it easier to understand the meaning and the repercussions of the main episodes of a history which for the most part remains unknown. 


Under the impetus of two groups of Indo-Europeans, the Medes and the Persians, we initially emerged as the victors of the peoples who for two thousand years had been quarrelling over Mesopotamia. The Achaemenid dynasty (559-330 B.C.) created the greatest Empire yet known--from the Black Sea to Central Asia and from India to Libya. 

It was also the world’s first real empire in which one ruler governed many different peoples. In order to make this innovation possible, the Persians divided the Empire into provinces, each under a “satrap,” who was a provincial governor. They used signal stations and even a visual telegraph system: semaphor beacons erected on mountain peaks which allowed messages to be passed rapidly from one end of the Empire to the other. They also worked out monetary systems and public finance, and standardized weights and measures. Thus, Persia taught the ancient world how to govern and administer a vast empire. Rome merely imitated Persia and frequently copied her methods. 

The founder of this Empire, Cyrus, deserves to be called “The Great” because he founded it on tolerance and justice. As a conqueror he should be considered the first advocate of human rights. He was the first in the ancient world to publish a charter which can be described as liberal: it freed prisoners of war and left them their land; conquered nations maintained their rights and their customs, for their laws and their religion were respected by the central power. Cyrus not only pardoned his valiant enemies, but he did not hesitate to entrust them with important responsibilities. Thus he can be deemed the liberator of peoples. This policy conformed to the Persian character. It was the policy of all sovereigns for whom peace made it possible to institute a moral order: Persia became known as a land where the persecuted might take asylum. 

Cyrus II the Great, Darius and Xerxes are our hero-kings; they continued their march in works of literature and in the fine arts. Europeans have, however, mostly been taught that Darius was beaten at Marathon and Xerxes in a sea battle at Salamis (480 B.c.) These moving victories by a lesser power over a greater, frequently lead people to forget that Persia later became mistress of the Aegean Sea (394 B.C). 

The Archaemenid decline ended in a unique phenomenon: Alexander of Macedon (356-323 B.c.). He took possession of the whole of Darius’ empire--with the exception of the Pontus and Chorasmia--and extended it slightly to the northeast by pushing the frontier to the river, Syr Darya. Far from carving up the Empire or exploiting it for the benefit of Greece, he imitated Cyrus, took his place and made Persia his own empire. 

The carving up happened after his death (323 B.C.) and, contrary to what can be read in most Western history books, Persia was not influenced by Greek culture, although we still have a charming little Greek statue in a museum. In fact, Alexander espoused the Persian civilization. This is a phenomenon which recurred with subsequent conquerors: the Persians submitted to them but they preserved their own culture and imposed it on their victors. 

Two hundred and fifty years before Christ, the Parthians imposed the Arsacid dynasty (250 B.c.-A.D. 224) which was to end almost five centuries later with the victory of the Persian Ardashir over Artabanus. The Sassanid dynasty (AD. 224-651) was thus founded in Opposition to the Parthians and to Rome. Ardashir had been the guardian of the temple of Zoroaster. The empire which he founded stretched from the Indus and the Syr Darya to the southern shores of the Persian Gulf. It played two important roles in world history--one political and the other cultural. 

The Persian Empire was the first barrier against the savage or half-savage nomads who swept in from the steppes or the Asian mountains. The Scythians, the white Huns, the Seljugs, and the Ottomans were contained for centuries at the price of Persian blood. The Indo-Europeans of the Eastern Roman Empire showed no gratitude. 

They thought only of profiting from our difficulties in the East in order to gain advantage in the conflict which for thousands of years had pitted us against the principal Mediterranean power of the day. 

When the Persian barrier was broken, a void was created at this meeting place of two worlds. Into this void swept Arabs from the West, then Turks and Mongols from the East. Thus, the history of eastern and western Europe, of Russia, of North Africa and of India was irrevocably altered. 

As for the Sassanid Renaissance, like the European one 1,200 years later, it was a synthesis. Shapur I (AD. 241-272) had, it is said, ordered the collection and translation of all religious, philosophical, astronomical and medical texts existing not only in the Byzantine Empire, but in India. When it is remembered that it was thanks to so-called Arab translations that Europe, from the twelfth century was to regain her knowledge of the great Greek texts, it can perhaps be claimed that there would have been no European Renaissance--or that it would have been quite different--without the work and much earlier example of the Persians, which the Arabs copied with such brilliance. 


In A.D. 642 Iran suffered an Arab invasion followed by a foreign domination which should have destroyed her: the country was ruled for several centuries by the Caliphs of Baghdad. Now, as was the case with the Greeks, the conquered overcame the conquerors. 

At first the Persians asserted their originality and independence by refusing Sunni doctrine and by developing the Shiite doctrine. Politically, this meant the refusal to recognize the spiritual sovereignty of the hereditary Caliphs of Baghdad; a vanquished and occupied country has nothing to call its own but its inner life. 

In the political order the decisive factor in the recovery of our independence was the victorious action of an Abbasid emissary (the Abbasids were descendants of Abbas, the Prophet's uncle), Abu Muslem Khorasan. From A.D. 745-750, helped by an army with a permanent majority of Iranians, he liberated Khorasan and took possession of what is called Iraq today. Thanks to him the Abbasid dynasty succeeded that of the Umayyads in Baghdad. 

The sciences and Persian art had moved eastwards to the province of Khorasan where they developed a hitherto unknown vigor and splendor. Nishapur under the Tahirids, and Samarkand and Bukhara under the Samanids became Irano-Islamic cultural centers. It was the golden age of Persian poetry, and notable poets were Ferdowsi (c. A.D.935-1020), the prince of epic verse, and the mystics Sana'i of Ghazna and Jalal ad-din Rumi (who died in 1273). Medicine and philosophy flourished under Rhazes and Avicenna. 

The breaking-up process began under the Mamelukes. The Mongol invasion merely completed the decomposition which was already well advanced, and radically destroyed the Empire. Nothing could disguise the disastrous effects of a conquest which was as brutal as it was inhumane. Genghis Khan and Hulagu destroyed most of the towns, particularly in Khorasan, and killed their populations: several million Iranians were massacred. And so the organizations which maintained Irano-Islamic cultural traditions were destroyed, and the nomadism, so much at odds with the true spirit of Persia, was intensified. 

Finally, from 1383, the little that remained of urban culture was swept away by Timor Tamerlane. Historians were to cite the horrible obelisk which he had erected in Baghdad with ninety-thousand severed heads. He spared a few artisans whom he deported to Samarkand with orders to embellish it. 