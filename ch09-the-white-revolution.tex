IN THE SECOND YEAR of my reign, when war was still raging and we were faced with acute problems, I declared a five-point program to assure the minimum needs of every citizen: health, food, clothing, housing and education for all. 

Since then, at every opportune moment, I have reiterated these aims and stressed the necessity of implementing them. If our nation wished to remain in the circle of dynamic, progressive and free nations of the world, it had no alternative but to completely alter the archaic order of society, and to structure its future on a new order compatible with the vision and needs of the day. This required a deep and fundamental revolution which would put an end to injustice, tyranny, exploitation, and reactionary forces which impeded progress. This revolution had to be based on spiritual principles and religious beliefs, and the preservation of individual and social freedoms. 

To realize these goals it was essential that land reform should take place and the feudal landlord-and-peasant system be abolished; that the relationship between workers and employers should be regulated so that labor should not feel exploited; that women--who after all make up half the population--should be treated as equals no longer legally categorized with lunatics and criminals; that the scourge of illiteracy should be removed; that no one should die of disease nor live in misery and wretchedness through lack of treatment or care; that backwardness in the villages should be ended; that the undeveloped outer regions should be connected with the rest of the nation; and in general, that conditions in harmony with today’s civilized world should prevail. 

With these aims in mind, in January 1963 I presented to my people the first stage of my White Revolution.\footnote{See \cref{app01} for full chronological listing of White Revolution Principles.} This program would construct a modern and progressive Iran on sound and strong foundations, so that my presence would no longer affect the destiny of the country. At that time, I received a personal note of congratulations from President Kennedy on my White Revolution. Upon my visit to the United States, in August 1967, President Johnson also declared: “The changes in Iran represent very genuine progress. Through your White Revolution,” Johnson continued, “Iran has risen to the challenge of new times and new generations . . . without violence, and without any bloodshed.” 

More than half the arable land of Iran belonged to private owners, of whom perhaps not more than thirty (some of them tribal khans) owned 40 villages or more. These landowners, who spent most of their time in Teheran or abroad, rarely resided on their estates, and had little interest in improving the agricultural and social conditions of the peasants who lived on them. They appointed agents to manage their landholdings. Unfortunately, these agents’ primary aim was their own financial gain. Thus, the peasants who worked these lands were often exploited. 

In order to change this land ownership system, in 1950 I issued a decree which would have distributed the Crown estates--more than 2,000 hamlets and villages that belonged personally to me--to the peasants. However, the Mossadegh government opposed any changes in the existing system of land tenure--Mossadegh himself was a large landowner. He stopped this distribution of Crown lands, and the program could only be implemented after his fall. By the time the program was completed, toward the end of 1958, more than 500,000 acres of Crown lands were shared among more than 25,000 farmers. In spite of these efforts, the main task, that of breaking up the big private estates, remained uncompleted. Finally, in January 1963, a new law was passed by plebiscite which limited private ownership of arable land. This was the First Principle of my White Revolution. 


Agrarian reform was carried out in three stages. First, no landowner could own more than one village. Peasants who worked the land had the right to buy the surplus with loans repayable over 15 years. Landowners were paid in shares of state-owned industries. 

The second stage of the reform provided that landowners who did not personally cultivate their lands had to rent them for 30 years or to sell them to those who did. 

Finally, in the third stage, landowners who had rented out lands had to share the income with the farmer or sell him the area he cultivated. Large landowners could only keep untilled land suitable for mechanized farming. I am against the exploitation of man by man, but not against the exploitation of machinery by man. To finance our land reforms, we sold shares in government-owned factories. This law, the Third Principle of our White Revolution, not only complemented our land reform program, but enabled the public to participate more fully in Iran's economic affairs. 

Our other methods of financing this land reform and our innovations in agricultural management--our agricultural banks, rural cooperatives, and our corporate farm ventures in which the farmer exchanged land for shares--have been detailed elsewhere.\footnote{See M. R. Pahlavi, White Revolution, translated from the Persian and published by the Imperial Pahlavi Library, 1967, and George Lenczowski, Iran Under the Pahlavis. Hoover Institution Press, Stanford, 1978.}

At the inception of our land reforms that January, I had predicted that the forces of the clergy (the Black reaction) and the communists (the Red destruction) would attempt to sabotage this program: the former, because they wished the nation to remain submerged in abject poverty and injustice; the latter, because their aim was the complete disintegration of the country. 

My prophecy was fulfilled. Widespread sabotage began, accompanied by murder and rioting, the most severe outbreaks of which were the rebellion in the south of the country and the disorders in Teheran in June 1963. 

These revolts were financed by large landowners angered by their failure to block agrarian reform. Although the confounded alliance of the Red and the Black--that is, of the revolutionary left and the most extreme religious reactionaries--had been outlined in Mossadegh's day, in 1963 an organized Islamic Marxist movement still was only a specter.


Only in the late seventies would Iran and my government experience fully the wrath of this unholy alliance. 

The 1963 Teheran riots were inspired by an obscure individual who claimed to be a religious leader, Ruhollah Khomeini. It was certain, however, that he had secret dealings with foreign agents. Later the radio stations run by atheist emigres belonging to the Tudeh Party accorded him the religious title of Ayatollah ("the sign of God”) and praised him to the skies, although he was anything but divinely inspired. These events were, in fact, a repetition of those that had occurred in Khorasan during my father’s reign, following the move to modernize men’s clothing. On that occasion the instigator was an adventurer whom nobody knew and who later was discovered to be a foreigner. Khomeini was neither arrested nor tried for his incendiary activities. He was simply exiled--first to Turkey, then to Iraq. 

The vast majority of the country's religious leaders, its real spiritual leaders, played absolutely no part in these events. The riots were financed by a group of landowners affected by the land reform law. The thugs who took part attacked defenseless women in the streets of Teheran, smashed buses taking children to school, set fire to the public library, destroyed a sports stadium and looted shops. 

By the time the third stage of our agrarian reform had been implemented, there remained very few large landowners. In effect 214 million peasant families had become owners of the land on which they worked. Undeniably, this White Revolution was not to the taste of the large landowners or clergy. However, land tenure reforms could not solve all Iran's agrarian problems. 

Unfortunately, Iran has few forests: 13,000 square miles in the north along the Caspian and the northern slopes of the Alburz Mountains and another 5-7.5 million acres scattered and dispersed in the west and southwest, around the central desert and along the banks of the Sea of Oman. Moreover, only a part of these forests--3.3 million acres--is suitable for forest industries. The rest, ravaged in the past, does not possess immediate economic value. 

From antiquity until the 18th century, Iran was particularly rich in forests of oaks, wild almonds, pistachios, and in the south, conifers. Throughout the ancient history of Persia there is no trace of the private ownership of forests. Islam teaches that streams and rivers, forests and grazing fields, and ponds and marshes can never become private property. It was during “feudal” periods, when the central power was weak or corrupt, that the big landowners seized vast regions that were wooded. I declared in 1963 that: 

“The forest is natural wealth to whose creation and development no person has contributed. Therefore it is only just that it should belong to all the inhabitants of the country.” 

Iran possesses 30,000 square miles of pastures of good or intermediate quality and 38,000 square miles of wooded prairies. Unfortunately these pastures can only provide food for about one half of the existing flocks. Thus, the meadows are exhausted and destroyed rapidly and the animals remain lean. This scarcity of good pasture had produced numerous abuses: renting of land at prohibitive prices, refusal to rent to cattle breeders, and the usage of meadows for more lucrative ends. 

After nationalization, conservation, development and cultivation of the forests became the responsibility of the Organization of Forests of Iran. To rehabilitate the forests which were destroyed, vast areas were prohibited for commercial use, and the making of charcoal was banned. The National Iranian Oil Company opened thousands of centers where fuel, kerosene, and petroleum could be purchased to take the place of charcoal. Thus the forests of the north were sufficient to provide the 200,000 tons of charcoal required for the whole country as well as the million cubic meters of wood that our industries consumed. 

Finally, more than 9 million trees were replanted in 26 regions, creating 70,000 acres of "green belts” around cities and on the borders of the major highways. Numerous national parks were established and 98,000 acres of new forests and 250,000 acres of various types of vegetation and trees were planted to limit the advance of the desert. These constituted the first steps on the way to restoring our dilapidated natural resources. 

As for pastures, they were placed at the disposal of the cattle breeders. The minister of agriculture undertook a program of development of the pastures and multiplication of wells, drinking troughs and shelter. 

In 1968, we focused our attention on the nationalization of our waters, both surface and subterranean. According to the Holy Koran, water belongs to everyone. Unfortunately Iran has always lacked water, While the average annual rainfall for the whole planet is 860 mm., Iran only receives 231 mm.; and depending on whether the year is dry or wet, our water resources vary between 280 and 520 billion cubic meters--an average of 378 billion cubic meters. 


Seventy-three percent of our rainfall is absorbed by our irrigated land or is lost in lakes and seas. One hundred three billion cubic meters on the average is available in rivers and springs. Our annual consumption in 1976 was about 90 percent of this figure. Obviously, just one dry year creates a dangerous water shortage. Two successive dry years mean catastrophe. This fact, together with our rapidly rising population, the extension of agriculture, the development of steel mills and petrochemical industries and finally the increased requirements for electrical energy, mandated a policy of water conservation. 

Before August 1953 five small dams had been constructed. They were followed by eight large ones. Altogether 13 billion cubic meters of water allowed 2 million acres, of which half was new land, to be cultivated and to produce 1,804 MGW of electricity. When I left Iran, five other dams were under construction. Studies were underway on the feasibility of obtaining water contained in underground limestone formations. Our water resources were also to be augmented through desalinization of sea water by means of nuclear reactors on the border of the Persian Gulf. 

I envisioned for the distant future, irrigation of 37.5 million acres in place of 5.6 million today. In order to achieve this end it was necessary to produce more energy. Now, even if it were possible in principle to produce around 10,000 MGW from the water sources of the dams of Iran, we would still not be there! 

From 1963 to 1977, production of electric energy increased from 2.3 billion to more than 20 billion KWH, the capacity of our electric plants from 850 MGW to more than 7,500. To this would soon be added 2,500 MGW to be produced by the two nuclear plants under construction. The final goal was the production of 25,000 MGW of atomic electrical energy. 

If I am accused of neglecting agriculture, it is the reverse which is true. In a country lacking sufficient moisture and humus, the investment necessary to irrigate and enrich the soil was of course considerable and it would have been difficult to do more than we did. 


A similar revolution was to occur in the commercial area--and the wealthy entrepreneurs would decry these reforms as well. The revolution began in 1963 when the Fourth Principle of the White Revolution--profit sharing for workers--was adopted. In its implementation, this law became one of the most progressive of its type in the world. 

Under its terms, employers were obliged to conclude collective agreements that provided:\footnote{During 1976, 530,000 workers in the public and private sectors were paid bonus/profit-sharing benefits of about 12 billion rials--equivalent to one or two months’ salary per person. From the law's inception in 1963 to 1975, the total sum of net profits paid to workers had multiplied by a factor of 128.}
\begin{enumerate}
\item Payment of bonuses based on higher productivity or reductions in Operating costs; 

\item Payment to the workers of a part of the net profit. 
\end{enumerate}
A special bank geared to the needs of salaried workers made low-interest loans of 4 percent a year--for home improvements and debt consolidation. Close cooperation between labor and management was encouraged. 

Three major principles governed our labor legislation: 
\begin{enumerate}
\item Every Iranian has a right to employment. A jobless worker would receive unemployment compensation equivalent to his minimum guaranteed income until employment offices had found him another job; 

\item The minimum wage for unskilled labor went up in direct proportion to the cost of living; 

\item Productivity schedules provided higher pay for higher individual output. 
\end{enumerate}
As important as all these steps were, they remained only a beginning. Our concept of democratic equality included economic equality. Consequently, workers should become co-owners of the factories and workshops which employed them. For, after all, these businesses functioned only thanks to their efforts. 

In August 1975, the Thirteenth Principle of our Revolution was enacted into law. All private units of production which had existed more than five years were required to sell 49 percent of their shares to their own workers and employees. Newly constructed state-owned industries were to distribute 99 percent of their shares to the general public. 

This principle completed the democratic economy and was to constitute a turning point in the industrial, economic and social evolution of Iran. The association of all the working population, with the capital of the major units of production, was to cement the union between owners, technicians and workers. A resurgence of the feudal capitalism which had oppressed our nation at the beginning of the century became impossible. 

At first, of course, we heard cries of “Sacrilege!” from some factory owners. A year later, however, the majority of owners initially affected by this legislation admitted that their enterprises had never before been so profitable. They had to agree that our reforms cost them nothing because the increase in productivity had raised the profits to a level that made the owners’ reduced percentages equal to their previous profits. One hundred fifty-three industrial enterprises, held by a small number of shareholders, had sold their shares to 163,000 workers and peasants. Transactions were underway concerning 320 large firms which were to sell shares estimated at 170 billion rials. 

In every case the purchase of shares was made possible by government loans. The Council for the Expansion of the Ownership of Units of Production provided the necessary credit, which was repaid through 10-year loan plans directly deducted from the dividends of the shares. 

The following figures are indicative of the success of this program: 

Fifteen years ago a worker made on the average 2,000 rials, or less than \$30 a month. No more. In 1978 the salary of a worker without particular qualifications was 10,000 rials, to which must be added 20 percent of the profits of the firm and where the law of August 1975 was vigorously applied, dividends from the acquired shares.\footnote{For example, when I visited a sugar factory near Quchan in 1973, I learned that 80 percent of the workers owned cars and 50 percent employed household help.} Moreover, low cost housing was available to workers. Food was cheap because the five basic food stuffs (meat, sugar, oil, rice and bread) were subsidized throughout Iran, and in addition special fixed-price food stores were set up especially for workers. 

I understand that many of these shops were burned with the arrival of the so-called Islamic Republic. 

The sale of shares of public-sector factories, except basic industries such as copper, steel, coal, oil, railroads and armaments, was scheduled for October 1978. Unfortunately, the rioting and pillaging of the insurgents which began in September of 1978 and continued until my departure stymied the implementation of this plan. 


In spite of these economic advances, a vast gulf still separated the peasants from the large landowners. 


The peasant, being at once poor, ignorant and often illiterate, lived far from the towns where justice was administered. Thus, he had little hope of obtaining justice that would be both equitable and inexpensive. 

The ordinary legal problems of the peasant concerned first of all his plot of land and then perhaps the few differences he might have with other farmers concerning the fields, water, animals or instruments used in his work. The sums involved did not usually amount to more than the value of a cow or a sheep or a few acres of arable land. Now, in order to settle these questions, he had to file a complaint, which meant acquiring the services of a lawyer in town, and then to travel one or more times to the inferior court, often situated at some distance. The summoning procedure also required an appearance at the prefecture, a process of appeal, and a journey to Teheran where the court was located. 

Meanwhile the courts were filled with thousands of these small cases, most of which were not worth the time and cost to the nation. Furthermore, the peasants were not in the habit of solving their differences by legal means. They took recourse to ruse, sometimes to violence, and vengeance, which would aggravate the situation and often end with the intervention of the state police. 

The logical solution seemed to me to be the most simple one: the peasant would not go to the law; instead, the law would go to the peasant, and to his own village. A local villager could better understand and solve the problems of another villager. What a distinguished jurist from the city might grasp only with difficulty from a pile of papers, an old villager on the location could understand perfectly and resolve wisely, for he would know the antecedents of the case and the roots of the litigation. And he spoke the same language as the litigants. 

I felt that the peasants would be quite willing to have local juridical problems settled by persons whom they knew and respected. In this way, the peasant would waste neither time, nor money, nor energy. The burden on the court system and the public treasury would be lightened as well, 

The first House of Equity was installed on an experimental basis in December 1963 in the village of Mehia near Isfahan. It was followed by a hundred others and institutionalized by the Ninth Principle of the White Revolution, in October 1965. At the end of 1977, there were 10,358 Houses serving 19,000 villages throughout the provinces, with a total population of more than ten million. 

The House of Equity was actually a court of village justice. The judges, who were five in number, were chosen for three years by the villagers themselves from among people locally respected. The government did not appoint them. Since the position was honorary, justice was free of charge to the litigants. 

The judges were not bound to complicated or useless procedures, They were free to summon or to visit the litigants, to hear witnesses, proceed to verify their testimony, and obtain the opinion of experts. 

The success of the House of Equity was immediate and spectacular. In 1965 they settled 18,000 cases to the satisfaction of everyone. The statistics, which stop at the end of 1977, indicate that 3 million differences had been resolved. If these cases had been adjudicated through the established judiciary, they would have required years to settle and would have been substantially more costly to the government and the litigants. 

The quality of the justice obtained in the Houses of Equity was outstanding. In the view of numerous legal experts, the decisions of these common magistrates were always well reasoned, just and logical. This performance by simple villagers astonished the most eminent jurists. I myself was less surprised by it. In visiting the peasants, I often posed questions concerning the operations of the Houses of Equity. Without fail, the peasants informed me that their opinions were always solicited and given serious consideration. 

In July 1966 we extended this institution to the cities, where minor differences among the citizenry were not lacking. Councils of Arbitration in the towns were also comprised of five arbitrators chosen for three years by the inhabitants of each quarter. 

An advisor, selected from among acting or retired judges, lawyers or notaries, provided advice in matters of juridical rules if so requested, but the final decision rested with the arbitrators alone. Their function was also honorary and cases were examined free of charge. 

At the end of 1977 we had 283 Councils of Arbitration operating in 203 cities with a total of 12 million inhabitants. Seven hundred fifty thousand cases had been settled in a fashion that gave satisfaction to lawyers and litigants alike. 

Because they were so successful, we decided in 1977 to further extend the jurisdiction of the Houses of Equity and Councils of Arbitration. These two revolutionary institutions inherited the jurisdiction of the tribunals of Iran's inferior courts. 

Was not this utilization--I would dare say enhancement--of popular wisdom at the roots a positive revolutionary procedure? I admit that it was a bold experiment. We placed confidence in men of the people and we asked them to judge through their souls and consciences. They made a duty of it and they judged well. 

We applied two basic principles: participation and decentralization. We thus achieved a popular judicial power. In addition, we demonstrated that there existed a popular elite, both of peasants and of townspeople. This elite is now being decimated but it cannot be annihilated. 


The Corps of Literacy, Health and Development, created by Principles 6,7,8,and 11 of the White Revolution, played essential roles in the renovation of the villages, improvement of hygiene, and the instruction and intellectual development of the rural population. They were composed of male conscripts and women volunteers. 

Since our country throughout this time was growing richer, it became increasingly easy for the government to improve the nation’s infrastructure and increase the number of schools, libraries, hospitals, dispensaries, etc. The work of the three Corps was always in addition to these efforts. These Corps brought progress to the most isolated regions and focused on “details” which the bureaucratic organizations had overlooked. 

The Development and Reconstruction Corps was initially responsible for identifying local problems and then helping to solve them. This required each member to ascertain the geographical, agricultural, economic, social, and cultural conditions of the district. Once determined, it was possible to assess the local problems on the spot. In turn, the Minister of Agriculture was apprised of these evaluations. Next, under the direction of the conscript, a training course in modern agricultural techniques was given. The villagers would sow an experimental field of a thousand square meters with seeds provided by the Ministry. On this land they would learn the best methods of cultivating, harvesting, irrigating, and fertilizing the land, as well as combatting pests. In experimental orchards they learned the correct methods for planting, grafting, pruning, etc. They also learned to improve hygiene, ventilation, and lighting in cowsheds and stables and how to vaccinate their animals. 

Among the services rendered in rural areas were: the construction of roads and water supplies; the creation of health service and medical assistance (where the Health Corps was absent); the construction of public baths, schools and libraries; the installation of electricity; the establishment of postal, telegraph and telephone service; the building of public quarters, cooperative stores, centers for professional education and even banks. 

The function of the Health Corps was to treat ailments, prevent or halt epidemics, and promote good hygienic conditions. In eight years the number of people in rural areas who benefitted from medical service increased from one million to almost eight million. In the course of these endeavors, they dug wells, installed pumps, purified canals and springs, channeled water, and installed baths. 

Their success was so great that in 1974 the Corps was made responsible for all medical and sanitary services in rural districts. Thus the Corps became the Organization of the Health Corps and Centers for Rural Hygiene. 

In 1976 this organization had 1,422 clinics and employed 1,240 doctors, with 400 laboratories and numerous dispensaries existing throughout the rural areas. One of the greatest achievements of the Health Corps was a new confidence of the peasants in “official” medicine. Before, the rural populace had frequented “healers”; but by 1978 they would go directly to the Health Corps clinic. This was an outstanding accomplishment in a country where in earlier days a doctor often had to give women patients an injection through a curtain under the surveillance of her father or husband. 

Like the other two Corps, the purpose of the Literacy Corps was to supplement the traditional efforts of the government. It did not replace them. 

There was in Iran under the Qajars hardly one literate person in a hundred--in a country which had given the world such intellects as al-Farabi, Rhazes, al-Hallaj, Avicenna, the two Ghazzali brothers, Hafiz, Suhrawardi, Omar Khayyam, Firdowsi, Ruzbehan Bagli, and so many others. 

Before 1963 less than 24 percent of children between the ages of 6 and 12 went to school in the provinces. The rest remained illiterate. In the cities, 74 percent attended school. Compulsory education was passed into law in 1943 but the law remained unenforced. Thus, on the national level, 85.1 percent of Iranians were illiterate. In 1956, 4 million children were ready for school, but the schools had room for only 1.7 million. More than 2 million young Iranians would have to live as illiterates. 

I felt that the solution to this intractable problem lay in the heart of every young Iranian. If we asked qualified conscripts to educate illiterates in the villages without schools, could one imagine that they would refuse? No! They agreed, with enthusiasm, and soon gained the respect and esteem of everyone. Many were those who admired our Literacy Corps. 

The results were spectacular: the number of pupils in Literacy Corps schools increased by 692 percent in fifteen years. During the first five years alone, 510,000 boys, 128,000 girls, 250,000 men and 12,000 women attended classes organized in the villages. 

The Literacy Corps also built and repaired schools, mosques, and public baths, dredged subterranean canals, planted trees, organized sports clubs, and installed letter boxes to enable regular mail delivery. 

By 1978 more than 100,000 had served in the Literacy Corps. Many of the conscripts and volunteers became licensed school teachers, while others worked in the village courts of justice. This type of education cost the state, or rather the Iranian community, but one-third of ordinary school costs. 

The three Corps were in effect the soldiers of the revolution--versatile, bold and courageous young men and women who never refused a new challenge. Assigned an area, they did all that was possible to improve local living conditions. 

To me, the future of the Corps of Literacy and Health rested in telecommunications. Three satellites each with three television channels would be put into fixed orbit. Thus, a medical soldier or nurse confronted with a serious illness in a remote village could seek medical advice from the most eminent specialist in Teheran or elsewhere. This specialist could then explain the required treatment. Such a system would have placed the talent and the experience of the greatest professors and the science of the best doctors at the disposal of the poorest and most isolated Iranians. At the same time, a telecommunication program costing at least 30 billion dollars would have permitted any small Iranian village to communicate automatically with the rest of the world. Here again, I wanted Iran to enter into the age of computerized telecommunications. That was in part the significance of the Great Civilization. 


Here is the sum total of our diverse efforts in education for the 
fifteen-year period 1963-1978: 
\begin{center}
    \begin{tabular}{lr}
    \multicolumn{2}{c}{Percentage Increase of Students } \\
    Nursery & 1350\% \\
    Primary School & 506\% \\
    Middle School & 263\% \\
    Secondary School & 331\% \\
    Technical and professional education & 1550\% \\
    Schools of the Literacy Corps & 692\% \\
    \end{tabular}%
\end{center}


The total number of Iranian students increased from 1.5 million in 1963 to more than 10 million in 1978. Credit for this achievement belongs largely to the Literacy Corps, which did more than just instruct. It instilled a thirst for knowledge. 

Thus, enrollment in rural primary schools, which in 1963 represented 39.4 percent of the nation’s total primary enrollment, rose by 1978 to 52.8 percent, roughly equivalent to the percentage of our rural population. The total number of schools in Iran was to multiply by a factor of 3.24. Finally, the education budget had increased in proportions which I doubt many countries have equaled: 

\begin{center}
    \begin{tabular}{l}
    \multicolumn{1}{c}{Education Budget} \\
    Third Development Plan (1963-67)--45 billion rials \\
    Fourth Development Plan (1968-72)--172 billion rials \\
    Fifth Development Plan (1973-77)--551 billion rials \\
    \end{tabular}%
    \end{center}


The budget for the Sixth Plan (1978-82) included 2,500-2,700 billion rials for education. It seems unlikely that this budget will ever be realized, Population trends have indicated that by 1982 we would have had 13.7 million students at school, roughly a 40 percent increase over 1978. What will now happen to these boys and girls? 

To educate was fine. However, we needed to improve the quality of teaching and adapt it to the necessities of the modern life of Iran. This concern for improvement was embodied in the 12th Principle of our White Revolution. 

Its primary aim was to diversify the materials, methods, and techniques of education in response to the real needs of Iranians of today and tomorrow. This diversification was reflected in the greater emphasis placed on the history of Iran, both modern and ancient--the renaissance of the traditions and folklore of Persian culture and the teaching of Persian language and literature, particularly its lovely poetry. Technical and professional education was enhanced. Classical education, starting from the secondary level, was reformed as well. 

We were quick to employ new techniques of teaching (audio-visual, television). Adult education through evening courses and apprenticeships was also provided. The three telecommunication satellites mentioned in connection with the Health Corps were to have made possible a televised education of the highest quality for all our villages. 

Free and obligatory education had been announced in principle by my father. Unfortunately, the means were lacking. We had been able to provide free primary education only in the schools of the Literacy Corps. With the adoption of Principle 15, free education for the first eight years of school was established throughout Iran. 

Education beyond this was free to all students who agreed to a period of government service after completion of their studies. The term of this service was determined by the time spent in government supported study. Thus, one who completed high school was obligated to four years of government service. Of course, one could free oneself from this obligation by paying back the cost of one’s studies, but it seemed to us that requiring the students to serve the state for a few years at the end of their education was more favor than inconvenience, since in this way they were assured of employment and of earning a good livelihood. 

In 1978 7.4 million Iranians were enrolled in our public schools. This figure includes nursery, primary, and professional! schools as well as the free night school and special training we provided illiterate adults. That year we had 185,000 students of both sexes in our own universities. In addition, about 100,000 pursued college study abroad and of these, 50,000 were enrolled in United States colleges. 

We had established eighteen universities and 137 colleges in Iran. Of these, the campuses at Shiraz and Isfahan were designed to stand with the most beautiful in the world. The sites had been carefully chosen and the installations studied in every detail. A third campus had been planned for Hamadan, where instruction was to be carried out in French. 

The majority of our students received not only scholarships but pocket money as well. Born at the end of the 50s, they were unaware of the difficulties which had faced their parents and grandparents. They found it quite natural that the best of facilities should be placed at their disposal. 

Today I have come to realize that the events of 1978-79 are attributable in part to the fact that I moved too rapidly in opening the doors of the universities, without imposing a more severe preliminary selection. The entrance examinations were too easy. Andre Malraux used to tell me that, “it is necessary to have ten thousand knights in a country.’ I believe he was right. Part of my error was my failure to demonstrate to my people that knights can be artisans, workers, and smiths as well as intellectuals and professionals. 

The most urgent task was to prepare our young for the multi-faceted, science-oriented world of tomorrow. We wanted them to be supported by solid spiritual and moral training, but to be armed as well with good technical knowledge. Everything possible was done to facilitate the entry of the young into universities. 

Many fathers who had never left their villages could not imagine in a dream what their sons were offered: namely, an easy life, nothing to do but study in a big city, often in an agreeable group, with libraries and laboratories at their disposal and prospects of trips to foreign countries. 

Some of our students were not prepared to face so many novelties. They lacked the spiritual maturity to confront the apparent ease of their new lives. Sometimes they slid into laziness but most often took to confrontation and disputation. They had received so much without any effort that it appeared natural to them to claim ever more. 

Like spoiled children, these students caused so many confrontations that Iranian universities finally sank into anarchy. Today, regrettably, students dictate the academic program since they “vote” at the age of 15. Many disheartened professors have left or are leaving a country, where authority has at the present become null and void. In Teheran alone, of the 2,000 faculty members of the University, 1,200 have asked for early retirement. 

If the educational system fails to provide our youth with the intellectual and technical tools necessary in the world of today and tomorrow, and if the teaching of modern sciences falters, Iran will sink back among the fifth rate powers. 


I believe that our culture belongs to every Iranian, that it is not the exclusive domain of the privileged. I tried to make it available to all. The Empress was also deeply involved with arts and letters. 

The concept that all things belonging to the past were reactionary, anti-progressive and depasse was widespread among Iran's bourgeois city dwellers. The attitude tended to denigrate Iranian culture and caused our people to neglect the works of art bequeathed by the past. 

To protect and restore the ancient artistic traditions of our people and culture, the Empress helped expand public libraries, organized contests, exhibitions and festivals, and in general stimulated and encouraged new talent. 

Working together, we were able to restore the most beautiful aspects of Persian culture to a place of honor. I have spoken of changes carried out in school programs which brought our classical poetry to eminence. Thanks to television, our old Persian music was saved from neglect, and came to be admired by a great number of people. 

At the same time as our traditional culture was being revived, the boldest and the most avant-garde artists met and performed at our festivals and in our modern theaters. The Empress was a crucial force in these endeavors. 

Architectural problems were serious, because the tendency, as in many other fields, was to imitate the West. Real estate developers erected modern buildings of a style and scale which were often out of harmony with the countryside and the people of Iran. The Empress intervened many times to persuade builders to restore certain quarters of villages rather than to destroy them in order to completely rebuild. However, as all foreigners who have visited our country know, we also innovated bold new styles with very fine results. 


Justice and common sense required that women enjoy the same political rights as men. Hence Principle 5 of the White Revolution reformed the electoral laws and made suffrage in Iran truly universal. 

Before our Revolution, Article 10 of the Iranian Electoral Law read: 

“The following are barred from voting: women; those who are not legally able and are under guardianship; the bankrupt; the insane; beggars and those who earn their livings by dishonorable means; criminals, thieves and other wrong-doers who have violated Islamic laws.” 

This is the mentality of the so-called revolutionaries who have usurped power in Iran. But we who wanted to place the nation on the path to progress could not relegate our mothers, sisters, wives and daughters to the same category as the insane and the criminal. This is not the concept of the Holy Koran! Women’s rights in Islam are considerably greater than is generally known. For example, women always had the right to manage their own fortunes. 

We, who are the heirs of a culture and a civilization which had never considered women as inferior beings, believe that in this area we are acting in accordance with the true spirit of our religion and our country’s past. 

Several thousand years ago, Iranian theology defined two groups of angels, each possessing equal rights. One group consisted of three men, the other of three women. Iran's great religious work of ancient times, the Din Kard, specifies that women have the right to manage their fortunes, to represent their husbands at court, to be the guardian of a child disinherited by its father, to become a judge or arbitrator and even, in certain cases, to fulfill the functions of priests. 

I have never believed that women were diabolical creatures if they showed their faces or arms, or went swimming, or skied or played basketball. If some women wish to live veiled, then that is their choice, but why deprive half of our youth of the healthy pleasure of sports? Some of our clergy do not understand that to emphasize sports is to improve the health of our people--the most precious possession of any nation. 

The use of the chaddor can only inhibit activities that have become commonplace among Iranian women since the White Revolution. Our women have successfully administered public affairs. We have had women ministers and secretaries in the cabinet, an ambassadress, female judges and university professors; women have been elected as heads of municipal councils, deputies and senators. They have also played a vital role in our fight against illiteracy and in the Health Corps. 

Today, under the social order of the so-called Islamic Republic, all this is impossible. When half the population is denied education and ts forced to live in the past, the whole of society suffers the consequences. Our own prophet Ali said: “Raise your children according to your generation.” This return to archaic practices will prevent Iranian women from raising their children for the 21st century! 


Health care and social security were two vital fields in which we lagged a good fifty years behind major western countries. 

Principle 17 of our White Revolution concerned nationwide Social Security and pensions. Social Security was to guarantee the working population against accident, sickness and incapacity to work. It was designed, moreover, to protect every Iranian through various stages of life. Particular attention was paid to the aged, young married couples and families. Gifted youngsters of both sexes were also put under the Social Security umbrella, 

Our Social Security system, in fact, was one of the most progressive in the world, paying pensions in the lower income brackets as high as 100 percent of salaries earned at retirement. Moreover, pensions were increased in direct proportion to the cost of living. 

Principle 16 provided free food for needy mothers and new-born children. This law would set the stage for broadly expanded national health care. We fought epidemics, pushed a vaccination program, environmental improvement and better sanitation. Health passes were issued to all citizens, recording vaccinations and other pertinent health data. 

I have always believed that the protection of public health is the principal objective of a government. For that reason we created a number of institutions slated for health care.\footnote{\cref{app02} contains a list of other national health organizations}

The oldest were the Red Lion and Sun, which dated from my father's reign, the Imperial Organization for Social Services, and the Organization for the Protection of Mothers and the Newly Born. In addition to providing health care and hygiene for handicapped and retarded youth, they taught handicrafts and gave vocational training. They trained nurses and social workers. They surveyed the condition of working women and youth in factories and where possible proposed legal remedies to ameliorate adverse conditions. 

The Empress was preoccupied with many of these institutions, putting her heart and soul into them. I shall cite several examples. 

Although today the cure of leprosy has become commonplace, the re-integration of cured lepers into society continues to pose great difficulties, for the fear of lepers is an ingrained one. Our lepers, once cured, were still ostracized. To solve this dilemma, the Empress offered a new approach: a perfect model village was built for the cured lepers. It caused such envy and curiosity in the surrounding areas that people flocked to see it. When the Empress herself visited the village she was overwhelmed by the affection showered upon her. 

In the course of her efforts to improve and modernize the condition of women, the Empress encountered difficulties born of deeply rooted manners and customs. She always found conciliatory solutions. For example, in Baluchistan where the girls were not allowed to speak to men, the girl volunteers of the three Corps were sent to assist them, while the conscripts looked after the men. 

Yet, despite the opposition of conservative forces, many prejudices were gradually surmounted. In most regions not only did school classes become mixed, not only were women treated by male doctors, but one could see nurses arriving on motorcycles to attend to male patients; one could also see female medical students and social workers on motorcycles. This indeed was a great revolution! 

It was the Empress who encouraged the Organization for the Protection of Children, which reviewed with the help of magistrates, current legislation concerning minors. This organization was particularly active in hospitals, children’s clinics and maternity wards and in teaching the care of infants and family planning. 

All the social organizations which functioned under the aegis of the Empress assisted and supplemented the work of official government institutions. In fact, their popular success was not surprising, for volunteers always put their hearts into their efforts. 


My major charitable endeavor was the Pahlavi Foundation, created in 1958 and charged with social and cultural responsibilities. In 1961 its administrative structure was established and I officially endowed it with sufficient funding to meet its goals. This endowment consisted of personal holdings: land, real estate, hotels, and shares of various enterprises such as the Omran Bank, Melli Insurance, factories, etc. In addition in New York an independent branch of the Foundation was set up in New York City where it purchased a piece of real estate. The anticipated revenues from the rentals of this property would be sufficient to support the works of the New York adjunct of the Foundation. 

Since my position as president was honorary, a director and an administrative council managed the foundation. They were controlled and supervised by a special commission comprised of the prime minister, the Presidents of the Senate and Chamber of Deputies, the president of the Supreme Court, and four individuals respected for their spiritual and moral values. Each year this commission published a report on the activities of the Foundation, which detailed its administrative and financial endeavors. 

The Foundation concentrated most of its activities in the cultural area and university students were its main beneficiaries. By 1977, 13,000 students from deserving families had received scholarships that enabled them to complete their studies at home or abroad, Although we had many State scholarship students, the Pahlavi scholars were considered an elite group. 

The Institute for the Translation and Publication of books became an important branch of the Foundation. In 1977 this Institute had published more than 500 sociological, religious, historical or literary works of international cultural interest, as well as many masterpieces from our own undiscovered literature. Annually, I would reward the winners of the “Best Book Contest” to authors and translators who had been selected by a panel of university professors. Also prizes were awarded to successful primary and secondary school students from poor families. 

The maintenance and repair of numerous mosques and the heating and lighting of many theological seminaries, especially those in Qum, were paid for by the Foundation, which subsidized financially needy religious publications as well. Certain social work activities of the Empress also benefited from the Foundation. 

In 1977, when the rents for 3-4 room apartments in Teheran had skyrocketed to between \$700-\$800 per month, the Foundation undertook the construction of low-rent housing. We hoped this endeavor would force speculators to lower their prices. In its first stage, 6,000 low-income apartments were built by the foundation. Others were to follow, as brick and cement production permitted. 

The creation of the Pahlavi Foundation was for me an act of a religious nature. For the Western reader, I must explain that such donations in Iran are irrevocable and inalienable. Thus, those responsible for the management of a foundation must use the funds at their disposal only for the purposes specified by the donor. While I was in Iran, this principle was never violated. Today, I do not know if this principle is still honored by the new regime. I do not know if the foundation is still financing studies of young Iranians abroad, to cite but one example. 

Due to the confusion in the Western press, I feel obliged to state clearly that I never profited in any manner from the Pahlavi Foundation. Moreover, in November 1978 I donated all my Iranian holdings to the Foundation. I acted out of conviction and therefore, have no reasons for regrets. 


Iran was a developing nation, with a population of 27 million in 1968, 36 million in 1978 and probably 50 million in 1990. This meant that each year we had literally a million new mouths to feed, new housing requirements, and a million new jobs to fill. These issues were further complicated by a massive migration from rural to urban areas. 

In 1956 Teheran had a population of 1.7 million; in 1978 it was 4.5 million. Other major towns showed similar increases: 

    \begin{tabular}{ccc}
          & 1956  & 1976 \\
    Isfahan & 255,000 & 1,000,000 \\
    Tabriz & 300,000 & 900,000 \\
    Meshed & 250,000 & 950,000 \\
    Shiraz & 171,000 & 750,000 \\
    \end{tabular}%

In Iran as elsewhere this migration posed problems of extreme complexity. 

First and foremost was housing for workers which required complex urban reforms. We had to rebuild and convert many residential quarters. This entailed especially strict control of the plans of most real estate owners and builders, who were too often concerned only with profits. Controls were placed on the rapidly rising land prices, and various forms of land speculation were penalized. This was Principle 18 of our White Revolution, adopted in 1977. At the same time, the government embarked on a program of low-cost public housing that utilized private contractors and resources of existing and newly created financial institutions like savings and commercial banks and government-guaranteed loans. Prospective home or apartment buyers were extended low-cost, long-term loans. 

By the end of 1977 some 31,000 government-sponsored housing units had been completed at a cost of 14.4 billion rials (\$200 million). Another 40,000 units were under construction. Plans also called for the construction of worker housing near factories and plants. 


Facilities for use by young people were on our drawing boards, more than 2,000 of them, and included schools, universities, hospitals, sports complexes, hotels and holiday camps. 

Unfortunately, since our infrastructure was weak, our housing construction could not keep pace with the rising population. We were still too dependent on imported materials for building; our port capacities were still inadequate. Thus, ships often waited six months or more to unload their cargoes, Our lack of an adequate highway system to transport unloaded materials inland further exacerbated our housing shortages. 

Another problem stemmed from the general insufficiency of personnel. Although Iran had no unemployment when I left and we employed over one million foreigners and were implementing extensive training programs for Iranians, we still suffered severe skilled labor shortages. The training of our labor force could not keep pace with the advances of our country. 

These weaknesses in our infrastructure played a role in the deterioration of our social and political climate. They contributed to our inflation problem, which we constantly sought to control. I believe these bottlenecks were temporary and that by 1982 Iran would have had the necessary infrastructure to meet its needs. 


Administrative reform of our bureaucracy constituted the 12th Principle of our White Revolution, and was indeed essential. We knew we were attacking a hydra-headed monster, which made its lair in a mountain of paper. 

Ideally, to administer the government is to serve one’s country. To achieve, or even to approach this ideal, the individual official must be imbued with concern for the common good. Our task was to impose a moral and intellectual reform in circles heretofore hostile to any change. Reform of our administration became more urgent as our population increased, our country expanded, and our bureaucracy decentralized. 

All these factors had contributed to the multiplication of public services and the creation of new programs. 

To attack this evil at its roots required Administrative reorganization which could only occur if the attitudes of the workers changed. To better achieve this end, we made the administrative revolution a part of our educational revolution, for its goal was to help Iranians acquire the needed intellectual and moral maturity. 


Our three Corps of Literacy, Health and Development had been conceived in part to minimize bureaucratic interference and to create a new mentality in the country. I hoped that our educational system, revitalized by the Ramsar Charter,\footnote{Concern for improvement was expressed in August 1968 in a document called “The Charter for Educational Revolution” which was drawn up at a teaching congress held at Ramsar.} would impart this new philosophy to Iran's youth. The nation would then see its young men and women step forward as trustworthy public servants with open minds toward the modern world, 

While our educational revolution offered hope for the future, the need for improvement was immediate. A Council for Administrative Reform was assigned to each governmental ministry and was charged with implementing reforms. These improvements included modernization, decentralization and simplification of bureaucratic procedures and the use of computers where possible. These committees did their best, and their work was evidenced by many advances in procedures and methods. 

On March 4, 1974 the Resurgence Party was formed. I believed that representatives of all social levels and all opinions could be gathered together in one party. I thought that through eliminating an opposition party, I could solicit the aid of all capable political personalities without concern for party politics. For the future I saw this organization as a great political and ideological school, able to engender the civic spirit necessary for administrative reform. 

The Resurgence Party would foster the convergence of many essential goals which we were then pursuing through government channels. However, experience was to show that the creation of this party was an error. President Sadat had to suppress the single party in Egypt and return to pluralism. I believe that he was right for the Resurgence Party did not succeed in achieving its objectives--it did not become the conduit of ideas, needs and wishes between the nation and the government. 

During the years 1975 through 1978, the stubborn strength of bureaucratic resistance became obvious. The monster changed form, sometimes it took on a progressive appearance, sometimes an honorable traditionalist dye, but it always resisted reform. 

In 1959 we had created the Organization of Imperial Inspection. It was a modern version of what the ancient Persians had called “the eyes and ears of the King” and operated much as does the Swedish ombudsman. It consisted of a group of people of proven integrity, who were directly responsible to me. They traveled throughout the country incognito to observe what was or was not functioning. Any Iranian could apply to this body to complain of injustice, bureaucratic delays, or irregularities in the use of public funds. 

Lies, extortion, and embezzlement had for so long been the practice that they had become literally endemic to our bureaucracy. In setting up the Organization of Imperial Inspection, I hoped to avoid these abuses by anticipating them, and at least for some officials, the very existence of the Imperial Inspectorate served in lieu of a conscience. 

However, in 1961 our Prime Minister, Ali Amini, dissolved the Organization of Imperial Inspection on the grounds of economy, although its yearly budget was only \$300,000. This decision was a major mistake, not soon enough rectified. On November 7, 1976, the institution was revived under the name of the Imperial Inspection Commission and Hossein Fardoust was appointed as its director. It was made up of representatives from administrative bodies, from the Resurgence Party--that was the time of our one-party system--the Chambers of Commerce, Industry and Mining, and finally from the mass media. The Commission’s function was to study the programs of ministries, monitor their operation and correct serious organizational problems. 

It was a modernized service for the inspection of the affairs of the country. This method of self evaluation seemed to me to be surer and also more impartial than that of the “loyal opposition” on which Western nations must rely. Opponents rarely base their criticism on objective observations. Unfortunately, like many of our later provisions, it hardly had time to bear fruit. 

While actively attempting administrative reform, we were at the same time occupied with decentralization, which of necessity multiplies the bureaucratic cadre. We were fighting against the clock, for with a vastly enlarged bureaucracy, failure in reform would have the gravest consequences. 

Today under the so-called Islamic Republic, the caliber of the government staff is of much less importance, for little remains to administer; and as the nation's wealth diminishes, so does the profit from graft. 

There were two objectives in the White Revolution whose realization would affect speculators and racketeers. They aimed at fighting inflation and speculation (Principle 14) and corruption (Principle 19). These endeavors I knew would set certain elements of the population against me: affluent and powerful people who cared little about the methods employed in obtaining their wealth. 

The 14th Principle was put into effect in August, 1975. At that time we were suffering an inflation of 20\% which endangered our economy and social equilibrium. Existing laws could not control rising prices. From 1975 to 1977 our fight against inflation seemed to be succeeding. The cost of living index was down 5\% and the growth of the money supply had been curbed without hampering economic expansion. But in 1977, consumer prices began to rise again. Price guidelines were not respected. Appeals to major firms to hold the line were ignored. 

We then made a major mistake--we asked student volunteers to work as price controllers. Their excessive zeal, their occasional threats, and their ignorance of commercial realities alienated many retailers. Some of these young controllers were probably simply set on sabotaging our government. 

As a result, retailers often found themselves caught between wholesalers who increasingly demanded higher prices and intransigent hotheaded young people who required small shopkeepers to sell at a loss. Some merchants in the bazaars felt that they were being unjustly wronged and strongly opposed the 14th Principle. 

In order to punish crimes against this principle, special tribunals were instituted which passed judgment without appeal. During August and September of 1977, 8,000 people were tried for price control violations. We made every effort to mete out punishment at the top. Owners of factories and large chain stores were heavily fined; some were imprisoned and others’ licenses were revoked. 

Sanctions were imposed on multi-national foreign companies, union officials, deputy prefects, mayors, and highly placed civil servants. Tons of merchandise stored for speculative purposes were confiscated and sold to consumers at fixed prices. At the same time, new laws were promulgated to regulate the market. Nevertheless, prices continued to rise. 

The 19th Principle was designed to protect our society against corruption, especially influence peddling. Henceforth, all government officials--ministers, governors of provinces, mayors, etc.--were required to disclose their net worth. Stocks had to be exchanged for Treasury Bonds (13 percent interest, tax exempt) or deposited and managed by banks, investment companies or similar organizations for the duration of any official's term of office. 

This Principle was not only part of my government's fight against corruption but also a pillar of Iran's moral and social order. Politics based on the selling of favors and nepotism had cost the country dearly and had to be ended. Employment in the government or in the provinces now was only open to those interested in working and serving the nation honestly. At the same time, it was equally important for highly placed officials to be well paid since low bureaucratic salaries are often a cause of corruption. 

This is my understanding of democracy. Our White Revolution assured everyone of the same economic and social privileges, with the same responsibilities. Therefore, these measures were not at all demagogic; they were taken for the protection of the public. 

Systematic application of the 19th Principle should have given high bureaucrats the prestige necessary for them to set an example for others. In the long run, we would have perhaps succeeded in ending corruption. 

Although corruption exists everywhere, we went to great lengths to free our government from it. On at least two occasions, I personally intervened and told foreign suppliers that we would not tolerate their practices. 

One involved a telecommunications contract with a multi-national concern. At issue were \$12 million in bribes to unauthorized Iranians. When we uncovered this scandal, we insisted that the foreign company repay this \$12 million to the Iranian government. 

A second incident involved an order of 80 F-14 airplanes from Grumman Corp. When we learned that Grumman planned to pay \$28 million to two uninvolved Iranian middlemen, we insisted that the company pay that amount to the government. They did so in the form of spare parts. 

Our attitude was the same in our negotiations with French firms engaged to build our subway system. At the outset, we said "No commissions, and they agreed. 

In mid-1978, I signed a decree regarding “the ethical conduct of the Imperial Family.” By the terms of this fiat all complaints against my family were to be referred to a special commission composed of three judges chosen by the Minister of Justice. This commission did not preclude anyone from bringing complaints regarding members of my family before ordinary courts. Many people warned me that I had thus provided my opposition with a rod to beat me. 

Our domestic politics was based on participation, decentralization, and democratization. Our White Revolution breathed life into these principles. 

The Houses of Equity, village councils, mayors, governors, and provincial councils with extended powers were instruments of political interaction. The participation of labor in management and profit-sharing produced similar results in our economic structure. 

State centralization would continue, of course, for the armed forces, the conduct of foreign affairs, finance, the state and local police, and public education. 

I viewed central control of public education as one means of insuring national unity. Iran is a mosaic of many tribes and nationalities with different cultures and traditions. Teaching the Persian language throughout our country fostered a common bond among all. 

The love of one’s village, the city or province of one’s birth, freed from the yoke of centralism, was not at all incompatible with love for the Iranian nation. On the contrary, it increased this love and devotion. 

Popular participation was compatible and often intermingled with decentralization, which I emphasized during the last years and which I had hoped to expand. 

These concepts required a radical reform of the administration. This is why we focused our efforts on training good civil servants who would be patriots, honest, devoted, and capable of taking initiative. By 1982, three million more people would join our work force. This added force would be comprised of engineers, doctors, technicians and other professionals graduating from our institutions of higher learning. 

The advent of democratization in the sense that I understand it is very difficult without participation or decentralization. I realized Iran's need for democracy early in my reign. 

One day my father told me that he hoped to leave me an empire “whose solid institutions made it capable of existing and practically governing itself by itself.” I was then very young and was hurt by this proposition--which I interpreted to mean a lack of confidence in me. I thought that Reza Shah was expressing doubt about my capacity to govern. 


Then came his abdication and the occupation. During these tribulations I realized that although one may inherit a crown, power can only be earned. Especially in a constitutional monarchy such as ours, power must be obtained with the help of the people. Thus, its only valid use is on behalf of the people. Later, while fighting against Iran’s oligarchy, I saw the creation of truly democratic institutions as an absolute necessity for Iran. 

Democratization at all levels of Iranian life can flourish only under the aegis of a constitutional monarchy. Iran had always been and remains an empire, that is, an assemblage of people whose ethnic character, language, manners, and even religious beliefs are different. (Muslims constitute an imposing majority.) Hence, the necessity of a sovereign who would bring about unity from above in order to realize a true imperial democracy. 

The union of these two words should not be surprising. According to Iran's Constitution, be it that of 1906 or 1950, although the Emperor submits his projects for acceptance by the government, he nevertheless remains a constitutional monarch. He reigns but does not rule. 

Imperial democracy is the gathering of all our ethnic groups under the same democratic standard within our frontiers. It is the union of all our social classes in the struggle for true progress. 