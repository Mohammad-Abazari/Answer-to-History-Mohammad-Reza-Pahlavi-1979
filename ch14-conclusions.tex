IRAN IS PREY TO a counterrevolution whose proclaimed goal is to annihilate all that our White Revolution accomplished. 

Five centuries after the Spanish Inquisition, Iran lives under the terror of its own Torquemada--one far more merciless. The fact is that people condemned by the Spanish Inquisition were spared if they repented. They could offer witnesses in their own behalf, a privilege denied by the Iranian Torquemada. 

Hatred, vengeance, and massacre can never serve the cause of Islam, whose tenets teach justice, goodness, forgiveness, and high morals. Thus, this explosion of hatred, unleashed supposedly in the “name of God,” is an insult to God and to our religion. And this insult, I repeat, risks great wrong to Islam, as the Inquisition once wronged Catholicism. 

For myself, I have always believed that real faith consists in respecting and following the spirit and soul of a religion, not in remaining the prisoner of a sectarian dogmatism. It is not in closing mixed schools, condemning women to wear the veil or to share their married life with another woman while depriving them of the right of divorce accorded their husbands; it is not by bringing women to an inferior condition that the spirit of Islam is served. On the contrary, it is in emancipating women, in giving them all the possibilities of education, in assuring them the highest dignity possible, and in providing them a state of complete equality with men in all fields that the truths of the Koran find expression. 

Is it conceivable for a human being worthy of this name to flog, stone, or cut off the hand of the wrongdoer on the pretext that these punishments were given under the caliphs in the Middle Ages? To combat evil where evil exists, in the purest spirit of Islam, is to educate, to foster goodness, and to pardon. 

The Iranian flag, which does not date from our dynasty, under whose folds millions of Iranians have sacrificed themselves during many centuries, is despised by the leaders of this so-called revolution. The kings who led Persia through triumphs and trials in one of the most beautiful of the histories of the civilized world now stand cursed by the moguls of Qom. This obliteration of our national identity and the cultural and spiritual heritage of Iran is abhorrent. Our identity and heritage are our greatest advantages, the essential foundations from which everything else can be recovered and without which all will be lost. 

Today, our Great Civilization may appear to have died for all time. I believe, however, that like those powerful rivers that disappear underneath the mountains, lost to view, only to emerge later in full force, Persian culture will rise to the surface again, nourished by the values, creations, thought, talent, and effort of the people. From their trials will be reborn both spiritual and material victories. 

Let us not forget that the Iranians who were living at the time of the Holy Prophet's birth were praised by him as those who “searched for the Truth.” I have sincerely sought the truth for myself and for my nation. Under my reign Iranians were not searching after falsehoods. 

My thoughts have never left my country. They remain there. I think of all those compatriots who, under the reign of my father and myself, pulled Iran out of darkness and servitude and transformed it into the great nation it was in 1978. Today, away from our country, I prove my gratitude to them by but a single means, yet one which I believe firmly to be the most powerful of all--prayer. 

I pray for those in the agony of poverty and death. 

I pray for our youth, deceived and misguided. 

I pray for those who suffer in silence. 

I pray for those who are hunted and slandered. 

I pray for those who remain blind to falsehood and deceit. May God enlighten them and remove hate forever from their hearts. O God Almighty, in Whom I have believed all my life, preserve our country and save our people. 
